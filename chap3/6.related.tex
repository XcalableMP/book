\section{Related Work}\label{sec:related}

%-- Coarray Imprementations
The Universicy of Rice has implemented coarray features with their own extension called CAF~2.0.
It is a source-to-source compiler based on the ROSE compiler. GASNet is used as its 
communication layer.
%
Houston University depeloped UH-CAF onto the Open64-base OpenUH compiler. It supports
coarray features defined in the Fortran 2008 standard. As the communication layer,
GASNet and ARMCI can be used selectively.
%
OpenCoarrays~\cite{OpenCo} is an open-source software project. It is an library 
which can be used with GNU Fortran (gfortran) V5.1 or later. It supports coarray features
specified in Fortran 2008 and a part of Fortran 2018.  As the communiation layer,
MPICH and GASNet can be used selectively.
%
In the vendors, Cray and Intel fully and Fujitsu partially support the coarray features
specified in Fortran 2008.


task間はF2018のcoarrayにある。違いは・・・

getのnonblockingは難しい。書式上、獲得したあたいがすぐに使用されることになるため、
nonblockingのrangeを大きく取れない。
putでは完了を遅延させるのにたいし、開始を先行させる技術(prefetching)が考えられる。
Crayはそのためのdirectiveをもつ。

RiceもCrayポインタを使っている。
Crayポインタはalias analysisで性能を落とす。


\section{Future Works}\label{sec:related}

Coarray C++はObject oriented programmingになれた人には自然な実装と感じるかもしれない。
我々の目指すCoarray Cはこれとは違い、HPCプログラムをCで書いている人向けに修正を最小限
にするものである。また、coarrayを抽象性の高い概念で定義するのではなく、
単にアドレスと長さで特定できるメモリの範囲と定義したい。これによって一般の
Cプログラマがキャストや引数渡しを使って同じ領域を目的によって連続配列や多次元配列
や単に先頭アドレスへのvoidポインタなどとして自由に使い回すように扱いたい。
これは美しくはないが、Cの大きなプログラムを抱えている利用者には必要である。
仕様書の定義には曖昧さが残っているので、まだ修正が必要である。
Fortranと同じような性能が出せるもので、かつ、FortranのCoarrayとの互換性を持たせたい。

Non-blocking PUT communication 
正確な実行時判定を行うためには、現在non-blocking PUT通信中のcoarrayのアドレスのrangeと相手のimage番号を
ハッシュテーブルに記憶し、GET通信を行う前にそのrangeとimage番号に重なりがないことをチェックし、
重なりがあったらそこでPUT通信を完了させる、という方法が考えられる。
しかしこれでは、重なりが無い通常のケースでも重なりのチェックのコストが大きい。
コンパイラでの解析と動的なチェックを併用する、正確さが劣ってもより高速な方法が望ましい。

