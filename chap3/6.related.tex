\section{Related Work}\label{sec:related}

%-- Coarray Imprementations
The Universicy of Rice has implemented coarray features with their own extension called CAF~2.0.
It is a source-to-source compiler based on the ROSE compiler. GASNet is used as its 
communication layer.
%
Houston University depeloped UH-CAF onto the Open64-base OpenUH compiler. It supports
coarray features defined in the Fortran 2008 standard. As the communication layer,
GASNet and ARMCI can be used selectively.
%
OpenCoarrays~\cite{OpenCo} is an open-source software project. It is an library 
which can be used with GNU Fortran (gfortran) V5.1 or later. It supports coarray features
specified in Fortran 2008 and a part of Fortran 2018.  As the communiation layer,
MPICH and GASNet can be used selectively.
%
In the vendors, Cray and Intel fully and Fujitsu partially support the coarray features
specified in Fortran 2008.


Coarray C++は

task間はF2018のcoarrayにある。違いは・・・

getのnonblockingは難しい。書式上、獲得したあたいがすぐに使用されることになるため、
nonblockingのrangeを大きく取れない。
putでは完了を遅延させるのにたいし、開始を先行させる技術(prefetching)が考えられる。
Crayはそのためのdirectiveをもつ。

RiceもCrayポインタを使っている。
Crayポインタはalias analysisで性能を落とす。


