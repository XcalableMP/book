\section{Related Work}\label{sec:related}

%-- Coarray Imprementations
The University of Rice has implemented coarray features with their own extension 
called CAF2.0~\cite{Rice}.
It is a source-to-source compiler based on the ROSE compiler. GASNet is used as its 
communication layer.
Similarly to our RS and RA methods, they use the Cray pointer to pass the data 
allocated in C into Fortran.
%
Houston University developed UH-CAF onto the Open64-base OpenUH compiler~\cite{HU}. 
It supports coarray features defined in the Fortran 2008 standard. As the communication 
layer, GASNet and ARMCI can be used selectively.
%
OpenCoarrays is an open-source software project~\cite{OpenCo}. It is an library 
which can be used with GNU Fortran (gfortran) V5.1 or later. It supports coarray features
specified in Fortran 2008 and a part of Fortran 2018.  As the communication layer,
MPICH and GASNet can be used selectively.
%
In the vendors, Cray and Intel fully and Fujitsu partially support the coarray features
specified in Fortran 2008.

In the latest Fortran standard Fortran~2018, a subset of coarrays is called team.
It is similar to the executing images in the term of XMP, but does not affect
the parallel execution among images.

While non-blocking PUT communication is effective, non-blocking GET communication
is difficult to put into practical use because the acquired data is used immediately.
Cray has the directive extension for prefetching remote coarray corresponding to 
the GET communication.

Coarray C++ is a coarray implementation into C++. The coarray features are implemented
with the template library unlike XMP/C based on C language.

