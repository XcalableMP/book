\section{Related Work}\label{sec:related}

%-- Coarray Imprementations
The University of Rice has implemented coarray features with their own extension 
called CAF2.0~\cite{Rice}.
CAF2.0 is a source-to-source compiler based on the ROSE compiler. GASNet is used as its 
communication layer.
Similarly to our RS and RA methods, the Cray pointer is used to pass the data 
allocated in C to Fortran.
%
Houston University developed UH-CAF on the Open64-base OpenUH compiler~\cite{HU}. 
UH-CAF supports the coarray features defined in the Fortran 2008 standard. As the communication 
layer, GASNet and ARMCI can be used selectively.
%
OpenCoarrays is an open-source software project~\cite{OpenCo}. OpenCoarrays is a library 
that can be used with GNU Fortran (gfortran) V5.1 or later and supports the coarray features
specified in Fortran 2008 and a part of Fortran 2018. As the communication layer,
MPICH and GASNet can be used selectively.
%
In the vendors, Cray and Intel fully support and Fujitsu partially supports the coarray features
specified in Fortran 2008.

In the latest Fortran standard, Fortran~2018, a subset of coarrays is referred to as a team.
It is similar to the executing images in the term of XMP, but does not affect
the parallel execution among images.

While non-blocking PUT communication is effective, non-blocking GET communication
is difficult to put into practical use because the acquired data is used immediately.
Cray has the directive extension for prefetching a remote coarray corresponding to 
the GET communication.

Coarray C++ is a coarray implementation in C++. The coarray features are implemented
with the template library, unlike XMP/C, which is based on the C language.

