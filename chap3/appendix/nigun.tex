\subsection{introよりrelated workの方がよいか?}



PGAS言語の一つであるCoarray Fortran(CAF)は,Fortran 2008仕様の一部として採用されたことも追い風となって,近年研究開発が盛んに進められている.フリーのものでは, Houston大学のOpenUHコンパイラを開発基盤としたUH-CAF[5]と,Rice大学のROSEコンパイラを開発基盤としたcaft[6]が有名である.近年リリースされたOpenCoarrayは,GNU gfortranにリンクできるライブラリである.我々の開発するOmni XMPもまた,CAFコンパイラとして利用することができる.ベンダではCrayとIntelが古くから提供しており,近年富士通からもリリースされている.これら3社は,Fortran 2008と2015に含まれるcoarray機能の実装をFortran2003のフル実装よりも先行させたことになる.
Omni XMPは,PCクラスタコンソーシアムのXcalableMP規格部会が制定する並列言語XcalableMP(XMP)のパイロット実装である.XMPは,FortranとCをベースとし,ディレクティブ行の挿入によって並列化を記述するが,Fortran 2008で定義されるcoarray機能も仕様として含んでいる.前者は「逐次プログラムに指示を与えて並列化する」という考え方からグローバルビューと呼ばれ,並列プログラミングが容易にできることを狙う.後者は「個々のノード(イメージ)の挙動を記述する」という考え方でローカルビューと呼ばれ,MPIに匹敵する性能がMPIよりも容易に出せることを狙っている.


