\begin{verbatim}

synmetric memory
リモートのアドレスをローカルで計算できる
これがあるから、片側通信が。。。


下位層:通信libの差異を吸収し、上位層にプリミティブを提供する。
XMP coarray実装のプリミティブi/f
put operation
get operation
- どちらも連続データのみ/nonblocking。
sync memory operation
- 発行したputをすべてremoteに書き込み、getをすべてlocalに読み込むまで待つ。

putにはnonblocking技術:runtimeで十分
getにはprefetch技術:コンパイラ技術


生産性の観点の評価
1. リンク時にstatic配列を集めるところが対MPI片側通信で重要。言語仕様/言語処理系だからできる。単にライブラリ群/ライブラリ呼出しではできない。
2. F90配列記述がMPI_Typeの定義を不要にしている。


klogin7$ which xmpf90
~/Project/OMNI-clone/bin/xmpf90
klogin7$ xmpf90 --version
Version:1.3.1, Git Hash:4a5736e
klogin7$ 


sync memoryがいつ必要か
- ユーザ記述
- getのとき、1subarray参照に対してsyncmemoryを1回にする。
- getの前に、同じremoteアドレスへのputがあればsyncmemoryで待つ


alignmentの問題か
64バイト未満でputのlatencyが高い。
src5, RUN5で改善 ・・・しなかった
メモリ割付けのalignmentの問題ではない。
残るはTofuのputの単位の問題と推測
マスク付き書き込みになる?
この点ではRDMAよりバッファリングが有利


A Source-to-Source Translation of Coarray Fortran with MPI for High Performance
https://dl.acm.org/citation.cfm?id=3155888&dl=ACM&coll=DL

Preliminary Implementation of
Coarray Fortran Translator Based on Omni XcalableMP
https://ieeexplore.ieee.org/document/7306099

\end{verbatim}
