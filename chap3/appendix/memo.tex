\begin{verbatim}

introduction
coarrays in XcalableMP
support F2008 coarrays
image setをtaskに対してmap
natural extension to C
basic implementation and issues
lower-level interface
evaluation
related work
Coarray C++は
task間はF2018のcoarrayにある。違いは・・・
Crayはgetをdirectiveで実装
conclusion

implementation for high performance
static coarrayの高速化
全部まとめて先にallocするコンパイラ技術
定数評価、構造体の大きめな見積り
allocatable coarrayの高速化:lowlevelに合わせて選択
RuntimeLibShare事前に巨大領域を取る
実行時allocのコストが大きいとき
RuntimeLibAllocation実行時allocのコストが小さいとき
contiguouity検出による高速化
次元を跨ぐ連続性抽出
バッファリング
有限サイズ、基礎データから、Nlocal, Nremote, Nbufの関係でアルゴリズム
CommpilerAllocationFortranがallocateしてregisterできるとき

implementation software layer
3種:one sided, collective, atomic. このうちcollectiveとatomicはMPIの機能をそのまま使う。それ以上の工夫はない。one sidedはallocate/freeとput/getと同期。lower-level通信層のバリエーションを吸収するライブラリ層を設けた。階層の図。
次元の概念とcontiguityを上位層で解決するため結果的に必要なインタフェースは少なくなった



\end{verbatim}
