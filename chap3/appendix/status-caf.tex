\begin{verbatim}

1.2  Interoperability with the global-view features (NEW since V1.0)
  Coarray features can be used inside the TASK directive blocks. As default,
  each coarray image is mapped one-to-one to a node of the current executing 
  task. I.e., num_images() returns the number of nodes of the current executing 
  task and this_image() returns each image index in the task.
     There are two directives to change the default rule above. A COARRAY 
  directive corresponding to a coarray declaration changes the image index set 
  of the specified coarray with the one of the specified nodes. An IMAGE 
  directive corresponding to one of a SYNC ALL statement, a SYNC IMAGES 
  statement, a call statement calling CO_SUM, CO_MAX, CO_MIN or CO_BROADCAST 
  changes the current image index set with the one of the specified nodes.
  See the language spacifications [3].

2. Declaration
  Either static or allocatable coarray data objects can be used in the program. 
  Use- and host-associations are available but common- or equivalence-
  association are not allowed in conformity with the Fortran2008 standard.
  Current restrictions against Fortran2008 coarray features:
    * Rank (number of dimensions) of an array may not be more than 7.
    * A coarray cannot be of a derived type nor be a structure component.
    * A coarray cannot be of quadruple precision, i.e., 16-byte real or 32-byte 
      complex.
    * Interface block cannot contains any specification of coarrays. To describe
      explicit interface, host-assocication (with internal procedure) and use-
      association (with module) can be used instead.
    * A pointer component of a derived-type coarray is not allowed.
    * An allocatable component of a derived-type coarray cannnot be referrenced
      as a coindexed object.
    * A derived-type coarray cannot be defined as allocatable.
      
2.1  Static Coarray
  E.g.
      real(8) :: a(100,100)[*], s(1000)[2,2,*]
      integer, save :: n[*], m(3)[4,*]
  The data object is allocated previously before the execution of the user 
  program.  A recursive procedure cannot have a non-allocatable coarray without 
  SAVE attribute.
  Current restrictions against Fortran2008 coarray features:
    * Each lower/upper bound of the shape must be such a simple expression that 
      is an integer constant literal, a simple integer constant expression, or
      a reference of an integer named constant defined with a simple integer 
      constant expression.
    * A coarray cannot be initialized with initialization or with a DATA 
      statement.
    
2.2  Allocatable Coarray
  E.g.
      real(8), allocatable :: a(:,:)[:], s(:)[:]
      integer, allocatable, save :: n[:], m(:)[:,:]
  The data object is allocated with an ALLOCATE statement as follows:
      allocate ( a(100,100)[*], s(1000)[2,2,*] )
  The allocated coarray is deallocated with an explicit DEALLOCATE statement or 
  with an automatic deallocation at the end of the scope of the name unless it 
  has SAVE attribute.
  Current restrictions against fortran2008 coarray features:
    * A scalar coarray cannot be allocatable.
    * An allocatable coarray as a dummy argument cannot be allocated or 
      deallocated inside the procedure.
    
3. Reference and definition of the remote coarrays
  For the performance of communication, it is recommended to use array 
  assignment statements and array expressions of coindexed objects as follows:
      a(:) = b(i,:)[k1] * c(:,j)[k2]    !! getting data from images k1 and k2
      if (this_image(1))  d[k3] = e     !! putting data d on k3 from e
  Current restrictions on the K computer and Fujitsu PRIMEHPC FX10:
    * The coindexed object/variable must be aligned with the 4-byte boundary 
      and the size of the array elements of them must be a multiple of 4 bytes.
  
4. Image control statements
  SYNC ALL, SYNC MEMORY and SYNC IMAGES statements are available.
  Current restrictions against Fortran2008 coarray features:
    * LOCK, UNLOCK, CRITICAL and END CRITICAL statements are not supported.
    * STAT= and ERRMSG= specifiers of image control statements are not 
      supported.
    * ERROR STOP statement is not supported.
    
5. Incrinsic Functions
  Inquire functions NUM_IMAGES, THIS_IMAGE, IMAGE_INDEX, LCOBOUND and LUBOUND
  are supported.
  Current restrictions against Fortran2008 coarray features:
    * ATOMIC_DEFINE and ATOMIC_REF subroutines are not supported.

6. Intrinsic Procedures in Fortran2015
  Argument SOURCE can be a coarray or a non-coarray.
  Intrinsic subroutines CO_BROADCAST, CO_SUM, CO_MAX and CO_MIN can be used 
  only in the following form:
    * CO_BROADCAST with two arguments SOURCE and SOURCE_IMAGE
      E.g.,  call co_broadcast(a(:), image)
    * CO_SUM, CO_MAX and CO_MIN with two arguments SOURCE and RESULT
      E.g.,  call co_max(a, amax)



\end{verbatim}
