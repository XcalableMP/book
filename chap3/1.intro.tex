\section{Introduction}\label{chap:intro}

\pagenumbering{arabic}
\setcounter{page}{1}

Omni XMPは,PCクラスタコンソーシアムのXcalableMP規格部会が制定する並列言語XcalableMP(XMP)の実装である.
XMPは,FortranとCをベースとし,ディレクティブ行の挿入によって並列化を記述するが,
Fortran 2008で定義されるcoarray機能も仕様として含んでいる.
前者は「逐次プログラムに指示を与えて並列化する」という考え方からグローバルビューと呼ばれ,
並列プログラミングが容易にできることを狙う.
後者は「個々のノード(イメージ)の挙動を記述する」という考え方でローカルビューと呼ばれる。
後者は、前者では記述困難な領域のアプリケーションを記述するため、それと、
局所的に性能を出したい部分をMPIよりも容易なプログラミング言語という手段で記述するために導入された。
そのため、XMPの文脈の中で自然な解釈ができて同時に使用できることと、
同時に、MPIに匹敵する高い性能が求められた。

我々は、
XMPコンパイラの一つの機能として、
Fortran 2008仕様で定義されたCoarray機能を実装した。
また、言語仕様をFortranに準じて、C言語ベースのCoarray機能もサポートした。
Fortranで定義されるimageはXMP仕様で定義されるnodeに1対1にmappingされるため、
coarray変数の定義・参照はnode間のput/getの片側通信としてそれぞれ実現される。

To implement one-sided communication, the XcalableMP (XMP) runtime library selects 
one of the {\bf communication libraries}, MPI-3, GASNet, or Fujitsu's native 
interface FJ-RDMA, as specified at build time. 
All coarray data must be {\bf registered} to the communication 
library to be referenced or defined via the communication library.

本章では、coarray機能を高性能で実現するために要求された技術と、その実現方法を示す。
通信ライブラリの特質から、Coarrayデータの割付けは可能な限り利用者プログラムの実行前に
集約すべきである。
片側通信ベースなので、遠隔側のアドレス計算を低コストで実現する工夫が必要となる。
通信が簡単に記述できることから、小粒度の頻繁な通信が起こりがちになるので、
それらをaggregateして高速化する技術が要求される。

これらに対応した実装により、【評価の成果を】


【要修正】
This chapter describes how the compiler and runtime techniques are 
implemented in the XMP compiler with some evaluation. 
In the rest of this chapter, 
Section 2 shows the basic implementation for coarray features, 
Section 3 describes how the implementation has solved the issues shown above, 
Section 4 evaluates the implementation with basic and practical applications, and 
Section 5 concludes.

