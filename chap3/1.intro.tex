\section{Introduction}\label{chap:intro}

\pagenumbering{arabic}
\setcounter{page}{1}

XcalableMP (XMP)~\cite{xmp} has complementary global-view and local-view 
programming models. The former is a directive-based language 
extension to the base languages Fortran and C, and the latter adopts 
the coarray features defined in Fortran 2008~\cite{coarray} and 
a part of the coarray features defined in Fortran 2018~\cite{coarray18}. 
%
The purpose of the coarray features as the local-view part of XMP is 
1) writing applications that are not suitable for global-view programming
and 2) writing important parts of programs that are critical to performance
with an easier programming model than MPI message passing.
Therefore, the coarray features in XMP must be naturally merged into the 
global-view XMP language and must exhibit high performance, comparable to that of MPI.

The Omni XMP compiler is an open-source implementation developed at RIKEN 
and the University of Tsukuba~\cite{omni}. 
The kernel of the Omni XMP compiler is a source-to-source compiler that 
converts an XMP program into a Fortran program by calling a runtime library.
%
The coarray translator has been implemented on the Omni XMP compiler.
Since the images are mapped one-to-one to XMP nodes, 
each image was implemented as a process, and the definition and reference 
to coarrays were implemented as inter-node one-sided communications.

This chapter describes the techniques used in the coarray compiler and 
the runtime library, and a comparison to MPI message passing.
The remainder of this chapter is organized as follows. 
Section 2 introduces the requirements of the coarray features.
Section 3 describes the implementation used to solve the requirements, and
Section 4 evaluates the performance and productivity of coarray programs.
Related research is described in Section 5, and Section 6 concludes this chapter.

