\section{Introduction}\label{chap:intro}

\pagenumbering{arabic}
\setcounter{page}{1}

XcalableMP (XMP)~\cite{xmp} has complementary programming models of
global-view and local-view. The former is a directive-base language 
extension to the base language Fortran and C, and the latter adopts 
the coarray features defined in Fortran 2008~\cite{coarray} and 
a part of the ones in Fortran 2018~\cite{coarray18}. 
%
The purpose of the coarray features as the local-view part of XMP is 
1) writing the application programs that is difficult for the global-view programming
and 2) writing such important parts of the program that is critical for the performance
with easier programming model than the MPI message passing.
Therefore, the coarray features in XMP must be naturally merged into the 
global-view XMP language and must perform with high performance comparable to MPI.

The Omni XMP compiler is an open-source implementation developed at RIKEN 
and the University of Tsukuba~\cite{omni}. 
Its kernel is a source-to-source compiler that converts an XMP program 
into a Fortran program by calling a runtime library.
%
The coarray translator has been implemented into the Omni XMP compiler.
Since the images is mapped one-to-one to XMP nodes, 
each image was implemented as a process, and 
the definition and reference to coarrays were implemented as the 
inter-node one-sided communications.

This chapter describes the techniques in the coarray compiler and 
the runtime library with some evaluation compared with the MPI message passing.
In the rest of this chapter, 
Section 2 introduces the requirements from the coarray features,
Section 3 describes the implementation to solve the requirements, and
Section 4 evaluates the performance and the productivity of coarray programs.
After related work is shown in Section 5, Section 6 concludes this chapter.


