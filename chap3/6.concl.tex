\section{Conclusion}\label{sec:concl}

This chapter described the coarray features in the context of XMP and 
the characteristic implementation of the coarray translator.

For memory allocation and registration, the RS, RA, and CA methods were
implemented corresponding to the communication library GASNet, FJ-RDMA,
and MPI-3.

For the coarray PUT and GET communications, DMA and four buffering methods
were described. The effect of the non-blocking PUT communication was
analyzed, and the knowledge is used to make the coarray version of the Himeno benchmark
from the original MPI version.
The measurement results on 1,024 nodes of the K computer, the coarray version 
is 27\% and 42\% faster than the original MPI version for Himeno sizes
L and XL, respectively.
The effect of the optimization of GET communication was also obvious on 
the ping-pong benchmark on HA-PACS/TCA and Fujitsu PRIMEHPC FX100.

As an evaluation of productivity, the coarray program uses fewer than half as
many characters as the MPI message passing program to write the same algorithm
as the Himeno benchmark.

