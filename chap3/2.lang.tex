\section{Coarray Features and the Key of the Implementation}\label{sec:lang}

The XcalableMP/Fortran compiler supports a major part of coarray features defined in 
Fortran~2008 standard~\cite{F08}, and intrinsic procedures 
{\em CO\_SUM}, {\em CO\_MAX}, {\em CO_MIN} and {\em CO_BROADCAST} defined in 
Fortran~2018 standard~\cite{F18} were supported.

This section introduces the coarray features and the issues for implementation of 
the coarray features.

\subsection{Images and Coarrays}
An {\em image} is defined as a instance of a program. 
Each image executes the same program and has its own data individually 
(SPMD: Single Program/Multiple Data).
Each image has a different image index, which is one of $1, \cdots, n$, 
where $n$ is the number of images. 
While Fortran standard does not specify actually where each image is executed, 
XcalableMP defines that the images are mapped in tern to the executing nodes. 
Therefore, image $k$ is always executed simply on executing node $k$, 
where $1 \leq k \leq n$ and 
$n$ is the number of images and also the number of the executing nodes. 

Note that the executing nodes can be a subset of the entire (initial) node set. 
For example, two distinct node sets can execute two coarray subprograms concurrently.
When the execution encounters a {\em TASK} directive specified with a subset of nodes,
the set of executing images changes into its subset corresponding to the nodes.
The number of images and my image number, which are given by inquire functions
{\em num\_images} and {\em this\_image}, also match with the executing images.
The {\em SYNC\_IMAGES} statement synchronizes in each set of executing nodes.
Unless the {\em TASK} directives are used, coarray features are 
compatible to the ones of the Fortran standard. 

#多重タスク実行がある場合、それぞれのタスクでcoarrayプログラムが原則として独立に実行される。
#あるタスクでallocateしたcoarrayは、そのタスクの有効範囲を超えない範囲で、
#タスクを実行するnode間で共有されるのが原則である。同期の範囲も原則としてその中である。

これを入れるかどうか:
\begin{itemize}
\item The XcalableMP compiler manages these values in stack.
\item タスク実行のときのスタックの使い方、構文が入れ子なので同じレベルに戻ることができること。
\end{itemize}


\subsubsection{Non-allocatable and allocatable coarrays}

Basically, variables are private to each image and cannot be accessed from 
the other images without using intrinsic atomic procedures or global-view features.
{\em Coarrays} are only the variables that are allowed to be
accessed from the other images in the usual Fortran contexts. 

Coarrays can be declared as scalar or array variables with the SAVE attribute
(the static attribute of C) as follows:

\begin{verbatim}
      real(8), save :: a(100,100)[*]
      type(user_defined_type), save :: s[2,2,*]
\end{verbatim}

The square bracket notation in the declaration distingishes coarray variables from 
the others (non-coarrays). It declares the virtual shape of the images and the last 
dimension must be defferred (as `*').
This type of coarrays above are called as {\em non-allocatable coarrays}. 

Alternatively, coarrays can be declared as {\em allocatable coarrays} with the 
following form:

\begin{verbatim}
      real(8), allocatable :: b(:,:)[:]
      type(user_defined_type), allocatable :: t[:,:,:]
\end{verbatim}

割付けcoarrayはALLOCATE文を使って割付けて,DEALLOCATE文を使って解放することができる.
Fortranでは,手続に局所的でsave属性を持たない割付け変数は,手続からの復帰時に割付け状態であると,
自動的に解放される.CAFではcoarray変数についてもこの機能(automatinc deallocation)が要求されている.


共通の制約として,これらは全イメージで集団的に実行されなければならず,全イメージで同じ形状を
割付けなければならない.これらの制約により,処理系はcoarrayのメモリ配置をイメージ間で
synmetric に保つことができるので,リモートcoarrayのアドレス計算をローカル側の情報だけで
行うことができる。
例えば,他イメージの変数の多次元の添字式から相対アドレスを計算するには,
自イメージの同名の変数の情報を参照すればよいので,他イメージのインデックス情報を受け取る必要も,
事前に収集してテーブルに保存しておく必要もない.
これによりMPIのlendezbous通信のような通信前のアドレス交換のための通信を不要にできる。

communication libraryによっては、Fortranランタイムシステムが割付けた領域を効率的な
片側通信の対象にすることができない。その場合には、communication libaryが提供する方法で
割付けた領域をFortranランタイムシステムに渡してFortranの変数であると認識させる実装が
求められる。
また、データを片側通信の対象とするためには、commnication libraryにそれを {\em register}
しなければならない。commnucation libraryはその領域をRDMAの対象とするためのpin-downなどの
必要な前処理を行う必要があるからである。
registrationコストの削減のためには、すべてのnon-allocatable coarrayは1カ所にまとめて
利用者プログラムの実行開始前に1回だけ行うようにするのが望ましい。


\subsection{Communication}

Coarray features include three types of communications between images, i.e.,
\begin{itemize}
\item reference and definition to remote coarrays (GET and PUT communication),
\item collective communication (intrinsic subroutines {\em CO\_SUM}, {\em CO\_MAX}, 
{\em CO_MIN} and {\em CO_BROADCAST}), and
\item atomic operations ({\em ATOMIC\_DEFINE} and {\em ATOMIC\_REF}).
\end{itemize}

Omni compiler implements the latter two communications using the corresponding 
functions in MPI. The rest of this section describes the former communication.


\subsubsection{GET communication}

GET communication is caused by referencing the {\em coindexed object}, 
which is represented by a coarray variable with cosubscripts enclosed by square brackets, 
e.g., $s[1,2]$ and $a(i,j)[k]$, where $s$ and $a$ are scalar and two-dimentional array coarrays,
respectively.
Coindexed objects can appear in most expressions.

coindexed objectの参照は、$a(:,:)[k]$や$a(2:N-1,1:N)[k]$のように全体配列や部分配列になることも
でき、一般に連続であるとは限らない。全体配列であっても形状引継ぎ配列の場合には不連続かもしれない。
一般にこの連続部分の長さは実行時でなければ分からない。
通信の高速化のためには,十分に長い範囲の連続性を実行時に抽出する必要がある.
一般にノード間通信で立ち上がりレイテンシ時間がほぼ無視できるようになるデータ量は
数千バイトであることから,配列や部分配列の1次元め(Fortranでは1番左の添字)が
連続であっても数千要素に満たない場合には,次元を跨いだ連続性まで抽出して,
十分に長い連続データとして通信する技術が必要である.
【fx100-latencyのグラフから数千バイトを言ってもよい。】


【右辺式がcoindexed objectのとき最適化できる。】


\subsubsection{PUT communication}

PUT communication is caused by an assignment statement with a {\em coindexed variable} 
as the left-hand side expression, e.g.,
\begin{verbatim}
      a(i,j)[k] = b(i,j) + c(i,j)
\end{verbatim}

Get通信と同様,a(i,j)[k]は代わりに全体配列にも部分配列にもなれるので,
coindexed variableについても次元を跨いだ連続性の抽出が必要である.
それに加えて、ローカル側、すなわち、右辺式データの連続性も意識に入れなければならない。
高速な通信を実現するには,左辺と右辺で共通に連続な区間を検出してその単位で通信を反復するか,
右辺データは連続区間にpackして左辺の連続区間を単位として通信を反復するなどの戦略がある。



\subsection{Argument Passing}
@@@
coarray変数の引数渡しをトランスレータ方式で実現するには大きな選択肢があった.
coarray変数を1つの記述子に変換し,記述子を通して実体のアドレスとその他の情報を取り出せるようにすれば,
引数渡しは記述子の受渡しだけとなる.しかしその場合には,
データ実体がFortranシステムから見えなくなるので,dopeベクトルも我々が生成・管理しなければならない.
そして,サブプログラム内でのcoarray変数の通常の参照・定義に対して,
dopeベクトルを使ったインデックスの計算式をソース-to-ソースで組み上げるか,
実行時ライブラリ呼出しに変換しなければならない.
これは大きな性能低下が予想される.
我々は,coarray変数を同じ形状をもつ非coarray変数に変換することで,
Fortranシステムのdopeベクトルの仕組みをそのまま使う方法を採った.
その結果新しく生じた課題については3.3節で議論する.


\subsection{Loose synchronization}
各イメージでは従来通りのデータフローを守りながら、image間では明示的な同期を書かない限り順序が保証されない仕様である。
この仕様が連続通信の自由度と制限を決める。図を使って説明。参照はblockingになる。定義はデータ依存関係の許す限り、次の同期までのnon-blockingが許される。ここに最適化のチャンスがある。


\subsection{Coarray C Language Specifications}

C言語にも配列式、配列代入を導入することでFortranと同様なcoarray featuresを仕様定義した。

coarrayはデータ実体に限る。ポインタはcoarrayになれない。形式的には、
1)基本型、2)ポインタを含まない構造型、または、3) 1or2or3の配列とする。
Cのcoarrayにおいてもstaticとallocatableがある。ファイル直下で宣言するか
static指示したcoarrayはstatic coarrayである。ライブラリ関数xxxをつかって
割付けたデータ領域はallocatable coarrayであり、ライブラリ関数は
allocatable coarrayへのポインタを返す。
Cのcoarrayは、通常のCの変数と同じように、引数渡しやcast演算によって自由に
その型と形状の解釈を変えることができる。

これらの仕様と制限は、Cプログラマ
にとっての使いやすさを考えて、Cらしいプログラミングスタイルを認めた。
Coarray C++とは違うアプローチである。

cf.\ air:/Users/iwashita/Desktop/coarray/Project\_Coarray/の下にいくつか
