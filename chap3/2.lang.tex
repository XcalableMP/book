\section{Requirements from Language Specifications}\label{sec:spec}

XMP Fortran language specification supports a major part of coarray features defined in 
Fortran~2008 standard~\cite{F08}, and intrinsic procedures 
{\tt CO\_SUM}, {\tt CO\_MAX}, {\tt CO\_MIN} and {\tt CO\_BROADCAST} defined in 
Fortran~2018 standard~\cite{F18} were supported.
And also XMP C language specification extended to support coarray features.

This section introduces the coarray features and what is required
to the compiler in order to implement the coarray features.


%-----------------------------------------------------------------------------
\subsection{Images Mapped to XMP Nodes}\label{sec:spec-image}
%-----------------------------------------------------------------------------

In the Fortran standard, an {\bf image} is defined as a instance of a program. 
Each image executes the same program and has its own data individually.
Each image has a different image index $k$.
While the Fortran standard itself does not specify where each image is executed, 
XMP specifies that images are mapped to executing nodes on a one-to-one basis.
Therefore, image $k$ is always executed on executing node $k$, where $1 \leq k \leq n$ and 
$n$ is the number of images and also the number of the executing nodes. 
Since each MPI rank number of {\tt MPI\_COMM\_WORLD} (0-origin) is 
always mapped to an XMP node number in order, image $k$ is corresponding to 
rank $(k - 1)$.

Note that the executing nodes can be a subset of the entire (initial) node set. 
For example, two distinct node sets can execute two coarray subprograms concurrently.
The first executing images at the start of the program is the entire images.
Coarray features are compatible to the ones of the Fortran standard unless 
the {\tt TASK} and {\tt END TASK} directives are used.
If the execution encounters a {\tt TASK} directive specified with a subset of nodes, 
the corresponding subset of the images will be the executing images for the task region. 
The current number of images and my image number, which are given by inquire functions
{\tt num\_images} and {\tt this\_image}, also match with the executing images, and
the {\tt SYNC\_IMAGES} statement synchronizes among the executing images.
When the execution encounters the {\tt END TASK} directive corresponding to the
{\tt TASK} directive, the set of executing image is reinstated.

%   Coarray features can be used inside the TASK directive blocks. As default,
%   each coarray image is mapped one-to-one to a node of the current executing 
%   task. I.e., num_images() returns the number of nodes of the current executing 
%   task and this_image() returns each image index in the task.
%      There are two directives to change the default rule above. A COARRAY 
%   directive corresponding to a coarray declaration changes the image index set 
%   of the specified coarray with the one of the specified nodes. An IMAGE 
%   directive corresponding to one of a SYNC ALL statement, a SYNC IMAGES 
%   statement, a call statement calling CO_SUM, CO_MAX, CO_MIN or CO_BROADCAST 
%   changes the current image index set with the one of the specified nodes.
%   See the language spacifications [3].

\requirement
The runtime library should manage the executing image set and the current image index 
in stack in order to reinstate them at the exit point of the task.


%-----------------------------------------------------------------------------
\subsection{Allocation of Coarrays}\label{sec:spec-coarray}
%-----------------------------------------------------------------------------

A {\bf coarray} or a coarray variable is a variable that can be referred from the other images. 
A coarray with the {\tt ALLOCATABLE} attribute is called an {\bf allocatable coarray}, 
otherwise called a non-allocatable coarray. A non-allocatable coarray may not be a pointer 
and must have an explicit shape and the {\tt SAVE} attribute. In order to help 
intuitive understanding, we call a non-allocatable coarray as a {\bf static coarray}. 
The lifetime of a static coarray is throughout execution of the program on all images even if
the coarray is declared in a procedure called with a subset of images.

On the other hand, an allocatable coarray is allocated with the {\tt ALLOCATE} statement and 
freed either explicitly with the {\tt DEALLOCATE} statement or implicitly at the end of the 
scope in which the {\tt ALLOCATE} statement is executed ({\bf automatic deallocation}).

Static coarrays can be declared as scalar or array variables as follows:
\begin{verbatim}
      real(8), save :: a(100,100)[*]
      type(user_defined_type), save :: s[2,2,*]
\end{verbatim}

The square bracket notation in the declaration distinguishes coarray variables from 
the others (non-coarrays). It declares the virtual shape of the images and the last 
dimension must be deferred (as `{\tt *}').

Allocatable coarrays can be declared as follows:
\begin{verbatim}
      real(8), allocatable :: b(:,:)[:]
      type(user_defined_type), allocatable :: t[:,:,:]
\end{verbatim}


A notable constraint is that at any synchronization point in program execution, 
coarrays must have the same dimensions (sizes of all axes) between all images
({\bf symmetric memory allocation}). 
Therefore, an static coarray must have the same shape between all images during 
the program execution, and an allocatable coarray must be allocated and deallocated 
collectively at the same time with the same dimensions between the executing images.
Thanks to the syn-metric memory allocation rule, all executing images can have
the same symmetrical memory layout, which makes it possible to calculate the address 
of the remote coarray with no prior inter-image communication.

\requirement
Static coarrays must be allocated and made accessible remotely
before the execution of the user program, and 
made inaccessible remotely and be freed after the execution of the user program.
In contrast, 
allocatable coarrays must be allocated and made accessible remotely
when the {\tt ALLOCATE} statement is encountered, and 
made inaccessible remotely and be freed when the {\tt DEALLOCATE} statement or 
the exit point of the scope that the corresponding {\tt ALLOCATE} statement is encountered 
is encountered.


%-----------------------------------------------------------------------------
\subsection{Communications}\label{sec:spec-comm}
%-----------------------------------------------------------------------------

Coarray features include three types of communications between images, i.e.,
reference and definition to remote coarrays,
collective communications (intrinsic subroutines {\tt CO\_SUM}, {\tt CO\_MAX}, 
{\tt CO\_MIN} and {\tt CO\_BROADCAST}), and
atomic operations ({\tt ATOMIC\_DEFINE} and {\tt ATOMIC\_REF}).

MPI library has the similar functions to 

Collective communications and atomic operations are in the form of 
intrinsic procedures and similar ones are found in MPI library.
Communication for reference and definition to remote coarrays are 
characteristic for coarray features.

%===========================================================
%\subsubsection{PUT communication}\label{sec:spec-put}
%===========================================================

PUT communication is caused by an assignment statement with a {\bf coindexed variable} 
as the left-hand side expression, e.g.,
\begin{verbatim}
      a(i,j)[k] = alpha * b(i,j) + c(i,j)
\end{verbatim}
This statement is to cause the PUT communication to the array element $a(i,j)$ on image $k$
with the value of the left-hand side.
%
Using Fortran array assignment statement, array-to-array PUT communication 
can be written easily. E.g., the following statement causes $M N$-element 
PUT communication.
\begin{verbatim}
      a(1:M,1:N)[k] = alpha * b(1:M,1:N) + c(1:M,1:N)
\end{verbatim}


%===========================================================
%\subsubsection{GET communication}\label{sec:spec-get}
%===========================================================

GET communication is caused by referencing the {\bf coindexed object}, 
which is represented by a coarray variable with cosubscripts enclosed by square brackets, 
e.g., $s[1,2]$ and $a(i,j)[k]$, where $s$ and $a$ are scalar and two-dimensional array 
coarrays, respectively.
%
A coindexed object can appear almost in any expressions including array expressions.

\requirement
To implement definition/reference to coindexed variable/object,
PUT/GET one-sided communication is suitable to be used.
%
To avoid costly processing such as remote procedure call, 
RDMA (Remote Direct Memory Access)-based implementation is desirable.
%
On PUT/GET communication for large data, redundant multiple memory copies 
should carefully be avoided for all software layers, 
the communication library, the runtime, the Fortran library, and the object.

% A reference of coindexed object is basically converted to a runtime library call
% to get the result of GET communication. 
% Note that the result can be a large array.
% For example, in in array assignment statement:
% \begin{verbatim}
%       c(1:M,1:N) = a(1:M,1:N)[k] + b(1:M,1:N)
% \end{verbatim}
% coindexed object $a(1:M,1:N)[k]$ should be converted to a function that
% returns an array value shaped $[M, N]$.


%-----------------------------------------------------------------------------
\subsection{Inter-image Synchronization}\label{spec-sync}
%-----------------------------------------------------------------------------

The access order of coarrays between images must be explicitly controled by the 
programmer using the {\bf image control statement}, 
such as {\tt SYNC ALL} and {\tt SYNC IMAGES} statements. 
It allows the compiler system to make PUT/GET communication asynchronous.
The portion between such two image control statements is called as {\bf segmemt}.

On the other hand, inside an image, the compiler system must maintain data 
dependency as before. It suppress the {\bf non-blocking communication},
which postpones waiting for completion.
In order to keep data dependency among the definitions and references to the same 
coarray in the same segment, the non-blocking communication should be restricted.
The example bellow in which the same remote coarray is accessed some times 
inside the same segment.
\begin{quote}
\begin{verbatim}
 1      if (this_image()==2) then
 2          a[2]=
 3          =a[2]
 4          a[2]=
 5      endif
\end{verbatim}
\end{quote}
In this case, the update of {\tt a} in line 2 must be completed before 
the reference in line 3, which must be completed before the update in line 4.


\requirement
%To reduce the latency overhead, non-blocking one-sided communication is effective.
%The compiler should generate non-blocking GET communication as long as possible.
In usual case, the completion point of the non-blocking is the end of the segment.
However, the possibility to meet the condition shown above should 
be took account. If the PUT communication is performed always in blocking, 
only the flow dependency should be care; the previous PUT communication (line~2) 
should be completed before the following GET communication (line~3).


%-----------------------------------------------------------------------------
\subsection{Subarrays and Data Contiguity}\label{sec:spec-contig}
%-----------------------------------------------------------------------------

In Fortran, if an array, except dummy argument, is declared as {\tt real a(1:M,1:N)}, 
the reference to the whole array (i.e., {\tt a} or {\tt a(:,:)} or {a(1:M,1:N)})
is defined {\bf contiguous}. 
We defined a term {\bf contiguous length} as the length how long the data is partially
contiguous. For example, the contiguous lengths of {\tt a(2,3)} and {\tt a(2:5,3)} are
1 and 4 respectively.  {\tt a(1:M,1:3)} is two-dimensionally contiguous and has 
contiguous length $2 \times {\tt M}$.
{\tt a(1:M-1,1:3)} is one-dimensionally contiguous and has 
contiguous length $({\tt M} - 1)$.


\requirement
通信の高速化のためには,十分に長い範囲の連続性を実行時に抽出する必要がある.
一般にノード間通信で立ち上がりレイテンシ時間がほぼ無視できるようになるデータ量は
数千バイトであることから,配列や部分配列の1次元め(Fortranでは1番左の添字)が
連続であっても数千要素に満たない場合には,次元を跨いだ連続性まで抽出して,
十分に長い連続データとして通信する技術が必要である.
【fx100-latencyのグラフから数千バイトを言ってもよい。】

GET通信と同様,a(i,j)[k]は代わりに全体配列にも部分配列にもなれるので,
coindexed variableについても次元を跨いだ連続性の抽出が必要である.
それに加えて、ローカル側、すなわち、右辺式データの連続性も意識に入れなければならない。
高速な通信を実現するには,左辺と右辺で共通に連続な区間を検出してその単位で通信を反復するか,
右辺データは連続区間にpackして左辺の連続区間を単位として通信を反復するなどの戦略がある。



%-----------------------------------------------------------------------------
\subsection{Coarray C Language Specifications}\label{sec:spec-c}
%-----------------------------------------------------------------------------

C言語にも配列式、配列代入を導入することでFortranと同様なcoarray featuresを仕様定義した。

coarrayはデータ実体に限る。ポインタはcoarrayになれない。形式的には、
1)基本型、2)ポインタを含まない構造型、または、3) 1or2or3の配列とする。
Cのcoarrayにおいてもstaticとallocatableがある。ファイル直下で宣言するか
static指示したcoarrayはstatic coarrayである。ライブラリ関数xxxをつかって
割付けたデータ領域はallocatable coarrayであり、ライブラリ関数は
allocatable coarrayへのポインタを返す。
Cのcoarrayは、通常のCの変数と同じように、引数渡しやcast演算によって自由に
その型と形状の解釈を変えることができる。

これらの仕様と制限は、Cプログラマ
にとっての使いやすさを考えて、Cらしいプログラミングスタイルを認めた。
Coarray C++とは違うアプローチである。

cf.\ air:/Users/iwashita/Desktop/coarray/Project\_Coarray/の下にいくつか
