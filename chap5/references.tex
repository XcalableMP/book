%%%%%%%%%%%%%%%%%%%%%%%% referenc.tex %%%%%%%%%%%%%%%%%%%%%%%%%%%%%%
% sample references
% %
% Use this file as a template for your own input.
%
%%%%%%%%%%%%%%%%%%%%%%%% Springer-Verlag %%%%%%%%%%%%%%%%%%%%%%%%%%
%
% BibTeX users please use
% \bibliographystyle{}
% \bibliography{}
%
%\biblstarthook{References may be \textit{cited} in the text either by number (preferred) or by author/year.\footnote{Make sure that all references from the list are cited in the text. Those not cited should be moved to a separate \textit{Further Reading} section or chapter.} If the citatiion in the text is numbered, the reference list should be arranged in ascending order. If the citation in the text is author/year, the reference list should be \textit{sorted} alphabetically and if there are several works by the same author, the following order should be used:
%\begin{enumerate}
%\item all works by the author alone, ordered chronologically by year of publication
%\item all works by the author with a coauthor, ordered alphabetically by coauthor
%\item all works by the author with several coauthors, ordered chronologically by year of publication.
%\end{enumerate}
%The \textit{styling} of references\footnote{Always use the standard abbreviation of a journal's name according to the ISSN \textit{List of Title Word Abbreviations}, see \url{http://www.issn.org/en/node/344}} depends on the subject of your book:
%\begin{itemize}
%\item The \textit{two} recommended styles for references in books on \textit{mathematical, physical, statistical and computer sciences} are depicted in ~\cite{science-contrib, science-online, science-mono, science-journal, science-DOI} and ~\cite{phys-online, phys-mono, phys-journal, phys-DOI, phys-contrib}.
%\item Examples of the most commonly used reference style in books on \textit{Psychology, Social Sciences} are~\cite{psysoc-mono, psysoc-online,psysoc-journal, psysoc-contrib, psysoc-DOI}.
%\item Examples for references in books on \textit{Humanities, Linguistics, Philosophy} are~\cite{humlinphil-journal, humlinphil-contrib, humlinphil-mono, humlinphil-online, humlinphil-DOI}.
%\item Examples of the basic Springer Nature style used in publications on a wide range of subjects such as \textit{Computer Science, Economics, Engineering, Geosciences, Life Sciences, Medicine, Biomedicine} are ~\cite{basic-contrib, basic-online, basic-journal, basic-DOI, basic-mono}. 
%\end{itemize}
%}

\begin{thebibliography}{99.}%
% and use \bibitem to create references.
%
% Use the following syntax and markup for your references if 
% the subject of your book is from the field 
% "Mathematics, Physics, Statistics, Computer Science"
%
  % Contribution
\bibitem{1303318} F. Cantonnet et al.
  ``Productivity analysis of the UPC language'', 18th International Parallel and Distributed Processing Symposium, pp. 254--260 (2004)
%
\bibitem{Katherine} Katherine Yelick et al.
  ``Productivity and Performance Using Partitioned Global Address Space Languages'', Proceedings of the international workshop on Parallel symbolic computation (2007)
%
\bibitem{CGPOP2011} Andrew I. Stone et al.
``Evaluating Coarray Fortran with the CGPOP Miniapp'', Proceedings of the Fifth Conference on Partitioned Global Address Space Programming Models (2011)
%
\bibitem{doi:10.1177/1094342017698214} Masahiro Nakao et al.
``Implementation and evaluation of the HPC challenge benchmark in the XcalableMP PGAS language'',
The International Journal of High Performance Computing Applications, pp. 1--14 (2017)
%
\bibitem{Jose:2010:UUM:2020373.2020378} Jose Jithin et al.
``Unifying UPC and MPI Runtimes: Experience with MVAPICH'',
Proceedings of the Fourth Conference on Partitioned Global Address Space Programming Model, pp. 5:1--5:10 (2010)
%
\bibitem{2013nakao} Masahiro Nakao et al.
``Productivity and Performance of the HPC Challenge Benchmarks with the XcalableMP PGAS Language'',
Proceedings of the Fourth Conference on Partitioned Global Address Space Programming Model, pp. 157--171 (2013)
%
\bibitem{2012nakao} Masahiro Nakao et al.
``Productivity and Performance of Global-View Programming with XcalableMP PGAS Language'',
Proceedings of the 2012 12th IEEE/ACM International Symposium on Cluster, Cloud and Grid Computing,
pp. 402--409 (2012)
%
\bibitem{xmp-spec} \url{https://xcalablemp.org/download/spec/xmp-spec-1.4.pdf}
%
\bibitem{nakao2014} Masahiro Nakao et al.
``XcalableACC: Extension of XcalableMP PGAS Language Using OpenACC for Accelerator Clusters'',
Proceedings of the First Workshop on Accelerator Programming Using Directives, pp. 27--36 (2014)
%
\bibitem{xacc-spec} \url{http://xcalablemp.org/XACC.html}
%
\bibitem{nakao2017} Masahiro Nakao et al.
``Implementing Lattice QCD Application with XcalableACC Language on Accelerated Cluster'',
IEEE International Conference on Cluster Computing (CLUSTER), pp. 429--438 (2017)
%
\bibitem{nakao2019} Masahiro Nakao et al.
  ``Evaluation of XcalableACC with Tightly Coupled Accelerators/InfiniBand Hybrid Communication on Accelerated Cluster''
  International Journal of High Performance Computing Applications (2019)
%
\bibitem{Numrich:1998:CFP:289918.289920} Numrich Robert W. et al.
``Co-array Fortran for parallel programming'',
SIGPLAN Fortran Forum, Vol. 17, No. 2, pp. 1--31 (1998)
%
\bibitem{pcj} Nowicki M. et al.
  ``PCJ - Java library for high performance computing in PGAS model'',
   International Conference on High Performance Computing \& Simulation, pp.202--209 (2014)
%
\bibitem{upc-1.3} \url{https://upc-lang.org/assets/Uploads/spec/upc-lang-spec-1.3.pdf}
%
\bibitem{upc_plus_plus} Yili Zheng et al.
  ``UPC++: A PGAS Extension for C++'',  IEEE 28th International Parallel and Distributed Processing Symposium, pp. 1105--1114 (2014)
%
\bibitem{Kumar:2014:HCP:2676870.2676879} Kumar Vivek et al.
``HabaneroUPC++: A Compiler-free PGAS Library'', 
  Proceedings of the 8th International Conference on Partitioned Global Address Space Programming Models,
  No.5, pp. 5:1--5:10 (2014)
%
\bibitem{Charles:2005:XOA:1103845.1094852} Charles Philippe et al.
``X10: An Object-oriented Approach to Non-uniform Cluster Computing'',
OOPSLA '05 Proceedings of the 20th annual ACM SIGPLAN conference on Object-oriented programming, systems, languages, and applications, Vol.40, No.10, pp. 519--538 (2005)
%
\bibitem{Chamberlain:2007:PPC:1286120.1286123} Chamberlain B.L. et al.
``Parallel Programmability and the Chapel Language'',
Int. J. High Perform. Comput. Appl. Vol.21 No.3 pp.291--312 (2007)
%
\bibitem{DBLP:journals/corr/FurlingerFK16} Karl F{\"{u}}rlinger et al.
``DASH: A C++ PGAS Library for Distributed Data Structures and Parallel Algorithms'',
  IEEE 18th International Conference on High Performance Computing and Communications; IEEE 14th International Conference on Smart City; IEEE 2nd International Conference on Data Science and Systems (HPCC/SmartCity/DSS), pp. 983--990 (2016)
%
\bibitem{pgas-ei} Masahiro Nakao et al.
``Linkage of XcalableMP and Python languages for high productivity on HPC cluster system'',
Workshop on PGAS programming models: Experiences and Implementations (2018)
\end{thebibliography}
