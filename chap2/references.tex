%%%%%%%%%%%%%%%%%%%%%%%% referenc.tex %%%%%%%%%%%%%%%%%%%%%%%%%%%%%%
% sample references
% %
% Use this file as a template for your own input.
%
%%%%%%%%%%%%%%%%%%%%%%%% Springer-Verlag %%%%%%%%%%%%%%%%%%%%%%%%%%
%
% BibTeX users please use
% \bibliographystyle{}
% \bibliography{}
%
%\biblstarthook{References may be \textit{cited} in the text either by number (preferred) or by author/year.\footnote{Make sure that all references from the list are cited in the text. Those not cited should be moved to a separate \textit{Further Reading} section or chapter.} If the citatiion in the text is numbered, the reference list should be arranged in ascending order. If the citation in the text is author/year, the reference list should be \textit{sorted} alphabetically and if there are several works by the same author, the following order should be used:
%\begin{enumerate}
%\item all works by the author alone, ordered chronologically by year of publication
%\item all works by the author with a coauthor, ordered alphabetically by coauthor
%\item all works by the author with several coauthors, ordered chronologically by year of publication.
%\end{enumerate}
%The \textit{styling} of references\footnote{Always use the standard abbreviation of a journal's name according to the ISSN \textit{List of Title Word Abbreviations}, see \url{http://www.issn.org/en/node/344}} depends on the subject of your book:
%\begin{itemize}
%\item The \textit{two} recommended styles for references in books on \textit{mathematical, physical, statistical and computer sciences} are depicted in ~\cite{science-contrib, science-online, science-mono, science-journal, science-DOI} and ~\cite{phys-online, phys-mono, phys-journal, phys-DOI, phys-contrib}.
%\item Examples of the most commonly used reference style in books on \textit{Psychology, Social Sciences} are~\cite{psysoc-mono, psysoc-online,psysoc-journal, psysoc-contrib, psysoc-DOI}.
%\item Examples for references in books on \textit{Humanities, Linguistics, Philosophy} are~\cite{humlinphil-journal, humlinphil-contrib, humlinphil-mono, humlinphil-online, humlinphil-DOI}.
%\item Examples of the basic Springer Nature style used in publications on a wide range of subjects such as \textit{Computer Science, Economics, Engineering, Geosciences, Life Sciences, Medicine, Biomedicine} are ~\cite{basic-contrib, basic-online, basic-journal, basic-DOI, basic-mono}. 
%\end{itemize}
%}

\begin{thebibliography}{99.}%
% and use \bibitem to create references.
%
% Use the following syntax and markup for your references if 
% the subject of your book is from the field 
% "Mathematics, Physics, Statistics, Computer Science"
%
% Contribution 
\bibitem{aaa} Masahiro Nakao et al.
``Implementation and evaluation of the HPC Challenge benchmark in the XcalableMP PGAS language'', International Journal of High Performance Computing Applications, 33(1), pp.110--123 (2017)
%
\bibitem{omni} \url{https://omni-compiler.org}
%
\bibitem{pro-env} Programming Environment Research Team.
\url{https://pro-env.riken.jp}
%
\bibitem{hpcs} High Performance Computing System laboratory, University of Tsukuba, Japan.
\url{https://www.hpcs.cs.tsukuba.ac.jp}
%
\bibitem{xcodeml} Mitsuhisa Sato et al.
  ``Omni Compiler and XcodeML: An Infrastructure for Source-to-Source Transformation'',
  Platform for Advanced Scientific Computing Conference (PASC16), Lausanne, Switzerland, Jun. (2016)
 %
\bibitem{ixpug} Masahiro Nakao et al.
  ``Performance Evaluation for Omni XcalableMP Compiler on Many-core Cluster System based on Knights Landing'',
  IXPUG Workshop Asia 2018, Tokyo, Japan, Jan. pp. 52--58 (2018)
%
\bibitem{github} \url{https://github.com/omni-compiler/omni-compiler}
%
%\bibitem{guide} \url{https://omni-compiler.org/manual/en/}
%
\bibitem{gasnet} \url{https://gasnet.lbl.gov}
%
\bibitem{scalasca} \url{https://www.scalasca.org}
%
\bibitem{hpcc} \url{https://icl.utk.edu/hpcc/}
%
\bibitem{hpca} Masahiro Nakao et al.
``Implementation and evaluation of the HPC Challenge benchmark in the XcalableMP PGAS language'',
  International Journal of High Performance Computing Applications, 33(1), 110-123. Mar. (2017)
%
\bibitem{hpcc-a} \url{https://www.hpcchallenge.org}%
%
\bibitem{blas} BLAS: Basic Linear Algebra Subprograms \url{http://www.netlib.org/blas/} (2016)
%
\bibitem{fft1} David H. Bailey.
  ``FFTs in external or hierarchical memory. Journal of Supercomputing'',
  Vol.4, pp.23--35 (1990)
%
\bibitem{fft2} Van Loan C.
  ``Computational Frameworks for the Fast Fourier Transform'',
  Society for Industrial and Applied Mathematics (1992)
%
\bibitem{ffte} Daisuke Takahashi.
  A Fast Fourier Transform Package. \url{http://www.ffte.jp} (2014)
%
\bibitem{randomaccess} Ponnusamy R. et al.
  ``Communication overhead on the CM5: an experimental performance evaluation'',
  Fourth Symposium on the Frontiers of Massively Parallel Computation, pp.108--115 (1992)
%
\bibitem{modified} HPL Algorithm Panel Broadcast. \url{http://www.netlib.org/benchmark/hpl/algorithm.html} (2016)
%
%\bibitem{XACC1} Masahiro Nakao et al.
%  ``XcalableACC: Extension of XcalableMP PGAS Language Using OpenACC for Accelerator Clusters'',
%  Proceedings of the First Workshop on Accelerator Programming Using Directives, pp.27--36 (2014)
%
%\bibitem{XACC2} Masahiro Nakao et al.
%  ``Evaluation of XcalableACC with Tightly Coupled Accelerators/InfiniBand Hybrid Communication on Accelerated Cluster'',
%  International Journal of High Performance Computing Applications, Jan. (2019)
%
%\bibitem{caf-hpcc} Guohua Jin et al.
%  ``Implementation and Performance Evaluation of the HPC Challenge Benchmarks in Coarray Fortran 2.0'',
%  Parallel Distributed Processing Symposium (IPDPS), 2011 IEEE International, pp.1089--1100 (2011)
%
%\bibitem{2012Chapel-HPCC} Brad Chamberlain et al.
%  ``Chapel HPC Challenge Entry'',
%  \url{http://www.hpcchallenge.org/presentations/sc2012/ChapelHPCC2012.pdf} (2011)
%
%\bibitem{2012X10-HPCC} Olivier Tardieu et al.
%  ``X10 for Productivity and Performance at Scale'',
%  \url{http://www.hpcchallenge.org/presentations/sc2012/x10-hpcc.pdf} (2012)
%
%\bibitem{Tardieu:2014:XAP:2555243.2555245} Olivier Tardieu et al.
%  ``X10 and APGAS at Petascale'',
%  Proceedings of the 19th ACM SIGPLAN Symposium on Principles and Practice of Parallel Programming pp.53--66 (2014)
%
%\bibitem{Kennedy:2007:RFH:1238844.1238851} Kennedy Ken et al.
%  ``The rise and fall of High Performance Fortran: an historical object lesson'',
%  Proceedings of the third ACM SIGPLAN conference on History of programming languages, pp.7-1--7-22 (2007)
%
%\bibitem{tsugane2016} Keisuke Tsugane et al.
%  ``Proposal for Dynamic Task Parallelism in PGAS Language XcalableMP'',
%  The 6th AICS International Symposium, p.57 (2016)
\end{thebibliography}
