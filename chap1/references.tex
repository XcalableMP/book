%%%%%%%%%%%%%%%%%%%%%%%% referenc.tex %%%%%%%%%%%%%%%%%%%%%%%%%%%%%%
% sample references
% %
% Use this file as a template for your own input.
%
%%%%%%%%%%%%%%%%%%%%%%%% Springer-Verlag %%%%%%%%%%%%%%%%%%%%%%%%%%
%
% BibTeX users please use
% \bibliographystyle{}
% \bibliography{}
%

\begin{thebibliography}{99.}%
% and use \bibitem to create references.
%
% Use the following syntax and markup for your references if 
% the subject of your book is from the field 
% "Mathematics, Physics, Statistics, Computer Science"
%
% Contribution 
\bibitem{caf} Robert W. Numrich and John Reid, ``Co-Array Fortran for
		parallel programming'', ACM SIGPLAN Fortran Forum, Vol.~17,
		No.~2 (1998).
\bibitem{upc} UPC Consortium, ``UPC Specifications, v1.2'', Lawrence
		Berkeley National Lab (LBNL-59208) (2005).
\bibitem{chapel} David Callahan, Bradford L.~Chamberlain and Hans
		P.~Zima,  ``The Cascade High Productivity Language'', Proc. 9th
		Int'l. Workshop on High-Level Parallel Programming Models and
		Supportive Environments (HIPS 2004), pp.~52--60 (2004).

\bibitem{ompt}
		{OpenMP Architecture Review Board}, ``OpenMP Application
		Programming Interface Version 5.0'' (2018).


\bibitem{MUST-project}
{The MUST Project},
\newblock https://www.itc.rwth-aachen.de/must.

\bibitem{Extrae-project}
{The Extrae Project},
\newblock https://tools.bsc.es/extrae.
%
% \bibitem{pro-env} Programming Environment Research Team.
% \url{https://pro-env.riken.jp}
% %
% \bibitem{hpcs} High Performance Computing System laboratory, University of Tsukuba, Japan.
% \url{https://www.hpcs.cs.tsukuba.ac.jp}
% %
% \bibitem{xcodeml} Mitsuhisa Sato et al.
%   ``Omni Compiler and XcodeML: An Infrastructure for Source-to-Source Transformation'',
%   Platform for Advanced Scientific Computing Conference (PASC16), Lausanne, Switzerland, Jun. (2016)
%  %
% \bibitem{ixpug} Masahiro Nakao et al.
%   ``Performance Evaluation for Omni XcalableMP Compiler on Many-core Cluster System based on Knights Landing'',
%   IXPUG Workshop Asia 2018, Tokyo, Japan, Jan. pp. 52--58 (2018)
% %
% \bibitem{github} \url{https://github.com/omni-compiler/omni-compiler}
% %
% %\bibitem{guide} \url{https://omni-compiler.org/manual/en/}
% %
% \bibitem{gasnet} \url{https://gasnet.lbl.gov}
% %
% \bibitem{scalasca} \url{https://www.scalasca.org}
% %
% \bibitem{hpcc} \url{https://icl.utk.edu/hpcc/}
% %
% \bibitem{hpca} Masahiro Nakao et al.
% ``Implementation and evaluation of the HPC Challenge benchmark in the XcalableMP PGAS language'',
%   International Journal of High Performance Computing Applications, 33(1), 110-123. Mar. (2017)
% %
% \bibitem{hpcc-a} \url{https://www.hpcchallenge.org}%
% %
% \bibitem{blas} BLAS: Basic Linear Algebra Subprograms \url{http://www.netlib.org/blas/} (2016)
% %
% \bibitem{fft1} David H. Bailey.
%   ``FFTs in external or hierarchical memory. Journal of Supercomputing'',
%   Vol.4, pp.23--35 (1990)
% %
% \bibitem{fft2} Van Loan C.
%   ``Computational Frameworks for the Fast Fourier Transform'',
%   Society for Industrial and Applied Mathematics (1992)
% %
% \bibitem{ffte} Daisuke Takahashi.
%   A Fast Fourier Transform Package. \url{http://www.ffte.jp} (2014)
% %
% \bibitem{randomaccess} Ponnusamy R. et al.
%   ``Communication overhead on the CM5: an experimental performance evaluation'',
%   Fourth Symposium on the Frontiers of Massively Parallel Computation, pp.108--115 (1992)
% %
% \bibitem{modified} HPL Algorithm Panel Broadcast. \url{http://www.netlib.org/benchmark/hpl/algorithm.html} (2016)
% %
% %\bibitem{XACC1} Masahiro Nakao et al.
% %  ``XcalableACC: Extension of XcalableMP PGAS Language Using OpenACC for Accelerator Clusters'',
% %  Proceedings of the First Workshop on Accelerator Programming Using Directives, pp.27--36 (2014)
% %
% %\bibitem{XACC2} Masahiro Nakao et al.
% %  ``Evaluation of XcalableACC with Tightly Coupled Accelerators/InfiniBand Hybrid Communication on Accelerated Cluster'',
% %  International Journal of High Performance Computing Applications, Jan. (2019)
% %
% %\bibitem{caf-hpcc} Guohua Jin et al.
% %  ``Implementation and Performance Evaluation of the HPC Challenge Benchmarks in Coarray Fortran 2.0'',
% %  Parallel Distributed Processing Symposium (IPDPS), 2011 IEEE International, pp.1089--1100 (2011)
% %
% %\bibitem{2012Chapel-HPCC} Brad Chamberlain et al.
% %  ``Chapel HPC Challenge Entry'',
% %  \url{http://www.hpcchallenge.org/presentations/sc2012/ChapelHPCC2012.pdf} (2011)
% %
% %\bibitem{2012X10-HPCC} Olivier Tardieu et al.
% %  ``X10 for Productivity and Performance at Scale'',
% %  \url{http://www.hpcchallenge.org/presentations/sc2012/x10-hpcc.pdf} (2012)
% %
% %\bibitem{Tardieu:2014:XAP:2555243.2555245} Olivier Tardieu et al.
% %  ``X10 and APGAS at Petascale'',
% %  Proceedings of the 19th ACM SIGPLAN Symposium on Principles and Practice of Parallel Programming pp.53--66 (2014)
% %
% %\bibitem{Kennedy:2007:RFH:1238844.1238851} Kennedy Ken et al.
% %  ``The rise and fall of High Performance Fortran: an historical object lesson'',
% %  Proceedings of the third ACM SIGPLAN conference on History of programming languages, pp.7-1--7-22 (2007)
% %
% %\bibitem{tsugane2016} Keisuke Tsugane et al.
% %  ``Proposal for Dynamic Task Parallelism in PGAS Language XcalableMP'',
% %  The 6th AICS International Symposium, p.57 (2016)
\end{thebibliography}
