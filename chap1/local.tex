\section{Local-view Programming}

\subsection{Introduction}

The programmer can use {\coarrays} to specify one-sided communication in
the local-view model.
% In XMP, put/get communication and some
% synchronization functions are supported.

Depending on the environment, such one-sided communication might achieve
better performance than global communication in the global-view model.
%
However, it is more difficult and complicated to
write parallel programs in the local-view model because the programmer
must specify every detail of
parallelization, such as data mapping, work mapping, and communication.

% XMP/Fortran supports coarray as its base language Fortran 2008. Coarray
% in XMP/C has its own original syntax since the C language does not support
% Coarray.

The {\coarray} feature in XMP/Fortran is upward-compatibile with that in
Fortran 2008; that in XMP/C is defined as an extension to the base
language.

An execution entity in local-view XMP programs is referred to as an 
``image'' while a {\node} in global-view ones.
%
These two words have almost the same meaning in XMP.

\subsection{Coarray Declaration}

\begin{XCexample}
int a[10]:[*];
\end{XCexample}

\begin{XFexample}
integer a(10)[*]
\end{XFexample}

In XMP/C, the programmer declares a {\coarray} by adding ``\|:[*]|''
after the array declaration. In XMP/Fortran, the programmer declares a
{\coarray} by adding ``\|[*]|'' after the array declaration.
%The asterisk symbol is used in both languages.

\begin{mynote}
  Based on Fortran 2008, {\coarrays} should have the same size among all
  images.
\end{mynote}

{\bf Coarrays} can be accessed in expressions by remote images as well the
local image.


\subsection{Put Communication}

When a {\coarray} appears in the left-hand side of an
assignment statement, it involves {\it put} communication.

\begin{XCexample}
int a[10]:[*], b[10];

if (xmpc_this_image() == 0)
  a[0:3]:[1] = b[3:3];
\end{XCexample}

\begin{XFexample}
integer a(10)[*]
integer b(10)

if (this_image() == 1) then
  a(1:3)[2] = b(3:5)
end if
\end{XFexample}

The integer in the square bracket specifies the target image index. The
image index is zero-based, in XMP/C, or one-based, in
XMP/Fortran. \|xmpc_this_image()| in XMP/C and \|this_image()| in
XMP/Fortran return the current image index.

% \begin{mynote}
% In XMP/Fortran, image index starts with 1 while it
% uses [] (similar to C 
% style for array dimension) to specify coarray dimension based on the
% standard Fortran 2008.
% \end{mynote}

% \begin{mynote}
% When coarray dimension appears on both side, 3
% nodes (target, source, current node) involve the communication.
% \end{mynote}

In the above example, in XMP/C, an image zero puts \|b[3:3]| to
\|a[0:3]| on image one; in XMP/Fortran, an image one puts \|b(3:5)| to
\|a(1:3)| on image two. The following figure illustrates the put
communication performed in the example.

\begin{figure}
  \centering
  \includegraphics[width=0.8\textwidth]{figs/put.png}
  \caption{Remote write to a coarray}
\end{figure}

% \begin{mynote}
%   The directives in the global-view model invoke
% point-to-point
% communication. On the other hand, coarrays in the local-view model
% invoke one-sided communication.
% \end{mynote}


\subsection{Get Communication}

When a {\coarray} appears in the right-hand side of an assignment
statement, it involves {\it get} communication.

\begin{XCexample}
int a[10]:[*], b[10];

if (xmpc_this_image() == 0)
  b[3:3] = a[0:3]:[1];
\end{XCexample}

\begin{XFexample}
integer a(10)[*]
integer b(10)

if (this_image() == 1) then
  b(3:5) = a(1:3)[2]
end if
\end{XFexample}

In the above example, in XMP/C, an image 0 gets \|a[0:3]| from an image
1 and copies it to \|b[3:3]|; in XMP/Fortran, an image 1 gets \|a(1:3)|
from an image 2 and copies it to \|b(3:5)| of an image 1. The following
figure illustrates the get communication performed in the example.

\begin{figure}
  \centering
  \includegraphics[width=0.8\textwidth]{figs/get.png}
  \caption{Remote read from a coarray}
\end{figure}

\begin{myhint}
  As illustrated above, get communication involves an extra step to send
  a request to the target {\node}. Put communication achieves better
  performance than get because there is no such extra step.
\end{myhint}


\subsection{Synchronization}

% 3.5. Tutorial
% Run the following sample using 2 images.

% XMP/C program
% #include <stdio.h>
% #include <xmp.h>
% int a[10]:[*], b[10]:[*], c[10][10]:[*];

% int main(){
%   int me = xmpc_this_image();

%   for(int i=0;i<10;i++)
%     a[i] = b[i] = i + 10 * me;

%   for(int i=0;i<10;i++)
%     for(int j=0;j<10;j++)
%       c[i][j] = (i * 10 + j) + 100 * me;

%   xmp_sync_all(NULL);

%   if(xmpc_this_image() == 0){
%     a[0:3] = a[5:3]:[1];            // Get
%     for(int i=0;i<10;i++)
%       printf("%d\n", a[i]);

%     b[0:5:2] = b[0:5:2]:[1];       // Get
%     printf("\n");
%     for(int i=0;i<10;i++)
%       printf("%d\n", b[i]);

%     c[0:5][0:5]:[1] = c[0:5][0:5]; // Put
%   }
%   xmp_sync_all(NULL);

%   if(xmpc_this_image() == 1){
%     printf("\n");
%     for(int i=0;i<10;i++){
%       for(int j=0;j<10;j++){
%       printf("  %3d",c[i][j]);
%       }
%       printf("\n");
%     }
%   }

%   return 0;
% }
% XMP/Fortran program
% program main
%   implicit none
%   include "xmp_coarray.h"
%   integer :: a(10)[*], b(10)[*], c(10,10)[*]
%   integer :: i, j, me

%   me = this_image()

%   do i=1, 10
%     b(i) = (i-1) + 10 * (me - 1)
%     a(i) = b(i)
%   end do

%   do i=1, 10
%     do j=1, 10
%       c(j,i) = ((i-1) * 10 + (j-1)) + 100 * (me - 1)
%     end do
%   end do

%   sync all

%   if (this_image() == 1) then
%     a(1:3) = a(6:8)[2] ! Get
%     do i=1, 10
%       write(*,*) a(i)
%     end do

%     b(1:10:2) = b(1:10:2)[2];  ! Get
%     write(*,*) ""
%     do i=1, 10
%       write(*,*) b(i)
%     end do

%     c(1:5,1:5)[2] = c(1:5,1:5) ! Put
%   end if

%   sync all

%   if (this_image() == 2) then
%     write(*,*) ""
%     do i=1, 10
%       write(*,*) c(:,i)
%     end do
%   end if
% end program main
% In the above example, 3 coarrays a, b, c are declared. a and b are 1-dimensional arrays and c is a 2-dimensional array. The following shows the initial values of each array.

% Image 0 in XMP/C, Image 1 in XMP/Fortran
% a : from 0 to 9
% b : from 0 to 9
% c : from 0 to 99
% Image 1 in XMP/C, Image 2 in XMP/Fortran
% a : from 10 to 19
% b : from 10 to 19
% c : from 100 to 199
% 3.5.1. One-sided communication for contiguous region
% In the first get communication, in XMP/C, image 0 gets a[5:3] from image 1 and stores them to a[0:3]. In XMP/Fortran, image 1 gets a[6:8] from image 2 and stores them to a(1:3)

% After the communication, array a has the following values.

% 15
% 16
% 17
% 3
% 4
% 5
% 6
% 7
% 8
% 9
% 3.5.2. One-sided communication for discontiguous region
% In the second get communication, in XMP/C, image 0 gets b[0:5:2] from image 1 and stores them to b[0:5:2]. In XMP/Fortran, image 1 gets b(1:10:2) from image 2 and stores them to b(1:10:2).

% After the communication, array b has the following values.

% 10
% 1
% 12
% 3
% 14
% 5
% 16
% 7
% 18
% 9
% 3.5.3. One-sided communication for multi-dimensional array
% In the put communication, in XMP/C, image 0 puts c[0:5][0:5] to on c[0:5][0:5] image 1. In XMP/Fortran, image 1 puts c(1:5,1:5) to c(1:5,1:5) on image 2. The communication has the block-strided communication pattern.

% After the communication, array c has the following values.

%   0    1    2    3    4  105  106  107  108  109
%  10   11   12   13   14  115  116  117  118  119
%  20   21   22   23   24  125  126  127  128  129
%  30   31   32   33   34  135  136  137  138  139
%  40   41   42   43   44  145  146  147  148  149
% 150  151  152  153  154  155  156  157  158  159
% 160  161  162  163  164  165  166  167  168  169
% 170  171  172  173  174  175  176  177  178  179
% 180  181  182  183  184  185  186  187  188  189
% 190  191  192  193  194  195  196  197  198  199


%\subsection{Synchronization Statements}

\subsubsection{sync all}

% Here, we introduce ``sync all,'' which is most frequently used among
% synchronization features for coarrays.

\begin{XCexample}
void xmp_sync_all(int *status)
\end{XCexample}

\begin{XFexample}
sync all
\end{XFexample}

At ``sync all,'' each image waits until all issued one-sided
communication is complete and then performs barrier synchronization
among the all images.

\begin{figure}
  \centering
  \includegraphics[width=0.7\textwidth]{figs/sync_all.png}
  \caption{{\tt sync all}}
\end{figure}

In the above example, the left image puts data to the right image and
both {\nodes} invoke \|sync all|. When both {\nodes} returns from it, the
execution continues to the following satements.

% \begin{XCexample}
% void xmp_sync_all(int *status)
% \end{XCexample}

% \begin{XFexample}
% sync all
% \end{XFexample}

% Each image waits until all one-sided communication issued is complete,
% and performs barrier synchronization.
% For details, see Tutorial (Local-view).


\subsubsection{sync images}

\begin{XCexample}
void xmp_sync_images(int num, int *image-set, int *status)
\end{XCexample}

\begin{XFexample}
sync images (image-set)
\end{XFexample}

Each image in the specified image set waits until all one-sided
communication issued is complete, and performs barrier synchronization
among the images.

\begin{XCexample}
int image_set[3] = {0,1,2};
xmp_sync_images(3, image_set, NULL);
\end{XCexample}

\begin{XFexample}
integer :: image_set(3) = (/ 1, 2, 3/)
sync images (image_set)
\end{XFexample}


\subsubsection{sync memory}

\begin{XCexample}
void xmp_sync_memory(int *status)
\end{XCexample}

\begin{XFexample}
sync memory
\end{XFexample}

Each image waits until all one-sided communication is complete. This
function/statement does not imply barrier synchronization, unlike
\|sync all| and \|sync images|, and therefore can be locally executed. 

% 20.2. Arguments

% XMP/C program
% void xmp_sync_all(int *status)
% void xmp_sync_images(int *status)
% void xmp_sync_memory(int *status)

% XMP/Fortran program
% sync all [stat=..] [errmsg=..]
% sync images (image-set) [stat=..] [errmsg=..]
% sync memory [stat=..] [errmsg=..]

% In XMP/C, if synchronization is successful, “XMP_STAT_SUCCESS” which is
% the constant defined in xmp.h is assigned to status. If any images have
% already ended, “XMP_STAT_STOPPED_IMAGE” is substituted to status. In
% case of other errors, a value other than the above two values is
% assigned to status.

% Similarly, if synchronization is successful in XMP/Fortran,
% “STAT_STOPPED_IMAGE” is assigned to the variable on the right-hand side
% of stat=, and if any images have already ended, “STAT_STOPPED_IMAGE” is
% assigned. In case of other errors, a value other than the above two
% values is assigned.

% Hint

% In XMP/Fortran, if you omit stat= and errmsg=, synchronization speed
% will be faster. In XMP/C, assignment of status can be omitted by using
% NULL like xmp_sync_all (NULL);

% \subsection{{\tt post}/{\tt wait} Construct}

% \subsection{{\tt lock}/{\tt unlock} Construct}

