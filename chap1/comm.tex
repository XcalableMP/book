\section{Data Communication}

\subsection{{\bf shadow} Directive and {\bf reflect} Construct}

Stencil computation frequently appears in scientific simulation programs,
where, to update an array element \|a[i]|, its neighboring elements
\|a[i-1]| and \|a[i+1]| are referenced. If \|a[i]| is on the boundary
region of a block-{\darray} on a {\node}, \|a[i+1]| may reside on
another (neighboring) {\node}.

Since it involves large overhead to copy \|a[i+1]| from the neighboring
{\node} to update each \|a[i]|, a technique of copying collectively the
elements on the neighboring {\node} to the area added to the {\darray}
on each {\node} is usually adopted. In XMP, such additional area is 
called ``shadow.''

\subsubsection{Declaring Shadow}

%\paragraph{Widths of lower/upper bounds are the same}

Shadow areas can be declared with the \|shadow| directive. In the example
below, an array \|a| has shadow areas of width one on both the lower and
upper bounds.

\begin{XCexample}
#pragma xmp nodes p[4]
#pragma xmp template t[16]
#pragma xmp distribute t[block] onto p
double a[16];
#pragma xmp align a[i] with t[i]
#pragma xmp shadow a[1]
\end{XCexample}

\begin{XFexample}
!$xmp nodes p(4)
!$xmp template t(16)
!$xmp distribute t(block) onto p
real :: a(16)
!$xmp align a(i) with t(i)
!$xmp shadow a(1)
\end{XFexample}

\begin{figure}
  \centering
  \includegraphics[width=\textwidth]{figs/shadow.png}
  \caption{Example of {\tt shadow} directive (1).}
\end{figure}

In the figure above, colored elements are those that each {\node} owns and
white ones are shadow.

\begin{mynote}
  Arrays distributed in a cyclic manner cannot have shadow.  
\end{mynote}

%\paragraph{Widths of lower/upper bounds are different}

In some programs, it is natural that the widths of the shadow area on
the lower and upper bounds are different. There is also a case where the
shadow area exists only on either of the bounds. In the example below,
it is declared that a {\darray} \|a| has a shadow area of width
one only on the upper bound.

\begin{XCexample}
#pragma xmp nodes p[4]
#pragma xmp template t[16]
#pragma xmp distribute t(block) onto p
double a[16];
#pragma xmp align a[i] with t[i]
#pragma xmp shadow a[0:1]
\end{XCexample}

\begin{XFexample}
!$xmp nodes p(4)
!$xmp template t(16)
!$xmp distribute t(block) onto p
real :: a(16)
!$xmp align a(i) with t(i)
!$xmp shadow a(0:1)
\end{XFexample}

\begin{figure}
  \centering
  \includegraphics[width=\textwidth]{figs/shadow_uneven.png}
  \caption{Example of {\tt shadow} directive (2).}
\end{figure}

The values on the left- and right-hand sides of a colon designate the
widths on the lower and upper bounds, respectively.

\subsubsection{Updating Shadow}

%\paragraph{General}

To copy data to shadow areas from neighboring {\nodes}, use the \|reflect|
construct. In the example below, the shadow areas of an array \|a| that
are of width one on both the upper and lower bounds are updated.

\begin{XCexample}
#pragma xmp reflect (a)

#pragma xmp loop on t[i]
for(int i=1;i<15;i++)
  a[i] = (a[i-1] + a[i] + a[i+1])/3;
\end{XCexample}

\begin{XFexample}
!$xmp reflect (a)

!xmp loop on t(i)
do i=2, 15
  a(i) = (a(i-1) + a(i) + a(i+1))/3
enddo
\end{XFexample}

\begin{figure}
  \centering
  \includegraphics[width=\textwidth]{figs/reflect.png}
  \caption{Example of {\tt reflect} construct (1).}
\end{figure}

With this \|reflect| directive, in XMP/C, {\node} \|p[1]| sends an element
\|a[4]| to the shadow area on the upper bound on {\node} \|p[0]| and
\|a[7]| to the shadow area on the lower bound on \|p[2]|; \|p[0]| sends
an element \|a[3]| to the shadow area on the lower bound on \|p[1]|, and
\|p[2]| sends \|a[8]| to the shadow area on the upper bound on \|p[1]|.

Similarly, in XMP/Fortran, {\node} \|p(2)| sends an element \|a(5)| to the
shadow area on the upper bound on {\node} \|p(1)| and \|a(8)| to the shadow
area on the lower bound on \|p(3)|; \|p(1)| sends an element \|a(4)| to
the shadow area on the lower bound on \|p(2)|, and \|p(3)| sends \|a(9)|
to the shadow area on the upper bound on \|p(2)|.

%\paragraph{Specify Width}

The default behavior of a \|reflect| directive is to update the whole of
the shadow area declared by the \|shadow| directive. However, there are
some cases where a specific part of the shadow area is to be updated to
reduce the communication cost at a point of the code.

To update only a specific part of the shadow area, add the \|width|
clause to the \|reflect| directive.

The values on the left- and right-hand sides of a colon in the \|width|
clause designate the widths on the lower and upper bounds to be updated,
respectively. In the example below, only the shadow area on the upper
bound is updated.

\begin{XCexample}
#pragma xmp reflect (a) width(0:1)
\end{XCexample}

\begin{XFexample}
!$xmp reflect (a) width(0:1)
\end{XFexample}

\begin{figure}
  \centering
  \includegraphics[width=\textwidth]{figs/reflect_width.png}
  \caption{Example of {\tt reflect} construct (2).}
\end{figure}

\begin{mynote}
  If the widths of the shadow areas to be updated on the upper and lower 
  bounds are equal, that is, for example, \|width(1:1)|, you can
  abbreviate it as \|width(1)|.
\end{mynote}

\begin{mynote}
  It is not possible to update the shadow area on a particular {\node}
  because \|reflect| is a collective operation.
\end{mynote}

% If no shadow area is specified on the lower bound, the reflect directive
% does not update it with or without a width clause. The below figure
% illustrates the behavior of a reflect directive for a distributed array
% a having a shadow area of width one only on the upper bound.

% \begin{figure}
%   \centering
%   \includegraphics{figs/reflect_uneven.png}
% \end{figure}

%\paragraph{Update periodic shadow}

The \|reflect| directive does not update either the shadow area on the
lower bound on the leading {\node} or that on the upper bound on the last
{\node}. However, the values in such areas are needed for stencil
computation if periodic boundary conditions are used in the computation.

To update such areas, add a \|periodic| qualifier into the \|width|
clause. Let’s look at the following example where an array \|a| having
shadow areas of width one on both the lower and upper bounds appears.

\begin{XCexample}
#pragma xmp reflect (a) width(/periodic/1:1)
\end{XCexample}

\begin{XFexample}
!$xmp reflect (a) width(/periodic/1:1)
\end{XFexample}

\begin{figure}
  \centering
  \includegraphics[width=\textwidth]{figs/reflect_periodic.png}
  \caption{Example of periodic {\tt reflect} construct.}
\end{figure}

The \|periodic| qualifier has the following effects, in addition to that of
a normal \|reflect| directive: in XMP/C, {\node} \|p[0]| sends an element
\|a[0]| to the shadow area on the upper bound on {\node} \|p[3]|, and
\|p[3]| sends \|a[15]| to the shadow area on the lower bound on \|p[0]|;
in XMP/Fortran, {\node} \|p(1)| sends an element \|a(1)| to the shadow area
on the upper bound on {\node} \|p(4)|, and \|p(4)| sends \|a(16)| to the
shadow area on the lower bound on \|p(1)|.

% \begin{mynote}
%   If the widths of the shadow areas to be updated on
% the upper and lower 
% bounds are equal, as shown by width(/periodic/1:1) in the above example,
% you can abbreviate it as width(/periodic/1).
% \end{mynote}

%\subsubsection{Multi-dimensional Shadow}

The \|shadow| directive and \|reflect| construct can be applied to
arrays distributed in multiple dimensions. The following programs are the
examples for two-dimensional {\bf distribution}.

\begin{XCexample}
#pragma xmp nodes p[3][3]
#pragma xmp template t[9][9]
#pragma xmp distribute t[block][block] onto p
double a[9][9];
#pragma xmp align a[i][j] with t[i][j]
#pragma xmp shadow a[1][1]
   :
#pragma xmp reflect (a)
\end{XCexample}

\begin{XFexample}
!$xmp nodes p(3,3)
!$xmp template t(9,9)
!$xmp distribute t(block,block) onto p
real :: a(9,9)
!$xmp align a(j,i) with t(j,i)
!$xmp shadow a(1,1)
   :
!$xmp reflect (a)
\end{XFexample}

\begin{figure}
  \centering
  \includegraphics[width=0.9\columnwidth]{figs/multi1.png}
  \caption{Example of multi-dimensional shadow (1).}
\end{figure}

The central {\node} receives data from the surrounding eight
{\nodes} to update its shadow areas. The shadow areas of the other {\nodes}
are also updated, which is omitted in the figure.

For some applications, data from ordinal directions are not
necessary. In such a case, the data communication from/to the ordinal
directions can be avoided by adding the \|orthogonal| clause to a
\|reflect| construct.

\begin{XCexample}
#pragma xmp reflect (a) orthogonal
\end{XCexample}

\begin{XFexample}
!$xmp reflect (a) orthogonal
\end{XFexample}

\begin{figure}
  \centering
  \includegraphics[width=0.9\columnwidth]{figs/multi_orthogonal.png}
  \caption{Example of multi-dimensional shadow (2).}
\end{figure}

\begin{mynote}
  The \|orthogonal| clause is effective only for arrays
  more than one dimension of which is distributed.
\end{mynote}

Besides, you can also add shadow areas to only specified dimension.

\begin{XCexample}
#pragma xmp nodes p[3]
#pragma xmp template t[9]
#pragma xmp distribute t[block] onto p
double a[9][9];
#pragma xmp align a[i][*] with t[i]
#pragma xmp shadow a[1][0]
  :
#pragma xmp reflect (a)
\end{XCexample}

\begin{XFexample}
!$xmp nodes p[3]
!$xmp template t[9]
!$xmp distribute t[block] onto p
real :: a(9,9)
!$xmp align a(*,i) with t(i)
!$xmp shadow a(0,1)
  :
!$xmp reflect (a)
\end{XFexample}

\begin{figure}
  \centering
  \includegraphics[width=0.9\columnwidth]{figs/1of2.png}
  \caption{Example of multi-dimensional shadow (3).}
\end{figure}

For the array \|a|, 0 is specified as the shadow width in
non-distributed dimensions.


\subsection{{\tt gmove} Construct}

The programmaers can specify a communication of {\darrays} in
the form of assignment statements by using the \|gmove| construct.
%
In other words, with the \|gmove| construct, any array assignment
between two arrays (i.e. {\it global data movement}) that may involve
inter-node communication can be specified.

There are three modes of \|gmove|; ``collective mode,'' ``in mode,'' and
``out mode.''
% While collective mode executes two-sided communication among the
% executing nodes, in/out modes execute one-sided communication among
% tasks with a task directive. While in mode uses get communication, out
% mode uses put communication.

\subsubsection{Collective Mode}

%\paragraph{Distributed array}

% Copying a part of array a to array b. Array assignment statements in a
% gmove construct uses triplet.

The global data movement involved by a {\it collective} \|gmove| is performed
collectively, and results in implicit synchronization among the 
{\bf executing nodes}.

\begin{XCexample}
#pragma xmp nodes p[4]
#pragma xmp template t[16]
#pragma xmp distribute t[block] onto p
int a[16], b[16];
#pragma xmp align a[i] with t[i]
#pragma xmp align b[i] with t[i]
     :
#pragma xmp gmove
  a[9:5] = b[0:5];
\end{XCexample}

\begin{XFexample}
!$xmp nodes p(4)
!$xmp template t(16)
!$xmp distribute t(block) onto p
integer :: a(16), b(16)
!$xmp align a(i) with t(i)
!$xmp align b(i) with t(i)
     :
!$xmp gmove
  a(10:14) = b(1:5)
\end{XFexample}

\begin{figure}
  \centering
  \includegraphics[width=\textwidth]{figs/gmove.png}
  \caption{Collective {\tt gmove} (1).}
\end{figure}

In XMP/C, \|p[0]| sends \|b[0]|-\|b[3]| to \|p[2]|-\|p[3]|, and \|p[1]|
sends \|b[4]| to \|p[3]|. Similarly, in XMP/Fortran, \|p(1)| sends
\|b(1)|-\|b(4)| to \|p(3)|-\|p(4)|, and \|p(2)| sends \|b(5)| to \|p(4)|.

\begin{XCexample}
#pragma xmp nodes p[4]
#pragma xmp template t1[16]
#pragma xmp template t2[16]
#pragma xmp distribute t1[cyclic] onto p
#pragma xmp distribute t2[block] onto p
int a[16], b[16];
#pragma xmp align a[i] with t1[i]
#pragma xmp align b[i] with t2[i]
     :
#pragma xmp gmove
  a[9:5] = b[0:5];
\end{XCexample}

\begin{XFexample}
!$xmp nodes p(4)
!$xmp template t1(16)
!$xmp template t2(16)
!$xmp distribute t1(cyclic) onto p
!$xmp distribute t2(block) onto p
integer :: a(16), b(16)
!$xmp align a(i) with t1(i)
!$xmp align b(i) with t2(i)
     :
!$xmp gmove
  a(10:14) = b(1:5)
\end{XFexample}

\begin{figure}
  \centering
  \includegraphics[width=\textwidth]{figs/gmove_cyclic.png}
  \caption{Collective {\tt gmove} (2).}
\end{figure}

While array \|a| is distributed in a cyclic manner, array \|b| is
distributed in a block manner.

In XMP/C, \|p[0]| sends \|b[0]| and \|b[4]| to \|p[2]| and
\|p[3]|. \|p[1]| sends \|b[1]| to \|p[2]|. Each element of \|p[2]| and
\|p[3]| will be copied locally. Similarly, in XMP/Fortran, \|p(1)| sends
\|b(1)| and \|b(5)| to \|p(3)| and \|p(4)|. \|p(2)| sends \|b(2)| to
\|p(3)|. Each element of \|p(3)| and \|p(4)| will be copied locally.

% \begin{mynote}
%   If the number of elements specified on the
% right-hand side is other than one, it will not work properly if the
%   number of elements differs between the right-hand side and the
%   left-hand side.
% \end{mynote}

By using this method, the {\bf distribution} of an array can be ``changed''
during computation.

\begin{XCexample}
#pragma xmp nodes p[4]
#pragma xmp template t1[16]
#pragma xmp template t2[16]
int W[4] = {2,4,8,2};
#pragma xmp distribute t1[gblock(W)] onto p
#pragma xmp distribute t2[block] onto p
int a[16], b[16];
#pragma xmp align a[i] with t1[i]
#pragma xmp align b[i] with t2[i]
     :
#pragma xmp gmove
  a[:] = b[:];
\end{XCexample}

\begin{XFexample}
!$xmp nodes p(4)
!$xmp template t1(16)
!$xmp template t2(16)
integer :: W(4) = (/2,4,7,3/)
!$xmp distribute t1(gblock(W)) onto p
!$xmp distribute t2(block) onto p
integer :: a(16), b(16)
!$xmp align a(i) with t1(i)
!$xmp align b(i) with t2(i)
     :
!$xmp gmove
  a(:) = b(:)
\end{XFexample}

\begin{figure}
  \centering
  \includegraphics[width=\textwidth]{figs/gmove_change.png}
  \caption{Collective {\tt gmove} (3).}
\end{figure}

In this example, the elements of an array \|b| that is distributed in
a block manner are copied to the corresponding elements of an array \|a|
that is distributed in a generalized-block manner.
%
For the arrays \|a| and \|b|, communication occurs if corresponding elements
reside in different {\nodes} (arrows illustrate communication between {\nodes} in
the figures).

%\paragraph{Scalar}

In the assignment statement, if a scalar (i.e. one element of an array or
a variable) is specified on the right-hand side and an array section are
specified on the left-hand side, a broadcast communication occurs for it.

\begin{XCexample}
#pragma xmp nodes p[4]
#pragma xmp template t[16]
#pragma xmp distribute t[block] onto p
int a[16], b[16];
#pragma xmp align a[i] with t[i]
#pragma xmp align b[i] with t[i]
     :
#pragma xmp gmove
  a[9:5] = b[0];
\end{XCexample}

\begin{XFexample}
!$xmp nodes p(4)
!$xmp template t(16)
!$xmp distribute t(block) onto p
integer :: a(16), b(16)
!$xmp align a(i) with t(i)
!$xmp align b(i) with t(i)
     :
!$xmp gmove
  a(10:14) = b(1)
\end{XFexample}

\begin{figure}
  \centering
  \includegraphics[width=\textwidth]{figs/gmove_one_element.png}
  \caption{Collective {\tt gmove} (4).}
\end{figure}

In this example, in XMP/C, an array element \|b[0]| of {\node} \|p[0]| will be
broadcasted to the specified array section on {\node} \|p[2]| and
\|p[3]|. Similarly, in XMP/Fortran, an array element \|b(1)| of {\node}
\|p(1)| will be broadcasted to the specified array section on {\node} \|p(3)| and \|p(4)|.

%\paragraph{Duplicated array and scalar}

Not only {\darrays} but also replicated arrays can be specified
on the right-hand side.

\begin{XCexample}
 #pragma xmp nodes p[4]
 #pragma xmp template t[16]
 #pragma xmp distribute t[block] onto p
 int a[16], b[16], c;
 #pragma xmp align a[i] with t[i]
      :
#pragma xmp gmove
   a[9:5] = b[0:5];
\end{XCexample}

\begin{XFexample}
 !$xmp nodes p(4)
 !$xmp template t(16)
 !$xmp distribute t(block) onto p
 integer :: a(16), b(16), c
 !$xmp align a(i) with t(i)
      :
!$xmp gmove
   a(10:14) = b(1:5)
\end{XFexample}

In this example, a replicated array \|b| is locally copied to
{\darray} \|a| without communication.

%\paragraph{Distributed array with different dimension}

\begin{XCexample}
#pragma xmp nodes p[4]
#pragma xmp template t1[8]
#pragma xmp template t2[16]
#pragma xmp distribute t1[block] onto p
#pragma xmp distribute t2[block] onto p
int a[8][16], b[8][16];
#pragma xmp align a[i][*] with t1[i]
#pragma xmp align b[*][i] with t2[i]
     :
#pragma xmp gmove
  a[0][:] = b[0][:];
\end{XCexample}

\begin{XFexample}
!$xmp nodes p(4)
!$xmp template t1(8)
!$xmp template t2(16)
!$xmp distribute t1(block) onto p
!$xmp distribute t2(block) onto p
integer :: a(16,8), b(8,16)
!$xmp align a(*,i) with t1(i)
!$xmp align b(i,*) with t2(i)
     :
#pragma xmp gmove
  a(:,1) = b(:,1)
\end{XFexample}

\begin{figure}
  \centering
  \includegraphics[width=0.9\columnwidth]{figs/gmove_different.png}
  \caption{Collective {\tt gmove} (4).}
\end{figure}

In this example, in XMP/C, \|b[0][0:2]| on \|p[0]|, \|b[0][2:2]| of
\|p[1]|, \|b[0][4:2]| on \|p[2]| and \|b[0][6:2]| on \|p[3]| are copied
to \|a[0][:]| on \|p[0]|. Similarly, in XMP/Fortran, \|b(1:2,1)| on
\|p(1)|, \|b(3:4,1)| of \|p(2)|, \|b(5:6,1)| on \|p(3)| and \|b(7:8,1)|
on \|p(4)| are copied to \|a(:,1)| on \|p(1)|.


\subsubsection{In Mode}

%It operates as in mode by setting in clause to gmove directive.

The right-hand side data of the assignment, all or part of which may
reside outside the {\enset}, can be transferred from its owner
{\nodes} to the {\bf executing nodes} with an {\it in} \|gmove|.

\begin{XCexample}
#pragma xmp nodes p[4]
#pragma xmp template t[4]
#pragma xmp distribute t[block] onto p
double a[4], b[4];
#pragma xmp align a[i] with t[i]
#pragma xmp align b[i] with t[i]
   :
#pragma xmp task on p[0:2]
#pragma xmp gmove in
  a[0:2] = b[2:2]
#pragma xmp end task
\end{XCexample}

\begin{XFexample}
!$xmp nodes p(4)
!$xmp template t(4)
!$xmp distribute t(block) onto p
real :: a(4), b(4)
!$xmp align a(i) with t(i)
!$xmp align b(i) with t(i)
   :
!$xmp task on p(1:2)
!$xmp gmove in
  a(1:2) = b(3:4)
!$xmp end task
\end{XFexample}

In this example, the \|task| directive divides four {\nodes} into
two sets, the first-half and the second-half. A \|gmove| construct that
is in an {\it in} mode copies data using a {\it get} operation from
the second-half {\node} to the first-half {\node}.

\begin{figure}
  \centering
  \includegraphics[width=0.9\columnwidth]{figs/gmove_in.png}
  \caption{In {\tt gmove}.}
\end{figure}


\subsubsection{Out Mode}

For the left-hand side data of the assignment, all or part of which may
reside outside the {\enset}, the corresponding elements can be
transferred from the {\bf executing nodes} to its owner {\nodes} with an {\it out}
\|gmove| construct.

\begin{XCexample}
#pragma xmp nodes p[4]
#pragma xmp template t[4]
#pragma xmp distribute t[block] onto p
double a[4], b[4];
#pragma xmp align a[i] with t[i]
#pragma xmp align b[i] with t[i]
   :
#pragma xmp task on p[0:2]
#pragma xmp gmove out
  b[2:2] = a[0:2]
#pragma xmp end task
\end{XCexample}

\begin{XFexample}
!$xmp nodes p(4)
!$xmp template t(4)
!$xmp distribute t(block) onto p
real :: a(4), b(4)
!$xmp align a(i) with t(i)
!$xmp align b(i) with t(i)
   :
!$xmp task on p(1:2)
!$xmp gmove out
  b(3:4) = a(1:2)
!$xmp end task
\end{XFexample}

A \|gmove| construct that is in {\it out} mode copies data using a {\it put}
communication from the first-half {\nodes} to the second-half {\nodes}.

\begin{figure}
  \centering
  \includegraphics[width=0.9\columnwidth]{figs/gmove_out.png}
  \caption{Out {\tt gmove}.}
\end{figure}


\subsection{{\bf barrier} Construct}

The \|barrier| construct executes a barrier synchronization.

\begin{XCexample}
#pragma xmp barrier
\end{XCexample}

\begin{XFexample}
!$xmp barrier
\end{XFexample}

You can specify a {\nset} on which the barrier synchroniation is to be
performed by using the \|on| clause. In the below example, a barrier
synchronization is performed among the first two {\nodes} of \|p|.

\begin{XCexample}
#pragma xmp barrier on p[0:2]
\end{XCexample}

\begin{XFexample}
!$xmp barrier on p(1:2)
\end{XFexample}


\subsection{{\bf reduction} Construct}

This construct performs a {\it reduction} operation. It has the same
meaning as the \|reduction| clause of the \|loop| construct, but this
construct can be specified anywhere as an {\it executable} construct.

\begin{XCexample}
#pragma xmp nodes p[4]
  :
sum = xmpc_node_num() + 1;
#pragma xmp reduction (+:sum)
\end{XCexample}

\begin{XFexample}
!$xmp nodes p(4)
  :
sum = xmp_node_num()
!$xmp reduction (+:sum)
\end{XFexample}

\begin{figure}
  \centering
  \includegraphics[width=0.9\columnwidth]{figs/reduction.png}
  \caption{{\tt reduction} construct (1).}
\end{figure}

You can specify the {\enset} by using the \|on| clause. In the
below example, only the values on the last two of the four {\nodes} are
targeted by the \|reduction| construct.

\begin{XCexample}
#pragma xmp nodes p[4]
  :
sum = xmpc_node_num() + 1;
#pragma xmp reduction (+:sum) on p[2:2]
\end{XCexample}

\begin{XFexample}
!$xmp nodes p(4)
  :
 sum = xmp_node_num()
 !$xmp reduction (+:sum) on p(3:4)
\end{XFexample}

\begin{figure}
  \centering
  \includegraphics[width=0.9\columnwidth]{figs/reduction_on.png}
  \caption{{\tt reduction} construct (2).}
\end{figure}

The operators you can use in the \|reduction| construct are as follows:

\begin{XCexample}
+
*
-
&
|
^
&&
||
max
min
\end{XCexample}

\begin{XFexample}
+
*
-
.and.
.or.
.eqv.
.neqv.
max
min
iand
ior
ieor
\end{XFexample}

\begin{mynote}
% Since the \|reduction| clause needs a loop statement,
% operators of
% firstmax, firstmin, lastmax, and lastmin are required. But, since the
% reduction directive does not need a loop statement, there are no such
% operators.
  In contrast to the \|reduction| clause of the \|loop| construct, which
  precedes loops, the \|reduction| construct does not accept operators of
  \|firstmax|, \|firstmin|, \|lastmax|, and \|lastmin|.
\end{mynote}

\begin{mynote}
  Similar to the \|reduction| clause, the \|reduction| construct may
  generate slightly different results in a parallel execution from those
  in a sequential execution, because the results depends on the order of
  combining the value.
\end{mynote}


\subsection{{\bf bcast} Construct}

The \|bcast| construct broadcasts the values of the variables on the
{\node} specified by the \|from| clause, that is, the {\it root node}, to
the {\nset} specified by the \|on| clause.
%
If there is no \|from| clause, the first {\node} of the {\enset}
is selected as the root {\node}.
%
If there is no \|on| clause, the current {\enset} of the
construct is selected as the {\enset}.

In the below example, the first {\node} of the {\nset} \|p|, that is,
\|p[0]| or \|p(1)|, is the root {\node}.

\begin{XCexample}
#pragma xmp nodes p[4]
  :
num = xmpc_node_num() + 1;
#pragma xmp bcast (num)
\end{XCexample}

\begin{XFexample}
!$xmp nodes p(4)
  :
num = xmp_node_num()
!$xmp bcast (num)
\end{XFexample}

\begin{figure}
  \centering
  \includegraphics[width=0.9\columnwidth]{figs/bcast.png}
  \caption{{\tt bcast} construct (1).}
\end{figure}

In the below example, the last {\node}, that is, \|p[3]| or \|p(4)|, is the \|from| clause.

\begin{XCexample}
#pragma xmp nodes p[4]
  :
num = xmpc_node_num() + 1;
#pragma xmp bcast (num) from p[3]
\end{XCexample}

\begin{XFexample}
!$xmp nodes p(4)
  :
num = xmp_node_num()
!$xmp bcast (num) from p(4)
\end{XFexample}

\begin{figure}
  \centering
  \includegraphics[width=0.9\columnwidth]{figs/bcast_from.png}
  \caption{{\tt bcast} construct (2).}
\end{figure}

In the below example, only the last three of four {\nodes}
are included by the {\enset} of the \|bcast| construct.

\begin{XCexample}
#pragma xmp nodes p[4]
  :
sum = xmpc_node_num() + 1;
#pragma xmp bcast (num) from p[3] on p[1:3]
\end{XCexample}

\begin{XFexample}
!$xmp nodes p(4)
  :
 sum = xmp_node_num()
 !$xmp bcast (num) from p(4) on p(2:4)
\end{XFexample}

\begin{figure}
  \centering
  \includegraphics{figs/bcast_from_on.png}
  \caption{{\tt bcast} construct (3).}
\end{figure}


\subsection{{\bf wait\_async} Construct}

Communication directives (i.e. \|reflect|, \|gmove|, \|reduction|,
\|bcast|, and \|reduce_shadow|) can perform asynchronous communication if
the \|async| clause is added. The \|wait_async| construct is used to
guarantee the completion of such an asynchronous communication.

\begin{XCexample}
#pragma xmp bcast (num) async(1)
    :
#pragma xmp wait_async (1)
\end{XCexample}

\begin{XFexample}
!$xmp bcast (num) async(1)
        :
!$xmp wait_async (1)
\end{XFexample}

Since the \|bcast| directive has an \|async| clause, communication may
not be completed immediately after the \|bcast| directive. The
completion of that communication is guaranteed with the \|wait_async|
construct having the same value as that of the \|async| clause.
%
Therefore, between the \|bcast| construct and the \|wait_async|
constructs, you may not reference the target variable of the \|bcast|
directive.

\begin{myhint}
  Asynchronous communication can be overlapped with the following
  computation to hide its overhead.
\end{myhint}

\begin{mynote}
  Expressions that can be specified as {\it tags} in the \|async| clause are
  of type int, in XMP/C, or integer, in XMP/Fortran.
\end{mynote}


\subsection{{\bf reduce\_shadow} Construct}

The \|reduce_shadow| directive adds the value of a shadow object to the
corresponding data object of the array.

\begin{XCexample}
#pragma xmp nodes p[2]
#pragma xmp template t[8]
#pragma xmp distribute t[block] onto p
int a[8];
#pragma xmp align a[i] with t[i]
#pragma xmp shadow a[1]
 :
#pragma xmp loop on t[i]
  for(int i=0;i<8;i++)
    a[i] = i+1;

#pragma xmp reflect (a)
#pragma xmp reduce_shadow (a)
\end{XCexample}

\begin{XFexample}
!$xmp nodes p(2)
!$xmp template t(8)
!$xmp distribute t(block) onto p
  integer a(8)
!$xmp align a(i) with t(i)
!$xmp shadow a(1)

!$xmp loop on t(i)
  do i=1, 8
    a(i) = i
  enddo

!$xmp reflect (a)
!$xmp reduce_shadow (a)
\end{XFexample}

% The \|shadow| directive adds a shadow are of width one to the
% distributed array \|a| of each node. Next, the \|reflect| construct
% updates the shadow area. Finally, the \|reduce_shadow| construct adds
% the value of the shadow to the value of the source element.

For the above example, in XMP/C, \|a[3]| on \|p[0]| has a value of eight,
and \|a[4]| on \|p[1]| has a value of ten. Similarly, in XMP/Fortran,
\|a(4)| of \|p(1)| has a value of eight, and \|a(5)| on \|p(2)| has a
value of ten.

\begin{figure}
  \centering
  \includegraphics[width=\textwidth]{figs/reduce_shadow.png}
  \caption{{\tt reduce\_shadow} construct (1).}
\end{figure}

The programmers can add the \|periodic| modifier to the \|width| clause
to reduce shadow objects to the corresponding data object periodically.

\begin{XCexample}
#pragma xmp reflect (a) width(/periodic/1)
#pragma xmp reduce_shadow (a) width(/periodic/1)
\end{XCexample}

\begin{XFexample}
!$xmp reflect (a) width(/periodic/1)
!$xmp reduce_shadow (a) width(/periodic/1)
\end{XFexample}

\begin{figure}
  \centering
  \includegraphics[width=\textwidth]{figs/reduce_shadow_periodic.png}
  \caption{{\tt reduce\_shadow} construct (2).}
\end{figure}

In addition to the first example, in XMP/C, \|a[0]| on \|p[0]| has a
value of two, and \|a[7]| on \|p[1]| has a value of 16. Similarly, in
XMP/Fortran, \|a(1)| in \|p(1)| has a value of two, and \|a(8)| in
\|p(2)| has a value of 16.