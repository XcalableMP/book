%%%%%%%%%%%%%%%%%%%% author.tex %%%%%%%%%%%%%%%%%%%%%%%%%%%%%%%%%%%
%
% sample root file for your "contribution" to a contributed volume
%
% Use this file as a template for your own input.
%
%%%%%%%%%%%%%%%% Springer %%%%%%%%%%%%%%%%%%%%%%%%%%%%%%%%%%
% RECOMMENDED %%%%%%%%%%%%%%%%%%%%%%%%%%%%%%%%%%%%%%%%%%%%%%%%%%%
\documentclass[graybox]{svmult}
% choose options for [] as required from the list
% in the Reference Guide
\usepackage{type1cm}        % activate if the above 3 fonts are
                            % not available on your system
%
\usepackage{makeidx}         % allows index generation
%\usepackage{graphicx}        % standard LaTeX graphics tool
                             % when including Fig.~files
\usepackage{multicol}        % used for the two-column index
\usepackage[bottom]{footmisc}% places footnotes at page bottom
\usepackage{newtxtext}       % 
\usepackage{newtxmath}       % selects Times Roman as basic font
% see the list of further useful packages
% in the Reference Guide
%\makeindex             % used for the subject index
                       % please use the style svind.ist with
                       % your makeindex program
%%%%%%%%%%%%%%%%%%%%%%%%%%%%%%%%%%%%%%%%%%%%%%%%%%%%%%%%%%%%%%%%%%%%%%%%%%%%%%%%%%%%%%%%%
% \usepackage{amsmath}
% \usepackage{ascmac}
\usepackage[dvipdfmx]{graphicx}  % for EPS and PDF 
% \usepackage{url}
\usepackage{fancyvrb}
\usepackage{multirow}
% \usepackage{makeidx}
% \usepackage{float}
% \usepackage[dvipdfmx]{color}
% \usepackage{ulem}
% \usepackage[switch*,pagewise]{lineno}
% \usepackage[dvipdfm,bookmarkstype=toc,urlcolor=black,%
%     linkcolor=black,citecolor=black,bookmarks=false]{hyperref}
% \usepackage{fancyhdr}

\usepackage{listings}
\lstset{%
 language={C},
% basicstyle={\scriptsize},%
% identifierstyle={\scriptsize},%
 basicstyle={\small},%
 identifierstyle={\small},%
% commentstyle={\small\itshape},%
%commentstyle={\scriptsize},%
commentstyle={\small},%
 keywordstyle={\small\bfseries},%
 ndkeywordstyle={\small},%
stringstyle={\small\ttfamily},
 frame={tb},
 breaklines=true,
 columns=[l]{fullflexible},%
 numbers=left,%
 xrightmargin=1zw,%
 xleftmargin=1.5zw,%
 numberstyle={\small},%
 stepnumber=1,
 numbersep=1zw,%
% lineskip=-0.1ex%
}

\def\XMP{XcalableMP}
\def\OMP{OpenMP}
\def\HPF{HPF}
\def\CAF{Co-array Fortran}
\def\MPI{MPI}
\def\Fort{Fortran}
\def\C{C}
\def\XMPF{XcalableMP Fortran}
\def\XMPC{XcalableMP C}

\def\openb{{\it [}}
\def\closeb{{\it ]}}
\def\bsquare{\rule[-2pt]{5pt}{10pt}}

\def\|{\verb|}

%\newenvironment{myfigure}{\begin{figure}[ht]\begin{center}}{\end{center}\end{figure}}

\def\Directive#1{{\tt #1}\index{#1@{\tt #1}}\index{Directive!#1@{\tt #1}}}

%\def\Syntax#1{\index{{\tt #1}}\index{Syntax!{\tt #1}}}
\def\Syntax#1{\index{Syntax!#1@{\tt #1}}}

\def\Term#1{{#1}\index{#1}}

%\def\Example#1{\index{#1}\index{Example!{\tt #1}}}
\def\Example#1{\index{Example!#1@{\tt #1}}}

\def\Intrinsic#1{\index{#1@{\tt #1}}\index{Intrinsic and Library Procedures!#1@{\tt #1}}}

\DefineVerbatimEnvironment{Fexample}{Verbatim}{numbers=left,numbersep=3pt,stepnumber=5,%
frame=single,label=\Fort}
\DefineVerbatimEnvironment{FexampleR}{Verbatim}{numbers=right,numbersep=3pt,stepnumber=5,%
frame=single,label=\Fort}

\DefineVerbatimEnvironment{Cexample}{Verbatim}{numbers=left,numbersep=3pt,stepnumber=5,%
frame=single,label=\C}
\DefineVerbatimEnvironment{CexampleR}{Verbatim}{numbers=right,numbersep=3pt,stepnumber=5,%
frame=single,label=\C}

\DefineVerbatimEnvironment{XFexample}{Verbatim}{numbers=left,numbersep=3pt,stepnumber=5,%
frame=single,label=\XMPF}
\DefineVerbatimEnvironment{XFexampleR}{Verbatim}{numbers=right,numbersep=3pt,stepnumber=5,%
frame=single,label=\XMPF}

\DefineVerbatimEnvironment{XCexample}{Verbatim}{numbers=left,numbersep=3pt,stepnumber=5,%
frame=single,label=\XMPC}
\DefineVerbatimEnvironment{XCexampleR}{Verbatim}{numbers=right,numbersep=3pt,stepnumber=5,%
frame=single,label=\XMPC}

\DefineVerbatimEnvironment{MPICexample}{Verbatim}{numbers=right,numbersep=3pt,stepnumber=5,%
frame=single,label=MPI C}
\DefineVerbatimEnvironment{MPIFexample}{Verbatim}{numbers=right,numbersep=3pt,stepnumber=5,%
frame=single,label=MPI Fortran}


\setcounter{secnumdepth}{4}
\setcounter{tocdepth}{3}
\setcounter{totalnumber}{6}
\usepackage{fancyhdr}

\let\olditemize\itemize
\renewcommand{\itemize}{
   \olditemize
   \setlength{\itemsep}{8pt}
   \setlength{\parskip}{0pt}
   \setlength{\parsep}{0pt}
}

% \parindent = 0pt
% \hoffset=0cm
% \oddsidemargin=0cm
% \evensidemargin=0cm
% \textwidth=16cm
% \topmargin=-1cm
% \voffset=0cm
% \textheight=24cm

\def\progenv{\baselineskip=10pt\tt\progspecial{`}\parindent=0.3cm}
\def\shellenv{\baselineskip=10pt\tt\progspecial{`}\parindent=0.3cm\nolineno}

\renewcommand{\topfraction}{.99}
\renewcommand{\bottomfraction}{.99}

\def\openb{{\it [}}
\def\closeb{{\it ]}}
\def\XMP{XMP}
\def\XACC{XACC}
\def\OACC{OpenACC}
\def\OMP{OpenMP}
\def\XMPF{XcalableMP Fortran}
\def\XMPC{XcalableMP C}
\def\XACCF{XcalableACC Fortran}
\def\XACCC{XcalableACC C}
\def\Syntax#1{\index{Syntax!#1@{\tt #1}}}
\def\Example#1{\index{Example!#1@{\tt #1}}}
%
\def\phrule{\vspace{0.2cm}\hrule\vspace{0.05cm}\hrule}
\def\qhrule{\vspace{0.2cm}\hrule}
\def\dhrule{\hrule\vspace{0.05cm}\hrule}
\def\bsquare{\rule[-2pt]{5pt}{10pt}}
%
\newenvironment{mytable}[3]{\begin{table}[ht]\caption{#1}\label{#2}\vspace*{-0.3cm}\begin{center}\begin{tabular}{#3}}{\end{tabular}\end{center}\end{table}}
\newenvironment{myfigure}{\begin{figure}[ht]\begin{center}}{\end{center}\end{figure}}
%
\DefineVerbatimEnvironment{XACCFexampleL}{Verbatim}{numbers=left,numbersep=3pt,stepnumber=5,frame=single,label=\XACCF}
\DefineVerbatimEnvironment{XACCCexampleR}{Verbatim}{numbers=right,numbersep=3pt,stepnumber=5,frame=single,label=\XACCC}
\DefineVerbatimEnvironment{XACCCexampleL}{Verbatim}{numbers=left,numbersep=3pt,stepnumber=5,frame=single,label=\XACCC}


%%%%%%%%%%%%%%%%%%%%%%%%%%%%%%%%%%%%%%%%%%%%%%%%%%%%%%%%%%%%%%%%%%%%%%%%%%%%%%%%%%%%%%%%%
\begin{document}



\title*{XcalableACC: an Integration of XcalableMP and OpenACC}
% Use \titlerunning{Short Title} for an abbreviated version of
% your contribution title if the original one is too long

\author{Akihiro Tabuchi and H. Murai, M. Nakao, T. Odajima, and T. Boku}
% Use \authorrunning{Short Title} for an abbreviated version of
% your contribution title if the original one is too long

\institute{
Akihiro Tabuchi \at Fujitsu Laboratories Ltd., 4-1-1 Kamikodanaka,
Nakahara-ku, Kawasaki, Kanagawa 211-8588, Japan,
\email{tabuchi.akihiro@fujitsu.com}
\and
Hitoshi Murai \at RIKEN Center for Computational Science, 7-1-26
Minatojima-minami-machi, Chuo-ku, Kobe, Hyogo 650-0047, Japan,
\email{h-murai@riken.jp}
\and
Masahiro Nakao \at RIKEN Center for Computational Science,
7-1-26 Minatojima-minami-machi, Chuo-ku, Kobe, Hyogo 650-0047, Japan,
\email{masahiro.nakao@riken.jp}
\and
Tetsuya Odajima \at RIKEN Center for Computational Science, 7-1-26
Minatojima-minami-machi, Chuo-ku, Kobe, Hyogo 650-0047, Japan,
\email{tetsuya.odajima@riken.jp}
\and
Taisuke Boku \at Center for Computationl Sciences, University of
Tsukuba, 1-1-1 Tennodai, Tsukuba, Ibaraki 305-8577, Japan,
\email{taisuke@ccs.tsukuba.ac.jp}
}

%
% Use the package "url.sty" to avoid
% problems with special characters
% used in your e-mail or web address
%
\maketitle

\abstract{XcalableACC (XACC) is an extension of XcalableMP for
accelerated clusters. It is defined as a diagonal integration of
XcalableMP and OpenACC, which is another directive-based language
designed to program heterogeneous CPU/accelerator systems. XACC has
features for handling distributed-memory parallelism, inherited from
XMP, offloading tasks to accelerators, inherited from OpenACC, and
two additional functions: data/work mapping among multiple accelerators
and direct communication between accelerators.
}

%\tableofcontents

%
% Sections
%

\section{Introduction}\label{chap:intro}

\pagenumbering{arabic}
\setcounter{page}{1}

XcalableMP (XMP)~\cite{xmp} has complementary programming models of
global-view and local-view. The former is a directive-base language 
extension to the base language Fortran and C, and the latter adopts 
the coarray features defined in Fortran 2008~\cite{coarray} and 
a part of the ones in Fortran 2018~\cite{coarray18}. 
%
The purpose of the coarray features as the local-view part of XMP is 
1) writing the application programs that is difficult for the global-view programming
and 2) writing such important parts of the program that is critical for the performance
with easier programming model than the MPI message passing.
Therefore, the coarray features in XMP must be naturally merged into the 
global-view XMP language and must perform with high performance comparable to MPI.

The Omni XMP compiler is an open-source implementation developed at RIKEN 
and the University of Tsukuba~\cite{omni}. 
Its kernel is a source-to-source compiler that converts an XMP program 
into a Fortran program by calling a runtime library.
%
The coarray translator has been implemented into the Omni XMP compiler.
Since the images is mapped one-to-one to XMP nodes, 
each image was implemented as a process, and 
the definition and reference to coarrays were implemented as the 
inter-node one-sided communications.

This chapter describes the techniques in the coarray compiler and 
the runtime library with some evaluation compared with the MPI message passing.
In the rest of this chapter, 
Section 2 introduces the requirements from the coarray features,
Section 3 describes the implementation to solve the requirements, and
Section 4 evaluates the performance and the productivity of coarray programs.
After related work is shown in Section 5, Section 6 concludes this chapter.


 \cleardoublepage
\section{XcalableACC Language}

{\XACC} is roughly defined as a diagonal integration of XMP
and OpenACC with some additional XACC extensions, where XMP directives
are for specifying distributed-memory parallelism, OpenACC for
offloading, and the extensions for other XACC-specific features.

The syntax and semantics of XMP and OpenACC directives appearing in XACC
codes follow those in XMP and OpenACC, respectively, unless specified
below.

\subsection{Data Mapping}

% This chapter defines a behavior of mixing {\XMP} and {\OACC}.
% Note that the existing {\OACC} is not extended in the {\XMP} extensions.
% The {\XMP} extensions can represent 
% (1) parallelization with keeping sequential code image using a
% combination of {\XMP} and {\OACC}, 
% and
% (2) communication among accelerator memories and between accelerator
% memory and host memory on different {\bf nodes}
% using {\XACC} directives or {\bf coarray} features.

% \subsection{Combination of {\XMP} and {\OACC}}

% \subsubsection{{\OACC} Directives on Data}

% \subsubsection*{Description}

% When {\bf distributed arrays} appear in {\OACC} constructs,
% global indices in {\bf distributed arrays} are used.
{\bf Global arrays} distributed with {\XMP} directives can be
globaly-indexed in {\OACC} constructs.
%
{\bf Global arrays} may appear in the {\tt update}, {\tt enter
data}, {\tt exit data}, {\tt host\_data}, {\tt cache}, and {\tt declare}
directives;
%
and the data clauses such as {\tt deviceptr}, {\tt present}, {\tt copy},
{\tt copyin}, {\tt copyout}, {\tt create}, and {\tt delete}.
%
% Data transfer of {\bf distributed array} by {\OACC} is performed on only
% {\bf nodes} which have elements specified by the global indices.
When data transfer of a {\bf global array} between host and accelerator
memory is specified by a {\OACC} directive, it is performed locally for
the local section of the array within each node.

\subsubsection*{Example}

In lines 2--6 of Fig.~\ref{code:ex-oacc-data},
the directives declare {\bf global arrays} \|a| and \|b|.
%
In line 8,
the {\tt enter data} directive transfers a section of \|a| from host
memory to accelerator memory. Note that \|a| is globally-indexed.
%
In line 9,
the {\tt data} directive transfers the whole of \|b| from host memory to 
accelerator memory.

\begin{myfigure}
\begin{minipage}{0.45\hsize}
\begin{center}
\begin{XACCFexampleL}
integer :: a(N), b(N)
!$xmp template t(N)
!$xmp nodes p(*)
!$xmp distribute t(block) onto p
!$xmp align a(i) with t(i)
!$xmp align b(i) with t(i)
...
!$acc enter data copyin(a(1:K))
!$acc data copy(b)
...
\end{XACCFexampleL}
\end{center}
\end{minipage}
%
\begin{minipage}{0.53\hsize}
\begin{center}
\begin{XACCCexampleR}
int a[N], b[N];
#pragma xmp template t[N]
#pragma xmp nodes p[*]
#pragma xmp distribute t[block] onto p
#pragma xmp align a[i] with t[i]
#pragma xmp align b[i] with t[i]
...
#pragma acc enter data copyin(a[0:K])
#pragma acc data copy(b)
{ ...
\end{XACCCexampleR}
\end{center}
\end{minipage}
\caption{{\XACC} code with {\tt enter\_data}
  directive.}\label{code:ex-oacc-data}
\end{myfigure}


\subsection{Work Mapping}

%\subsubsection*{Description}

In order to parallelize a loop statement among nodes and on accelerators,
{\XMP} {\tt loop} directive and {\OACC} {\tt loop} directive are used.
%
While an {\XMP} {\tt loop} directive parallelizes a loop statement among
nodes, {\OACC} {\tt loop} directive further parallelizes
the loop statement on accelerators within each node.
%
For ease of writing, the nesting order of {\XMP} {\tt loop} directive and
{\OACC} {\tt loop} directive does not matter.

When {\tt acc} clause appears in {\XMP} loop directive with {\tt reduction} clause,
the directive performs a reduction operation for a variable specified in
the {\tt reduction} clause on accelerator memory.

\subsubsection*{Restriction}

\begin{itemize}
\item In an {\OACC} {\bf compute region}, no \XMP directives except for
	  {\tt loop} directive with no {\tt reduction} clause is allowed.
\item In an {\OACC} {\bf compute region}, the parameter (i.e., the lower
	  bound, upper bound, and step) of the target loop must
	  remain unchanged.
\item An {\tt acc} clause can be specified in an {\XMP} loop directive 
	  only when a {\tt reduction} clause is also specified.
\end{itemize}

\subsubsection*{Example 1}

\begin{myfigure}
\begin{minipage}{0.45\hsize}
\begin{center}
\begin{XACCFexampleL}
integer :: a(N), b(N)
!$xmp template t(N)
!$xmp nodes p(*)
!$xmp distribute t(block) onto p
!$xmp align a(i) with t(i)
!$xmp align b(i) with t(i)
...
!$acc parallel loop copy(a, b)
!$xmp loop on t(i)
do i=0, N
  b(i) = a(i)
end do
!$acc end parallel
\end{XACCFexampleL}
\end{center}
\end{minipage}
%
\begin{minipage}{0.53\hsize}
\begin{center}
\begin{XACCCexampleR}
int a[N], b[N];
#pragma xmp template t[N]
#pragma xmp nodes p[*]
#pragma xmp distribute t[block] onto p
#pragma xmp align a[i] with t[i]
#pragma xmp align b[i] with t[i]
...
#pragma acc parallel loop copy(a, b)
#pragma xmp loop on t[i]
for(int i=0;i<N;i++){
  b[i] = a[i];
}

\end{XACCCexampleR}
\end{center}
\end{minipage}
\caption{\XACC code with {\OACC} {\tt loop} construct.}\label{code:ex-oacc-loop}
\end{myfigure}

In lines 2--6 of Fig.~\ref{code:ex-oacc-loop},
the directives declare {\bf global arrays} \|a| and \|b|.
%
In line 8, the {\tt copy} clause on the {\tt parallel} directive
transfers \|a| and \|b| from host memory to accelerator memory.
%
In lines 8--9,
the {\tt parallel} directive and {\XMP} {\tt loop} directive
parallelize the following loop on an accelerator within
a node and among nodes, respectively.

\subsubsection*{Example 2}

\begin{myfigure}
\begin{center}
\begin{XACCFexampleL}
integer :: a(N), sum = 10
!$xmp template t(N)
!$xmp nodes p(*)
!$xmp distribute t(block) onto p
!$xmp align a(i) with t(i)
...
!$acc parallel loop copy(a, sum) reduction(+:sum)
!$xmp loop on t(i) reduction(+:sum) acc
do i=0, N
  sum = sum + a(i)
end do
!$acc end parallel loop
\end{XACCFexampleL}
\begin{XACCCexampleL}
int a[N], sum = 10;
#pragma xmp template t[N]
#pragma xmp nodes p[*]
#pragma xmp distribute t[block] onto p
#pragma xmp align a[i] with t[i]
...
#pragma acc parallel loop copy(a, sum) reduction(+:sum)
#pragma xmp loop on t[i] reduction(+:sum) acc
for(int i=0;i<N;i++){
  sum += a[i];
}
\end{XACCCexampleL}
\end{center}
\caption{{\XACC} code with {\OACC} {\tt loop} construct
  with {\tt reduction} clause.}\label{code:ex-oacc-loop-reduction}
\end{myfigure}

In lines 2--5 of Fig.~\ref{code:ex-oacc-loop-reduction},
the directives declare a {\bf global array} \|a|.
%
In line 7,
the the {\tt copy} clause on the {\tt parallel} directive transfers \|a|
and a variable \|sum| from host memory to accelerator memory.
%
In lines 7--8,
the {\tt parallel} directive and {\XMP} {\tt loop} directive parallelize
the following loop on an accelerator within a {\bf node} and in among
nodes, respectively.
%
After finishing the calculation of the loop,
the OpenACC {\tt reduction} clause and the {\XMP} {\tt reduction} clause
with {\tt acc} in lines 7--8 perform a reduction operation for \|sum|
first on the accelerator within a node and then among all {\bf nodes}.



%%%%%%%%%%%%%%%%%%%%
\subsection{Data Communication and Synchronization}

%\subsubsection{Overview}

% Moreover,
% {\tt reflect\_init} and {\tt reflect\_do} directives are added as
% extensions of the {\tt reflect} directive.

% {\XACC} directives are directives which are added an {\tt acc} clause to
% the above directives.

When an \|acc| clause is specified in an XMP's communication
and synchronization directive, the directive works for the data on 
accelerator memory to transfer it.

The \|acc| clause can be specified on the following XMP's communication
and synchronization directives:

\begin{itemize}
  \item \|reflect|
%  \item \|reflect_init| and \|reflect_do|
  \item \|gmove|
  \item \|barrier|
  \item \|reduction|
  \item \|bcast|
  \item \|wait_async|
\end{itemize}

Note that while a \|gmove| directive with \|acc| 
and {\bf coarray} features can perform communication both between
accelerators and between 
accelerator and host memory that may be on different {\bf nodes},
%
other directives with \|acc| can perform communication only between
accelerators.

% This section describes only the extended parts of {\XACC} directives
% from {\XMP} directives. 
% For other information, refer to the {\XMP} specification\cite{xmp}.





% \subsubsection{reflect Construct}\label{sec:reflect}
% \subsubsection*{Synopsis}
% The {\tt reflect} construct assigns the value of a
% reflection source to the corresponding shadow object.

% \subsubsection*{Syntax}
% \begin{tabular}{ll}
%  \verb![F]! & \verb|!$xmp| {\tt reflect} \verb|(| {\it array-name}
%  {\openb}, {\it array-name}{\closeb}... \verb|)| {\bsquare} \\
%  &\hspace{0.1cm} {\bsquare} {\openb}{\tt width (} {\it reflect-width}
%      {\openb}, {\it reflect-width}{\closeb}... {\tt )}{\closeb}
%      {\openb}{\tt orthogonal}{\closeb}
%      {\openb}{\tt async (} {\it async-id} {\tt )}{\closeb} {\openb}{\tt acc}{\closeb}\\
% \verb![C]! & \verb|#pragma xmp| {\tt reflect} \verb|(| {\it array-name}
%      {\openb}, {\it array-name}{\closeb}... \verb|)| {\bsquare} \\
%  &\hspace{0.1cm} {\bsquare} {\openb}{\tt width (} {\it reflect-width}
%      {\openb}, {\it reflect-width}{\closeb}... {\tt )}{\closeb}
%      {\openb}{\tt orthogonal}{\closeb}
%      {\openb}{\tt async (} {\it async-id} {\tt )}{\closeb} {\openb}{\tt acc}{\closeb}\\
% \end{tabular}

% \vspace{1em}
% where {\it reflect-width} must be one of:
% \vspace{1em}

% \begin{tabular}{ll}
%  \hspace{0.5cm} & {\openb}{\tt /periodic/}{\closeb} {\it int-expr} \\
%                 & {\openb}{\tt /periodic/}{\closeb} {\it int-expr} : {\it int-expr}
% \end{tabular}

% \subsubsection*{Description}
% When the {\tt acc} clause is specified,
% the {\tt reflect} construct updates each of the shadow object of the
% array specified by {\it array-name} on accelerator memory with the value of its corresponding
% reflection source.

% \subsubsection*{Restriction}
% \begin{itemize}
%  \item When the {\tt acc} clause is specified,
%    the arrays specified by the sequence of {\it array-name}'s must be allocated on accelerator memory.
%  \item This construct must not appear in {\OACC} {\bf compute region}.
% \end{itemize}

\subsubsection*{Example}

In lines 2--5 of Fig. \ref{code:reflect},
the directives declare a {\bf global array} \|a|.
In line 6, 
the {\tt shadow} directive allocates the shadow areas of the \|a|.
In line 8,
the {\tt enter data} directive transfers \|a| with the shadow areas from
host memory to accelerator memory.
In line 9,
the {\tt reflect} directive updates the shadow areas of the {\bf
distributed array} {\it a} on accelerator memory on all nodes.

\begin{myfigure}
\begin{minipage}{0.45\hsize}
\begin{center}
\begin{XACCFexampleL}
integer :: a(N)
!$xmp template t(N)
!$xmp nodes p(*)
!$xmp distribute t(block) onto p
!$xmp align a(i) with t(i)
!$xmp shadow a(1)
...
!$acc enter data copyin(a)
!$xmp reflect (a) acc
\end{XACCFexampleL}
\end{center}
\end{minipage}
%
\begin{minipage}{0.53\hsize}
\begin{center}
\begin{XACCCexampleR}
int a[N];
#pragma xmp template t[N]
#pragma xmp nodes p[*]
#pragma xmp distribute t[block] onto p
#pragma xmp align a[i] with t[i]
#pragma xmp shadow a[1]
...
#pragma acc enter data copyin(a)
#pragma xmp reflect (a) acc
\end{XACCCexampleR}
\end{center}
\end{minipage}
\caption{Code example in {\tt reflect} construct}\label{code:reflect}
\end{myfigure}

% \subsubsection{{\tt reflect\_init} and {\tt reflect\_do} Constructs}\label{sec:gmove}

% \subsubsection*{Synopsis}
% Since the {\tt reflect\_init} construct performs the initialization
% processes of the {\tt reflect} construct,
% the {\tt reflect\_do} construct performs communication of the {\tt reflect} construct.

% \subsubsection*{Syntax}
% \begin{tabular}{ll}
%  \verb![F]! & \verb|!$xmp| {\tt reflect\_init} \verb|(| {\it array-name}
%  {\openb}, {\it array-name}{\closeb}... \verb|)| {\bsquare} \\
%  &\hspace{0.1cm} {\bsquare} {\openb}{\tt width (} {\it reflect-width}
%      {\openb}, {\it reflect-width}{\closeb}... {\tt )}{\closeb}
%      {\openb}{\tt orthogonal}{\closeb}
%      {\openb}{\tt async (} {\it async-id} {\tt )}{\closeb} {\openb}{\tt acc}{\closeb}\\
% \verb![C]! & \verb|#pragma xmp| {\tt reflect\_init} \verb|(| {\it array-name}
%      {\openb}, {\it array-name}{\closeb}... \verb|)| {\bsquare} \\
%  &\hspace{0.1cm} {\bsquare} {\openb}{\tt width (} {\it reflect-width}
%      {\openb}, {\it reflect-width}{\closeb}... {\tt )}{\closeb}
%      {\openb}{\tt orthogonal}{\closeb}
%      {\openb}{\tt async (} {\it async-id} {\tt )}{\closeb} {\openb}{\tt acc}{\closeb}\\
% \end{tabular}

% \vspace{1em}
% where {\it reflect-width} must be one of:
% \vspace{1em}

% \begin{tabular}{ll}
%  \hspace{0.5cm} & {\openb}{\tt /periodic/}{\closeb} {\it int-expr} \\
%                 & {\openb}{\tt /periodic/}{\closeb} {\it int-expr} : {\it int-expr}
% \end{tabular}

% \vspace{1em}

% \begin{tabular}{ll}
%  \verb![F]! & \verb|!$xmp| {\tt reflect\_do} \verb|(| {\it array-name}
%  {\openb}, {\it array-name}{\closeb}... \verb|)| {\openb}{\tt async (} {\it async-id} {\tt )}{\closeb} {\openb}{\tt acc}{\closeb}\\
% \verb![C]! & \verb|#pragma xmp| {\tt reflect\_do} \verb|(| {\it array-name}
%      {\openb}, {\it array-name}{\closeb}... \verb|)| 
%      {\openb}{\tt async (} {\it async-id} {\tt )}{\closeb} {\openb}{\tt acc}{\closeb}\\
% \end{tabular}

% \subsubsection*{Description}
% The {\tt reflect} construct is divided into {\tt reflect\_init} and {\tt
% reflect\_do} constructs to improve performance like the MPI persistent
% communication\cite{mpi}.

% As a typical example, 
% if a {\tt reflect} construct is called repeatedly with the same
% condition in a loop statement,
% inserting a {\tt reflect\_init} construct before the loop statement 
% and replacing the {\tt reflect} construct with a {\tt reflect\_do}
% construct will improve its performance
% because unneeded initialization processes are removed.

% \subsubsection*{Restriction}
% \begin{itemize}
%  \item When the {\tt acc} clause is specified,
%    the arrays specified by the sequence of {\it array-name}'s must be
% 	   allocated on accelerator memory.
%  \item These constructs must not appear in {\OACC} {\bf compute region}.
%  \item The {\tt reflect\_init} directive must execute before the {\tt
% 	   reflect\_init} directive executes.
% \end{itemize}

% \subsubsection*{Example}
% \begin{myfigure}
% \begin{minipage}{0.45\hsize}
% \begin{center}
% \begin{XACCFexampleL}
% integer :: a(N)
% !$xmp template t(N)
% !$xmp nodes p(*)
% !$xmp distribute t(block) onto p
% !$xmp align a(i) with t(i)
% !$xmp shadow a(1)
% ...
% !$acc enter data copyin(a)
% !$xmp reflect_init (a) acc
% ...
% !$xmp reflect_do (a) acc
% \end{XACCFexampleL}
% \end{center}
% \end{minipage}
% %
% \begin{minipage}{0.53\hsize}
% \begin{center}
% \begin{XACCCexampleR}
% int a[N];
% #pragma xmp template t[N]
% #pragma xmp nodes p[*]
% #pragma xmp distribute t[block] onto p
% #pragma xmp align a[i] with t[i]
% #pragma xmp shadow a[1]
% ...
% #pragma acc enter data copyin(a)
% #pragma xmp reflect_init (a) acc
% ...
% #pragma xmp reflect_do (a) acc
% \end{XACCCexampleR}
% \end{center}
% \end{minipage}
% \caption{Code example in {\tt reflect\_init} and {\tt reflect\_do}
%   constructs}\label{code:reflect_initdo}
% \end{myfigure}

% In lines 2-5 of Fig. \ref{code:reflect_initdo},
% the directives declare {\bf distributed array} {\it a}.
% In line 6,
% the {\tt shadow} directive allocates shadow areas of the {\bf
% distributed array} {\it a}.
% In line 8,
% the {\tt enter data} directive transfers the {\bf distributed array}
% {\it a} with the shadow areas from host memory to accelerator memory.
% In line 9,
% the {\tt reflect\_init} directive performs initialization processes for
% the {\tt reflect\_do} construct which targets the {\bf distributed
% array} {\it a}.
% In line 11,
% the {\tt reflect\_do} directive updates the shadow areas of the {\bf
% distributed array} {\it a} on accelerator memory between neighboring
% {\bf nodes}
% without its initialization processes.

% \subsubsection{gmove Construct}\label{sec:gmove}
% \subsubsection*{Synopsis}
% The {\tt gmove} construct allows an assignment statement,
% which may cause communication, to be executed possibly in parallel by
% the executing {\bf nodes}.

% \subsubsection*{Syntax}
% \begin{tabular}{ll}
% \verb![F]! & \verb|!$xmp| {\tt gmove} {\openb}{\tt in} $\vert$ {\tt
%  out}{\closeb} {\openb}{\tt async (} {\it async-id} {\tt )}{\closeb} {\openb}{\tt acc}{\openb}({\it variable}){\closeb}{\closeb}\\
% \verb![C]! & \verb|#pragma xmp| {\tt gmove} {\openb}{\tt in} $\vert$ {\tt
%  out}{\closeb} {\openb}{\tt async (} {\it async-id} {\tt )}{\closeb} {\openb}{\tt acc}{\openb}({\it variable}){\closeb}{\closeb}\\
% \end{tabular}

% \subsubsection*{Description}
% \begin{itemize}
%  \item When the {\tt acc} clause is specified and the variable is not specified by {\it variable} in the parenthesis,
% variables of both sides in the assignment statement on accelerator memory are targeted.
%  \item When the {\tt acc} clause is specified and the variable is specified by {\it variable} in the parenthesis,
% the specified variable on accelerator memory is targeted, 
% and the unspecified variable on host memory is targeted.
% \end{itemize}

% \subsubsection*{Restriction}
% \begin{itemize}
%  \item The variables targeted on accelerator memory must be allocated on accelerator memory.
%  \item This construct must not appear in {\OACC} {\bf compute region}.
% \end{itemize}

% \subsubsection*{Example}
% \begin{myfigure}
% \begin{minipage}{0.45\hsize}
% \begin{center}
% \begin{XACCFexampleL}
% integer :: a(N), b(N)
% !$xmp template t(N)
% !$xmp nodes p(*)
% !$xmp distribute t(block) onto p
% !$xmp align a(i) with t(i)
% !$xmp align b(i) with t(i)
% ...
% !$acc enter data copyin(a, b)
% !$xmp gmove acc
%   a(:) = b(:)

% !$xmp gmove acc(b)
%   a(:) = b(:)
% \end{XACCFexampleL}
% \end{center}
% \end{minipage}
% %
% \begin{minipage}{0.53\hsize}
% \begin{center}
% \begin{XACCCexampleR}
% int a[N], b[N];
% #pragma xmp template t[N]
% #pragma xmp nodes p[*]
% #pragma xmp distribute t[block] onto p
% #pragma xmp align a[i] with t[i]
% #pragma xmp align b[i] with t[i]
% ...
% #pragma acc enter data copyin(a, b)
% #pragma xmp gmove acc
%   a[:] = b[:];

% #pragma xmp gmove acc(b)
%   a[:] = b[:];
% \end{XACCCexampleR}
% \end{center}
% \end{minipage}
% \caption{Code example in {\tt gmove} construct}\label{code:gmove}
% \end{myfigure}

% In lines 2-6 of Fig. \ref{code:gmove},
% the directives declare {\bf distributed arrays} {\it a} and {\it b}.
% In line 8,
% the {\tt enter data} directive transfers the {\bf distributed arrays} {\it a} and {\it b} from host memory to accelerator memory.
% In lines 9-10,
% the {\tt gmove} construct copies the whole {\bf distributed array} {\it b} to
% that of the {\bf distributed array} {\it a} on accelerator memories.
% In lines 12-13,
% the {\tt gmove} construct copies the whole {\bf distributed array} {\it b} on accelerator memory to
% that of the {\bf distributed array} {\it a} on host memory.

% \subsubsection{barrier Construct}\label{sec:barrier}
% \subsubsection*{Synopsis}
% The {\tt barrier} construct specifies an explicit barrier
% at the point at which the construct appears.

% \subsubsection*{Syntax}
% \begin{tabular}{ll}
% \verb![F]! & \verb|!$xmp| {\tt barrier} {\openb}{\tt on} {\it nodes-ref}
%  $\vert${\it template-ref}{\closeb} {\openb}{\tt acc}{\closeb}\\
% \verb![C]! & \verb|#pragma xmp| {\tt barrier} {\openb}{\tt on} {\it
%      nodes-ref} $\vert$ {\it template-ref}{\closeb} {\openb}{\tt acc}{\closeb}\\
% \end{tabular}

% \subsubsection*{Description}
% \begin{itemize}
%  \item When the {\tt acc} clause is specified,
% the barrier construct blocks until all ongoing asynchronous operations on accelerators are completed.
%  \item When the {\tt acc} clause is not specified,
% the barrier construct does not guarantee that an ongoing asynchronous operation on accelerator is completed.
% \end{itemize}

% \subsubsection*{Example}
% \begin{myfigure}
% \begin{minipage}{0.45\hsize}
% \begin{center}
% \begin{XACCFexampleL}
% !$xmp nodes p(*)
% ...
% !$xmp barrier acc
% \end{XACCFexampleL}
% \end{center}
% \end{minipage}
% %
% \begin{minipage}{0.53\hsize}
% \begin{center}
% \begin{XACCCexampleR}
% #pragma xmp nodes p[*]
% ...
% #pragma xmp barrier acc
% \end{XACCCexampleR}
% \end{center}
% \end{minipage}
% \caption{Code example in {\tt barrier} construct}\label{code:barrier}
% \end{myfigure}

% In line 1,
% the {\tt nodes} directive defines {\tt node set} {\it p}.
% In line 3,
% the {\tt barrier} directive performs a barrier operation for accelerators on all {\bf node}.

% \subsubsection{reduction Construct}\label{sec:reduction}
% \subsubsection*{Synopsis}
% The {\tt reduction} construct performs a reduction operation among {\bf nodes}.

% \subsubsection*{Syntax}
% \Syntax{reduction}

% \begin{tabular}{ll}
% \verb![F]! & \verb|!$xmp| {\tt reduction (} {\it reduction-kind} {\it
%   :} {\it variable} {\openb}, {\it variable} {\closeb}... {\tt )}
%  {\bsquare} \\
%  & \hspace{5cm} {\bsquare} {\openb}{\tt on} {\it node-ref} $\vert$ {\it
%      template-ref}{\closeb} {\openb}{\tt async (} {\it async-id} {\tt )}{\closeb} {\openb}{\tt acc}{\closeb}\\
% \end{tabular}

% \vspace{0.5cm}
% where {\it reduction-kind} is one of:

% \begin{tabular}{ll}
%  \hspace{0.5cm} & {\tt +} \\
%  & {\tt *} \\
% % & {\tt -} \\
%  & {\tt .and.} \\
%  & {\tt .or.} \\
%  & {\tt .eqv.} \\
%  & {\tt .neqv.} \\
%  & {\tt max} \\
%  & {\tt min} \\
%  & {\tt iand} \\
%  & {\tt ior} \\
%  & {\tt ieor} \\
% \end{tabular}

% \vspace{0.5cm}

% \begin{tabular}{ll}
%  \hspace{-\parindent}
%  \verb![C]! & \verb|#pragma xmp| {\tt reduction (} {\it reduction-kind} {\it
%   :} {\it variable} {\openb}, {\it variable} {\closeb}... {\tt )}
%  {\bsquare} \\
%  & \hspace{5cm} {\bsquare} {\openb}{\tt on} {\it node-ref} $\vert$ {\it
%      template-ref}{\closeb} {\openb}{\tt async (} {\it async-id} {\tt )}{\closeb} {\openb}{\tt acc}{\closeb} \\
% \end{tabular}
% \vspace{0.5cm}

% where {\it reduction-kind} is one of:

% \begin{tabular}{ll}
%  \hspace{0.5cm} & {\tt +} \\
%  & {\tt *} \\
% % & {\tt -} \\
%  & {\verb|&|} \\
%  & {\tt |} \\
%  & {\verb|^|} \\
%  & {\verb|&&|} \\
%  & {\tt ||} \\
%  & {\tt max} \\
%  & {\tt min} \\
% \end{tabular}

% \subsubsection*{Description}
% When the {\tt acc} clause is specified,
% the {\tt reduction} construct performs a type of
% reduction operation specified by {\it reduction-kind} for the specified
% local variables among the accelerators and 
% sets the reduction results to the variables on each of the accelerators.

% \subsubsection*{Restriction}
% \begin{itemize}
%  \item When the {\tt acc} clause is specified,
%    the variables specified by the sequence of {\it variable}'s must be allocated on accelerator memory.
%  \item This construct must not appear in {\OACC} {\bf compute region}.
% \end{itemize}

% \subsubsection*{Example}
% \begin{myfigure}
% \begin{minipage}{0.45\hsize}
% \begin{center}
% \begin{XACCFexampleL}
% integer :: a
% !$xmp nodes p(*)
% ...
% !$acc enter data copyin(a)
% !$xmp reduction(+:a) acc
% \end{XACCFexampleL}
% \end{center}
% \end{minipage}
% %
% \begin{minipage}{0.53\hsize}
% \begin{center}
% \begin{XACCCexampleR}
% int a;
% #pragma xmp nodes p[*]
% ...
% #pragma acc enter data copyin(a)
% #pragma xmp reduction(+:a) acc
% \end{XACCCexampleR}
% \end{center}
% \end{minipage}
% \caption{Code example in {\tt reduction} construct}\label{code:reduction}
% \end{myfigure}

% In line 2,
% the {\tt nodes} directive defines {\bf node set} {\it p}.
% In line 4,
% the {\tt enter data} directive transfers the local variable {\it a} from host memory to accelerator memory.
% In line 5,
% the {\tt reduction} directive calculates a total value of the variable {\it a} stored on each accelerator
% memory in each {\bf node}.

% \subsubsection{bcast Construct}\label{sec:bcast}
% \subsubsection*{Synopsis}
% The {\tt bcast} construct performs broadcast communication from a specified {\bf node}.

% \subsubsection*{Syntax}

% \begin{tabular}{ll}
%  \verb![F]! & \verb|!$xmp| {\tt bcast} \verb|(| {\it variable}
%  {\openb}, {\it variable}{\closeb}... \verb|)|
%  {\openb}{\tt from} {\it nodes-ref} $\vert$ {\it template-ref}{\closeb}
%  {\bsquare} \\
%  & \hspace{4.8cm} {\bsquare} {\openb}{\tt on} {\it nodes-ref}{\closeb}
%      $\vert$ {\it template-ref}{\closeb}
%      {\openb}{\tt async (} {\it async-id} {\tt )}{\closeb} {\openb}{\tt acc}{\closeb}\\

%  \verb![C]! & \verb|#pragma xmp| {\tt bcast} \verb|(| {\it variable}
%  {\openb}, {\it variable}{\closeb}... \verb|)|
%  {\openb}{\tt from} {\it nodes-ref}  $\vert$ {\it
%      template-ref}{\closeb} {\bsquare} \\
%  & \hspace{4.8cm} {\bsquare} {\openb}{\tt on} {\it nodes-ref} $\vert$ {\it
%      template-ref}{\closeb}
%  {\openb}{\tt async (} {\it async-id} {\tt )}{\closeb} {\openb}{\tt acc}{\closeb}\\
% \end{tabular}

% \subsubsection*{Description}
% When the {\tt acc} clause is specified, 
% the values of the variables specified by the sequence of {\it variable}'s on accelerator memory
% (called {\bf broadcast variables}) are broadcasted
% from the {\bf node} specified by the {\tt from} clause (called the
% {\bf source node}) to each of the {\bf nodes} in the {\bf node set} specified
% by the {\tt on} clause. After executing this construct,
% the values of the {\bf broadcast variables} become the same as those in the {\bf source node}.

% \subsubsection*{Restriction}
% \begin{itemize}
%  \item When the {\tt acc} clause is specified,
%    the variables specified by the sequence of {\it variable}'s must be allocated on accelerator memory.
%  \item This construct must not appear in {\OACC} {\bf compute region}.
% \end{itemize}

% \subsubsection*{Example}
% \begin{myfigure}
% \begin{minipage}{0.45\hsize}
% \begin{center}
% \begin{XACCFexampleL}
% integer :: a
% !$xmp nodes p(*)
% ...
% !$acc enter data copyin(a)
% !$xmp bcast(a) acc
% \end{XACCFexampleL}
% \end{center}
% \end{minipage}
% %
% \begin{minipage}{0.53\hsize}
% \begin{center}
% \begin{XACCCexampleR}
% int a;
% #pragma xmp nodes p[*]
% ...
% #pragma acc enter data copyin(a)
% #pragma xmp bcast(a) acc
% \end{XACCCexampleR}
% \end{center}
% \end{minipage}
% \caption{Code example in {\tt bcast} construct}\label{code:bcast}
% \end{myfigure}

% In line 2,
% the {\tt nodes} directive defines {\bf node set} {\it p}.
% In line 4,
% the {\tt enter data} directive transfers the local variable {\it a} from host memory to accelerator memory.
% In line 5,
% the {\tt bcast} directive broadcasts the variable {\it a} stored on accelerator memory to all {\it nodes}.

% \subsubsection{wait\_async Construct}\label{sec:waitasync}
% \subsubsection*{Synopsis}
% The {\tt wait\_async} construct guarantees asynchronous
% communications specified by {\it async-id} are complete.

% \subsubsection*{Syntax}
% \begin{tabular}{ll}
% \verb![F]! & \verb|!$xmp| {\tt wait\_async ( {\it async-id} {\openb},
%  {\it async-id} {\closeb}...)} {\openb}{\tt on} {\it nodes-ref} $\vert$
%  {\it template-ref}{\closeb} {\openb}{\tt acc}{\closeb}\\
% \verb![C]! & \verb|#pragma xmp| {\tt wait\_async ( {\it async-id} {\openb},
%  {\it async-id} {\closeb}...)} {\openb}{\tt on} {\it nodes-ref} $\vert$
%  {\it template-ref}{\closeb} {\bsquare} \\
% & \hspace{13.5cm} {\bsquare} {\openb}{\tt acc}{\closeb}\\
% \end{tabular}

% \subsubsection*{Description}
% When the {\tt acc} clause is specified,
% the {\tt wait\_async} construct blocks and therefore
% statements following it are not executed until all of the asynchronous
% communications that are specified by {\it async-id}'s and issued on the accelerators in
% {\bf node set} specified by the {\tt on} clause are complete.

% \subsubsection*{Restriction}
% This construct must not appear in {\OACC} {\bf compute region}.

% \subsubsection*{Example}
% \begin{myfigure}
% \begin{minipage}{0.45\hsize}
% \begin{center}
% \begin{XACCFexampleL}
% integer :: a
% !$xmp nodes p(*)
% ...
% !$acc enter data copyin(a)
% !$xmp reduction(+:a) acc async(1)
% ...
% !$xmp wait_async(1) acc
% \end{XACCFexampleL}
% \end{center}
% \end{minipage}
% %
% \begin{minipage}{0.53\hsize}
% \begin{center}
% \begin{XACCCexampleR}
% int a;
% #pragma xmp nodes p[*]
% ...
% #pragma acc enter data copyin(a)
% #pragma xmp reduction(+:a) acc async(1)
% ...
% #pragma xmp wait_async(1) acc
% \end{XACCCexampleR}
% \end{center}
% \end{minipage}
% \caption{Code example in {\tt wait\_async} construct}\label{code:waitasync}
% \end{myfigure}

% In line 2,
% the {\tt nodes} directive defines {\bf node set} {\it p}.
% In line 4,
% the {\tt enter data} directive transfers the local variable {\it a} from host memory to accelerator memory.
% In line 5,
% the {\tt reduction} directive performs asynchronously.
% In line 7,
% the {\tt wait\_async} construct blocks until the asynchronous reduction operation at line 5 is complete.


\subsection{Coarrays} \label{sec:coarray}

%\subsubsection*{Synopsis}
In {\XACC}, programmers can specify one-sided communication (i.e., put
and get operation) for data on accelerator memory using {\bf coarray}
features.
%
A combination of {\bf coarray} and the {\tt host\_data} construct
enables one-sided communication between accelerators.

%\subsubsection*{Description}
If {\bf coarrays} appear in a {\tt use\_device} clause of an enclosing
{\tt host\_data} construct, 
data on accelerator memory is selected as the target of the communication.
%
The synchronization for {\bf Coarray} operations on accelerators is
similar to that in {\XMP}.

\subsubsection*{Restriction}

\begin{itemize}
 \item {\bf Coarrays} on accelerator memory can be declared only with
	   the {\tt declare} directive.
	   % For example,
	   % {\tt enter data} and {\tt copy} directives cannot declare a {\bf coarray} on accelerator memory.
 \item No {\bf coarray} syntax is allowed in the {\OACC} {\bf compute
	   region}.
\end{itemize}

\subsubsection*{Example}

In line 3 of Fig.~\ref{code:coarray},
the {\tt declare} directive declares a {\bf coarray} \|a| and an
array \|b| on accelerator memory.
%
In lines 6--7,
{\bf node} 1 performs a put operation, where
the the whole of \|b| on the accelerator memory of {\bf node} 1 is
transferred to \|a| on the accelerator memory of
{\bf node} 2. 
%
In lines 9--10,
{\bf node} 1 performs a get operation, where
the whole of \|a| on the accelerator memory of {\bf node} 3 is
transferred to \|b| on the host memory of {\bf node} 1.
%
In line 13,
the {\tt sync all} statement in XACC/F or the {\tt xmp\_sync\_all}
function in XACC/C performs a barrier synchronizes among all {\bf nodes}
and guarantees the completion of ongoing coarray accesses.

\begin{myfigure}
\begin{minipage}{0.45\hsize}
\begin{center}
\begin{XACCFexampleL}
integer :: a(N)[*]
integer :: b(N)
!$acc declare create(a, b)
...
if(this_image() == 1) then
!$acc host_data use_device(a, b)
  a(:)[2] = b(:)

!$acc host_data use_device(a)
  b(:) = a(:)[3]
end if
...
sync all
\end{XACCFexampleL}
\end{center}
\end{minipage}
%
\begin{minipage}{0.53\hsize}
\begin{center}
\begin{XACCCexampleR}
int a[N]:[*];
int b[N];
#pragma acc declare create(a, b)
...
if(xmp_node_num() == 1){
#pragma acc host_data use_device(a, b)
  a[:]:[2] = b[:];

#pragma acc host_data use_device(a)
  b[:] = a[:]:[3];
}
...
xmp_sync_all(NULL);
\end{XACCCexampleR}
\end{center}
\end{minipage}
\caption{XACC code with coarray}\label{code:coarray}
\end{myfigure}



\subsection{Handling Multiple Accelerators}

{\XACC} also has a feature for handling multiple accelerators. This
section provides a brief overveiw of this feature. Please refer to
\cite{xacc} for more detail.

%%%%%%%%%%%%%%%%%%%%%%%%%%%%%%%%%%%%
\subsubsection{{\tt devices} Directive}

%\subsection*{Synopsis}
%The \|devices| directive declares a set of devices.

% \subsubsection*{Syntax}
% \begin{tabular}{ll}
%   \verb![F]! & \verb|!$acc| {\tt devices} {\it devices-decl} {\openb}, {\it devices-decl} {\closeb}...\\
%   \verb![C]! & \verb|#pragma acc| {\tt devices} {\it devices-decl} {\openb}, {\it devices-decl} {\closeb}...
% \end{tabular}

% \vspace{1em}
% where {\it device-decl} is one of:
% \vspace{1em}

% \begin{tabular}{lll}
%   \hspace{0.5cm} & & {\it devices-name} \verb|(| {\it devices-spec} \verb|)| \\
%   \hspace{0.5cm} & & {\it devices-name} \verb|(| {\it devices-spec} \verb|)| {\openb} {\tt =} {\it predefined-devices-ref} {\closeb} \\
%   %{\openb}, {\it devices-spec} {\closeb}...
%   \hspace{0.5cm} & \verb![C]! & {\it devices-name} \verb|[| {\it devices-spec} \verb|]| \\
%   \hspace{0.5cm} & \verb![C]! & {\it devices-name} \verb|[| {\it devices-spec} \verb|]| {\openb} {\tt =} {\it predefined-devices-ref} {\closeb}
%   %{\openb} \verb|[| {\it devices-spec} \verb|]|... {\closeb}
% \end{tabular}

% \vspace{1em}
% and {\it devices-spec} is one of:
% \vspace{1em}

% \begin{tabular}{ll}
%  \hspace{0.5cm} & * \\
%                 & {\it int-expr}
% \end{tabular}

% \vspace{1em}
% and {\it predefined-devices-ref} is one of:
% \vspace{1em}

% \begin{tabular}{ll}
%   \hspace{0.5cm} & {\it device-type-name} \verb|(| * $\vert$ {\it int-expr} $\vert$ {\it int-expr} : {\it int-expr}\verb|)| \\
%                  & {\it device-type-name} \verb|[| * $\vert$ {\it int-expr} $\vert$ {\it int-expr} : {\it int-expr}\verb|]|
% %                & {\it device-type-name} \verb|(| {\it int-expr} \verb|)|\\
% %                & {\it device-type-name} \verb|(| {\it int-expr} : {\it int-expr} \verb|)|
% \end{tabular}

% \subsubsection*{Description}

The \|devices| directive declares a {\bf device array} that corresponds
to a device set. This directive is analogous to the \|nodes| directive
for nodes in {\XMP}.

% The first and third forms are used to declare a device array that
% corresponds to a set of the entire default devices.
% The second and fourth forms are used to declare a device array, each
% device of which is
% assigned to a device of the device set is specified by {\it
% predefined-devices-ref} at the corresponding position.

%% In the first and second forms which use parentheses,
%% the corresponding position is Fortran's array element order, as if the device set were a one-dimensional device array.
%% In the third and fourth forms which use square brackets,
%% the corresponding position is C's array element order, as if the device set were a one-dimensional device array.

% \subsubsection*{Restriction}
% \begin{itemize}
% \item {\it devices-name} must not conflict with any other local name in
%       the same scoping unit.
%  %% \item When the {\bf acc} clause is specified,
%  %%   the arrays specified by the sequence of {\it array-name}'s must be allocated on accelerator memory.
%  \item This construct must not appear in {\OACC} {\bf compute region}.
% \end{itemize}

\subsubsection*{Example}

The following are examples of the devices declaration. The device array
\|d| corresponds to a set of entire default devices, and \|e| is a
subset of the predefined device array \|nvidia|. The program must be
executed by a node which is equipped with four or more NVIDIA
accelerator devices.
%
\begin{myfigure}
\begin{minipage}{0.45\hsize}
\begin{center}
\begin{XACCFexampleL}
!$acc devices d(*)
!$acc devices e(2) = nvidia(3:4)  
\end{XACCFexampleL}
\end{center}
\end{minipage}
%
\begin{minipage}{0.53\hsize}
\begin{center}
\begin{XACCCexampleR}
#pragma acc devices d[*]
#pragma acc devices e[2] = nvidia[2:2]
\end{XACCCexampleR}
\end{center}
\end{minipage}
\caption{XACC code with {\tt devices} directive.}\label{code:devices}
\end{myfigure}


% \subsubsection{Default Device Set}

% \subsection*{Synopsis}

% The default device set is the targeting device set when the {\tt
% on\_device} clause is omitted.

% \subsection*{Description}

% The default device set is the device set which contains the all {\OACC}
% default devices on the node.
% The device type of each device of the set equals to {\it
% acc\_device\_default}, and the size of the set equals to a result of
% {\it acc\_get\_num\_devices(acc\_device\_default)}.
% %In {\tt declare} directive with {\tt layout} clause or {\tt barrier_device} directives, it is assumed that the default device set is specified for {\tt on_device} clause.


% \subsubsection{Device Reference}

% \subsection*{Synopsis}
% The device reference is used to reference a device set.

% \subsubsection*{Syntax}
% \begin{tabular}{llll}
%              & {\it devices-ref} & {\bf is} & {\it devices-name} {\openb}\verb|(| {\it devices-subscript} \verb|)|{\closeb}\\
%   \verb![C]! & {\it devices-ref} & {\bf is} & {\it devices-name} {\openb}\verb|[| {\it devices-subscript} \verb|]|{\closeb}
% \end{tabular}

% \vspace{1em}
% where {\it devices-subscript} must be one of:
% \vspace{1em}

% \begin{tabular}{ll}
%  \hspace{0.5cm} & {\it int-expr} \\
%                 & {\it triplet}
% \end{tabular}

% \subsubsection*{Description}

% A device reference by {\it devices-name} represents a device set
% corresponding to the device array specified by the name or its
% subarray.
% %% A device reference by ``*'' represents the executing device set.

% \subsubsection*{Example}
% Assume that {\it d} is the name of a device array.
% \begin{itemize}
% \item To specify a device set to which the declared device array corresponds,\\

% \begin{myfigure}
% \begin{minipage}{0.43\hsize}
% \begin{center}
% \begin{XACCFexampleL}
% !$acc devices e(1) = d(1)
% !$acc devices f(3) = d(2:4)
% \end{XACCFexampleL}
% \end{center}
% \end{minipage}
% %
% \begin{minipage}{0.51\hsize}
% \begin{center}
% \begin{XACCCexampleR}
% #pragma acc devices e[1] = d[0]
% #pragma acc devices f[3] = d[1:3]
% \end{XACCCexampleR}
% \end{center}
% \end{minipage}
% %\caption{Code example in {\XACC} device reference}\label{code:device_ref}
% \end{myfigure}

% \item To specify a device array that corresponds to the executing device
% 	  array set in the {\tt barrier} directive.

% \begin{myfigure}
% \begin{minipage}{0.43\hsize}
% \begin{center}
% \begin{XACCFexampleL}
% !$acc barrier_device on_device(d)
% \end{XACCFexampleL}
% \end{center}
% \end{minipage}
% %
% \begin{minipage}{0.51\hsize}
% \begin{center}
% \begin{XACCCexampleR}
% #pragma acc barrier_device on_device(d)
% \end{XACCCexampleR}
% \end{center}
% \end{minipage}
% \end{myfigure}

% \end{itemize}


%%%%%%%%%%%%%%%%%%%%%%%%%%%%%%%%%%%%%%%%%%%%%
\subsubsection{{\tt on\_device} Clause}

%\subsection*{Synopsis}

% The {\tt on\_device} clause specifies a execution device set for the
% directive.

% \subsubsection*{Syntax}

% \begin{tabular}{llll}
%              & \verb|on_device(| {\it devices-ref} \verb|)|\\
% \end{tabular}

% \subsubsection*{Description}
The \|on_device| clause in a directive specifies a device set that the
directive targets.

The \|on_device| clause may appear on \|parallel|, \|parallel loop|,
\|kernels|, \|kernels loop|, \|data|, \|enter data|, 
\|exit data|, \|declare|, \|update|, \|wait|, and \|barrier_device|
directives.

The directive is applied to each device in the device set in parallel.
If there is no \|layout| clause, the all devices process the
directive for same data or work redundantly.

% If no {\tt on\_device} clause appears on a {\tt declare} directive with
% a {\tt layout} clause, it is assumed that the default device set is
% specified by {\tt on\_device} clause.
% If no {\tt on\_device} clause appears on a {\tt barrier\_device}
% directive, it is assumed that the default device set is specified by
% {\tt on\_device} clause.
% If no {\tt on\_device} clause appears on a {\tt data}, {\tt enter data},
% {\tt exit data}, or {\tt update} directives, if the arrays are alreadly
% declared by {\tt declare} directive, the device set that specified at
% the {\tt declare} directive is targeted.
% In the other cases, the directive behaves the same as normal {\OACC}.

%\section{Data and Work Mapping Clauses}

%%%%%%%%%%%%%%%%%%%%%%%%%%%%%%%%%%%%%%%%%%%%%%%%%%
\subsubsection{{\tt layout} Clause}

% \subsection*{Synopsis}

The {\tt layout} clause specifies data or work mapping on devices.

% \subsubsection*{Syntax}
% \vspace{1em}
% In {\tt declare} directive:
% \vspace{1em}

% \begin{tabular}{ll}
%   \verb![F]! & \verb|layout( (| {\it dist-format} {\openb}, {\it dist-format} {\closeb} ... \verb|) )|\\
%   \verb![C]! & \verb|layout( [| {\it dist-format} \verb|]| {\openb} \verb|[| {\it dist-format} \verb|]| {\closeb} ... \verb|)|
% \end{tabular}

% \vspace{1em}
% where {\it dist-format} must be one of:
% \vspace{1em}

% \begin{tabular}{ll}
%  \hspace{0.5cm} & {\tt *} \\
%                 & {\tt block}
% \end{tabular}

% \vspace{1em}
% In {\tt loop}, {\tt parallel loop}, and {\tt kernels loop} construct:
% \vspace{1em}

% \begin{tabular}{ll}
%   \verb![F]! & \verb|layout(| {\it array-name} \verb|(| {\it layout-subscript} {\openb}, {\it layout-subscript} {\closeb} ... \verb|) )|\\
%   \verb![C]! & \verb|layout(| {\it array-name} \verb|[| {\it layout-subscript} \verb|]| {\openb} \verb|[| {\it layout-subscript} \verb|]| {\closeb} ... \verb|)|
% \end{tabular}

% \vspace{1em}
% where {\it layout-subscript} must be one of:
% \vspace{1em}

% \begin{tabular}{ll}
%  \hspace{0.5cm} & {\it scalar-int-variable} {\openb} \{ {\tt +} $\vert$ {\tt -} \} {\it int-expr} {\closeb}\\
%                 & {\tt *}
% \end{tabular}


% \subsubsection*{Description}

The {\tt layout} clause may appear on {\tt declare} directives and on
{\tt loop}, {\tt parallel loop}, and {\tt kernels loop} constructs.
If the {\tt layout} clause appears on a {\tt declare} directive, it
specifies the data mapping to the device set for arrays which are
appeared in data clauses on the directive.
``{\tt *}'' represents that the dimension is not distributed, and {\tt
block} represents that the dimension is divided into contiguous blocks,
which are distributed onto the device array.

% If the {\tt layout} clause appears on a {\tt loop}, {\tt parallel loop},
% or {\tt kernels loop} directive, it specifies the mapping for the
% immediately following loop.
% If {\it loop-index} appears in {\it layout-subscript}, the loop is
% distributed to the device set in the same manner as the dimension where
% the {\it loop-index} appears.
% If there is no {\tt on\_device} clause on the construct, it is assumed
% that the device set on which the array is distributed is specified by
% {\tt on\_device} clause.

% \subsubsection*{Restriction}

% \begin{itemize}
% \item {\it loop-index} must be a control variable of a loop. % in the associated loop nest.
% \end{itemize}

\subsubsection*{Example}

The following are examples of the \|layout| clause.
In line 2, the \|devices| directive defines a device set \|d|.
In lines 3-4, the \|declare| directive declares that an array \|a|
is distributed and allocated on \|d|.
In lines 6--9, the \|kernels loop| directive distributes and offloads
the following loops to \|d|.
%
\begin{myfigure}
\begin{minipage}{0.47\hsize}
\begin{center}
\begin{XACCFexampleL}
integer :: a(N)
!$acc devices d(*)
!$acc declare create(a)
!$acc+layout((block)) on_device(d)
...
!$acc kernels loop layout(a(i))
do i = 1, N
  a(i) = i * 2
end do
\end{XACCFexampleL}
\end{center}
\end{minipage}
%
\begin{minipage}{0.51\hsize}
\begin{center}
\begin{XACCCexampleR}
int a[N];
#pragma acc devices d[*]
#pragma acc declare create(a) \
        layout([block]) on_device(d)
...
#pragma acc kernels loop layout(a[i])
for(int i = 0; i < N; i++){
  a[i] = i * 2;
}
\end{XACCCexampleR}
\end{center}
\end{minipage}
\caption{Xacc Code example with {\tt layout} clause.}\label{code:layout_clause}
\end{myfigure}


%%%%%%%%%%%%%%%%%%%%%%%%%%%%%%%%%%%%%%%
\subsubsection{{\tt shadow} Clause}

%\subsection*{Synopsis}

% The {\tt shadow} clause allocates the shadow area for a distributed
% array on devices.

% \subsubsection*{Syntax}
% \begin{tabular}{ll}
%   \verb![F]! & \verb|shadow( (| {\it shadow-width} {\openb}, {\it shadow-width} {\closeb} ... \verb|) )|\\
%   \verb![C]! & \verb|shadow( [| {\it shadow-width} \verb|]| {\openb} \verb|[| {\it shadow-width} \verb|]| {\closeb} ... \verb|)|
% \end{tabular}

% \vspace{1em}
% where {\it shadow-width} must be one of:
% \vspace{1em}

% \begin{tabular}{ll}
%  \hspace{0.5cm} & {\it int-expr} \\
%                 & {\it int-expr} {\tt :} {\it int-expr}
% %                & {\tt *}
% \end{tabular}

% \subsubsection*{Description}

The {\tt shadow} clause in the {\tt declare} directive specifies the
width of the shadow area of arrays, which is used to communicate the
neighbor element of the block of the arrays.
%
% When {\it shadow-width} is of the form ``{\it int-expr} : {\it int-expr},''
% the shadow area of the width specified by the first {\it int-expr} is
% added at the lower bound and that specified by the second
% one at the upper bound in the dimension.
% %
% When {\it shadow-width} is of the form {\it int-expr}, the shadow
% area of the same width specified is added at both the upper and lower
% bounds in the dimension.
%

% \subsubsection*{Restriction}
% \begin{itemize}
% \item {\tt shadow} clause must appear with {\tt layout} clause.
% \item The value specified by {\it shadow-width} must be a non-negative integer.
% \item The number of {\it shadow-width} must be equal to the number of
% 	  dimensions (or rank) of the arrays on the {\tt declare}
% 	  directive.
% \item If an array is also distributed on {\bf nodes}, a {\it
% 	  shadow-width} of {\tt shadow} clause must be same as the {\it
% 	  shadow-width} of {\XMP} {\tt shadow} directive for the same
% 	  dimension.
% \end{itemize}

\subsubsection*{Example}

The following are examples of the \|shadow| clause.
In line 2, the \|devices| directive defines a device set \|d|.
In lines 3-5, the \|declare| directive declares that an array \|a|
is distributed and allocated with shadow areas on the device set \|d|.
In lines 7--10, the \|kernels loop| construct divides and offloads the
loop to the device set \|d|.
In line 11, the \|reflect| directive updates the shadow areas of the
distributed array \|a| on the device memory.
%
\begin{myfigure}
\begin{minipage}{0.47\hsize}
\begin{center}
\begin{XACCFexampleL}
integer :: a(N)
!$acc devices d(*)
!$acc declare create(a)
!$acc+layout((block))
!$acc+shadow((1:1)) on_device(d)
...
!$acc kernels loop layout(a(i))
do i = 1, N
  a(i) = i * 3
end do
!$acc reflect(a)
\end{XACCFexampleL}
\end{center}
\end{minipage}
%
\begin{minipage}{0.51\hsize}
\begin{center}
\begin{XACCCexampleR}
int a[N];
#pragma acc devices d[*]
#pragma acc declare create(a) \
        layout([block]) \
        shadow([1:1]) on_device(d)
...
#pragma acc kernels loop layout(a[i])
for(int i = 0; i < N; i++){
  a[i] = i * 3;
}
#pragma acc reflect(a)
\end{XACCCexampleR}
\end{center}
\end{minipage}
\caption{XACC code with {\tt shadow} clause.}\label{code:shadow_clause}
\end{myfigure}


%\section{Synchronization on Accelerators}

%%%%%%%%%%%%%%%%%%%%%%%%%%%%%%%%%%%%%%%%%%%%%%%%%%%%%%%
\subsubsection{{\tt barrier\_device} Construct}

% \subsection*{Synopsis}

The \|barrier_device| construct specifies an explicit barrier among
devices at the point which the construct appears.

% \subsubsection*{Syntax}
% \begin{tabular}{ll}
%   \verb![F]! & \verb|!$acc barrier_device|       {\openb}\verb|on_device(| {\it devices-ref} \verb|)|{\closeb}\\
%   \verb![C]! & \verb|#pragma acc barrier_device| {\openb}\verb|on_device(| {\it devices-ref} \verb|)|{\closeb}
% \end{tabular}

% \subsubsection*{Description}

The \|barrier_device| construct blocks accelerator devices until all
ongoing asynchronous operations on them are completed regardless of the
host operations. 
% The construct is performed among the device set specified by the
% \|on_device| clause.
% If no \|on_device| clause is specified, then it is assumed that the
% default device set is specified in it.

% \subsubsection*{Restriction}
% \begin{itemize}
% \item This construct must not appear in {\OACC} {\bf compute region}.
% \end{itemize}

\subsubsection*{Example}

The following are examples of the \|barrier_devices| construct.
In lines 1--2, the \|devices| directives define device sets \|d| and
\|e|.
In lines 4--5, the first \|barrier_device| construct performs a
barrier operation for all devices, 
and the second one performs a barrier operation for devices in the
device set \|e|.
%
\begin{myfigure}
\begin{minipage}{0.455\hsize}
\begin{center}
\begin{XACCFexampleL}
!$acc devices d(*)
!$acc devices e(2) = d(1:2)
...  
!$acc barrier_device
!$acc barrier_device on_device(e)  
\end{XACCFexampleL}
\end{center}
\end{minipage}
%
\begin{minipage}{0.535\hsize}
\begin{center}
\begin{XACCCexampleR}
#pragma acc devices d[*]
#pragma acc devices e[2] = d[0:2]
...
#pragma acc barrier_device
#pragma acc barrier_device on_device(e)
\end{XACCCexampleR}
\end{center}
\end{minipage}
\caption{XACC Code with {\tt barrier\_device}
  construct.}\label{code:barrier_device}
\end{myfigure}
 \cleardoublepage
\section{Omni XcalableACC Compiler}
We also develop Omni Compiler as a reference implementation for XACC.
Omni Compiler is a source-to-source compiler that translates the base language (C or Fortran) and XACC/XMP directives into runtime calls.

\begin{figure}[!t]
\centering
\includegraphics[scale=0.94,clip]{figs/flow2.eps}
\caption{Compile flow of Omni Compiler}
\label{fig:flow}
\end{figure}

Figure \ref{fig:flow} shows the compile flow of Omni Compiler.
First, 
Omni Compiler translates XACC/XMP directives present in the user code into runtime calls. 
Moreover, a part of the base language and OpenACC directives are modified if necessary.
Next, the translated code is compiled by a native compiler to generate an object file.
The native compiler must support OpenACC.
Finally, the object file and the runtime are linked by the native compiler to generate an execution file.
For details on how Omni Compiler translates the user code, please refer to section V of \cite{nakao2014}.

%In order to transfer data between NVIDIA GPUs across compute nodes,
%we implement the following three methods for Omni Compiler.
%(1) Using TCA/IB hybrid communication.
%(2) Using GPUDirect RDMA with CUDA-Aware MPI.
%(3) Using MPI and CUDA.
%Item (1) performs communication with the smallest latency, but a TCA system is required for the computing environment.
%Item (2) is superior in performance to Item (3), but Item (2) also requires specific software and hardware (e.g., MVAPICH2-GDR and Mellanox InfiniBand).
%Whereas Items (1) and (2) can realize direct communication between GPUs without the CPU, 
%Item (3) copies the data from the accelerator memory to the host memory using CUDA and then transfers the data to other compute nodes using MPI.
%Therefore, although Item (3) has the lowest performance, it does not require specific software and hardware. \cleardoublepage
\section{Performance of Lattice QCD Application}

This section describes the evaluations of XACC performance and productivity for a lattice quantum chromodynamics (Lattice QCD) application.

\subsection{Overview of Lattice QCD}
The Lattice QCD is a discrete formulation of QCD that describes the strong interaction among ``quarks'' and ``gluons.''
While the quark is a species of elementary particles, the gluon is a particle that mediates the strong interaction.
The Lattice QCD is formulated on a four-dimensional lattice (time: T and space:ZYX axes).
We impose a periodic boundary condition in all the directions.
The quark degree of freedom is represented as a field that has four components of ``spin'' and three components of ``color,''
namely a complex vector of $12 \times N_{site}$ components,
where $N_{site}$ is the number of lattice sites.
The gluon is defined as a $3\times 3$ complex matrix field on links (bonds between neighboring lattice sites).
During a Lattice QCD simulation,
one needs to solve many times a linear equation for the matrix that represents the interaction between the quark and gluon fields.
This linear equation is the target of present work.
The matrix acts on the quark vector and has nonzero components only for the neighboring sites, and thus sparse.

\subsection{Implementation}
We implemented a Lattice QCD code based on the existing Lattice QCD application Bridge++\cite{bridge++}.
Since our code was implemented by extracting the main kernel of the Bridge++,
it can be used as a mini-application to investigate its productivity and performance more easily than
use of the original Bridge++.

\begin{figure}[h]
\centering
\begin{minipage}{0.45\hsize}
\begin{lstlisting}
S = B
R = B
X = B
sr = norm(S)
T = WD(U,X)
S = WD(U,T)
R = R - S
P = R
rr = norm(R)
rrp = rr
do{
\end{lstlisting}
\end{minipage} \hspace{0.5cm}
\begin{minipage}{0.45\hsize}
\begin{lstlisting}[firstnumber=10]
  T = WD(U,P)
  V = WD(U,T)
  pap = dot(V,P)
  cr = rr/pap
  X = cr * P + X
  R = -cr * V + R
  rr = norm(R)
  bk = rr/rrp
  P = bk * P + R
  rrp = rr
}while(rr/sr > 1.E-16)
\end{lstlisting}
\end{minipage}
\caption{Lattice QCD pseudo code}\label{fig:pseudo}
\end{figure}

Fig. \ref{fig:pseudo} shows a pseudo code of the implementation,
where the CG method is used to solve quark propagators.
In Fig. \ref{fig:pseudo},
{\it WD()} is the Wilson-Dirac operator\cite{PhysRevD.10.2445},
{\it U} is a gluon, the other uppercase characters are quarks.
The Wilson-Dirac operator is a main kernel in the Lattice QCD,
which calculates how the quarks interact with each other under the influence of the gluon.

\begin{figure}[h]
\centering
\begin{lstlisting}
typedef struct Quark {
 double v[4][3][2];
} Quark_t;
typedef struct Gluon {
 double v[3][3][2];
} Gluon_t;
Quark_t v[NT][NZ][NY][NX], tmp_v[NT][NZ][NY][NX];
Gluon_t u[4][NT][NZ][NY][NX];

#pragma xmp template t[NT][NZ]
#pragma xmp nodes n[NODES_T][NODES_Z]
#pragma xmp distribute t[block][block] onto n
#pragma xmp align v[i][j][*][*] with t[i][j]
#pragma xmp align tmp_v[i][j][*][*] with t[i][j]
#pragma xmp align u[*][i][j][*][*] with t[i][j]
#pragma xmp shadow v[1:1][1:1][0][0]
#pragma xmp shadow tmp_v[1:1][1:1][0][0]
#pragma xmp shadow u[0][1:1][1:1][0][0]
...
int main(){
...
#pragma acc enter data copyin(v, tmp_v, u)
\end{lstlisting}
\caption{Declaration of distributed arrays for Lattice QCD}\label{fig:distributedQCD}
\end{figure}

Fig. \ref{fig:distributedQCD} shows how to declare distributed arrays of the quark and gluon.
In lines 1-8, the quark and gluon structure arrays are declared.
The last dimension ``[2]'' of both structures represents real and imaginary parts for a complex number.
{\it NT}, {\it NZ},  {\it NY} and {\it NX} are the numbers of TZYX axis elements.
In lines 10-18, distributed arrays are declared where the macro constant values {\it NODES\_T} and {\it NODES\_Z} indicate the number of {\tt nodes} on the T and Z axes.
Thus,
the program is parallelized on T and Z axes.
Note that an ``*'' in the {\bf align} directive means that the dimension is not divided.
In the {\bf shadow} directive,
halo regions are added to the arrays because each quark and gluon element is affected by its neighboring orthogonal elements.
Note that ``0'' in the {\bf shadow} directive means that no halo region exists.
In line 22,
the {\bf enter data} directive transfers the distributed arrays from host memory to accelerator memory.

\begin{figure}[h]
\centering
\begin{lstlisting}
#pragma xmp reflect(v) width(/periodic/1:1,/periodic/1:1,0,0) orthogonal acc
#pragma xmp reflect(u) width(0,/periodic/1:0,/periodic/1:0,0,0) orthogonal acc
WD(tmp_v, u, v);
#pragma xmp reflect(tmp_v) width(/periodic/1:1,/periodic/1:1,0,0) orthogonal acc
WD(v, u, tmp_v);
\end{lstlisting}
\caption{Calling Wilson-Dirac operator}\label{fig:callingDirac}
\end{figure}

Fig. \ref{fig:callingDirac} shows how to call {\it WD()}.
The  {\bf reflect} directives are inserted before {\it WD()} in order to update own halo region.
In line 2,
``1:0'' in {\bf width} clause means only the lower halo region is updated because only it is needed in {\it WD()}.
The {\it u} is not updated before the second {\it WD()} function because values of {\it u} are not updated in {\it WD()}.
Moreover,
the {\bf orthogonal} clause is added because diagonal updates of the arrays are not required in {\it WD()}.

\begin{figure}[h]
\centering
\begin{lstlisting}
void WD(Quark_t v_out[NT][NZ][NY][NX], const Gluon_t u[4][NT][NZ][NY][NX], const Quark_t v[NT][NZ][NY][NX])
{
#pragma xmp align v_out[i][j][*][*] with t[i][j]
#pragma xmp align u[*][i][j][*][*] with t[i][j]
#pragma xmp align v[i][j][*][*] with t[i][j]
#pragma xmp shadow v_out[1:1][1:1][0][0]
#pragma xmp shadow u[0][1:1][1:1][0][0]
#pragma xmp shadow v[1:1][1:1][0][0]
 ...
#pragma xmp loop (t,z) on t[t][z]
#pragma acc parallel loop collapse(4) present(v_out, u, v)
 for(int t=0;t<NT;t++)
  for(int z=0;z<NZ;z++)
   for(int y=0;y<NY;y++)
    for(int x=0;x<NX;x++){
\end{lstlisting}
\caption{A portion of Wilson-Dirac operator}\label{fig:dirac}
\end{figure}

Fig. \ref{fig:dirac} shows a part of the Wilson-Dirac operator code.
All arguments in {\it WD()} are distributed arrays.
In XMP and XACC,
distributed arrays which are used as arguments must be redeclared in function to pass their information to a compiler.
Thus, the {\bf align} and {\bf shadow} directives are used in {\it WD()}.
In line 10, the {\bf loop} directive parallelizes the outer two loop statements.
In line 11, the {\bf parallel loop} directive parallelizes all loop statements.
In the loop statements,
a calculation needs neighboring and orthogonal elements.
Note that while the {\it WD()} updates only the {\it v\_out},
it only refers the {\it u} and {\it v}.

\begin{figure}[h]
\centering
\begin{lstlisting}
double norm(const Quark_t v[NT][NZ][NY][NX])
{
#pragma xmp align v[i][j][*][*] with t[i][j]
#pragma xmp shadow v[1:1][1:1][0][0]
 double a = 0.0;

#pragma xmp loop (t,z) on t[t][z]
#pragma acc parallel loop collapse(7) present(v) reduction(+:a)
 for(int t=0;t<NT;t++)
  for(int z=0;z<NZ;z++)
   for(int y=0;y<NY;y++)
    for(int x=0;x<NX;x++)
     for(int i=0;i<4;i++)
      for(int j=0;j<3;j++)
       for(int k=0;k<2;k++)
        a += v[t][z][y][x].v[i][j][k]*v[t][z][y][x].v[i][j][k];

#pragma xmp reduction (+:a)
 return a;
}
\end{lstlisting}
\caption{L2 norm calculation code}\label{fig:norm}
\end{figure}

Fig. \ref{fig:norm} shows L2 norm calculation code in the CG method.
In line 8,
the {\bf reduction} clause performs a reduction operation for the variable {\it a} in each accelerator when finishing the next loop statement.
The calculated variable {\it a} is located in both host memory and accelerator memory.
However, at this point,
all {\tt nodes} have individual values of {\it a}.
To obtain the total value of the variable {\it a},
the XMP {\bf reduction} directive in line 18 also performs a reduction operation among {\tt nodes}.
Since the total value is used on only host after this function,
the XMP {\bf reduction} directive does not have {\bf acc} clause.

\section{Performance Evaluation}\label{sec:performance}
\subsection{Result}
\begin{table}[h]
\renewcommand{\arraystretch}{1.2}
\centering
\caption{Evaluation environment} \label{tab:ha-pacs/tca}
\begin{tabular}{l|l}\hline
CPU & Intel Xeon-E5 2680v2 2.8 GHz x 2 Sockets \\
Memory & DDR3 1866MHz 59.7GB/s 128GB \\
GPU & NVIDIA Tesla K20X (GDDR5 250GB/s 6GB) x 4 GPUs \\
Network & InfiniBand Mellanox Connect-X3 Dual-port QDR 8GB/s \\
\multirow{2}{*}{Software} & Intel 16.0.2, CUDA 7.5.18, Omni OpenACC compiler 1.1\\
 & MVAPICH2 2.1\\ \hline
\end{tabular}
\end{table}

This section evaluates the performance level of XACC on the Lattice QCD code.
For comparison purposes,
those of MPI+CUDA and MPI+OpenACC are also evaluated.
For performance evaluation,
we use the HA-PACS/TCA system\cite{hapacs} the hardware specifications and software environments of which are shown in Table \ref{tab:ha-pacs/tca}.
Since each compute node has four GPUs,
we assign four {\tt nodes} per compute node and direct each {\tt node} to deal with a single GPU.
We use the Omni OpenACC compiler\cite{2013tabuchi} as a backend compiler in the Omni XACC compiler.
We execute the Lattice QCD codes with strong scaling in regions (32,32,32,32) as ({\it NT},{\it NZ},{\it NY},{\it NX}).
The Omni XACC compiler provides various types of data communication among accelerators\cite{nakao2014}.
We use the MPI+CUDA implementation type because it provides a balance of versatility and performance.

\begin{figure}[h]
\centering
\includegraphics[scale=0.58,clip]{figs/performance32.eps}
\caption{Performance results} \label{fig:performance}
\end{figure}

Fig. \ref{fig:performance} shows the performance results that indicate the time required to solve one CG iteration as well as the performance ratio values that indicate the comparative performance \
of XACC and other languages.
When the performance ratio value of a language is greater than 1.00,
the performance result of the language is better than that of XACC.
Fig. \ref{fig:performance} shows that the performance ratio values of MPI+CUDA are between 1.04 and 1.18,
and that those of MPI+OpenACC are between 0.99 and 1.04.
Moreover,
Fig. \ref{fig:performance} also shows that the performance results of both MPI+CUDA and MPI+OpenACC become closer to those of XACC as the number of {\tt nodes} increases.

\subsection{Discussion}
To examine the performance levels in detail,
we measure the time required for the halo updating operation for two {\tt nodes} and more.
The halo updating operation consists of the communication and pack/unpack processes for non-contiguous regions in the XACC runtime.

\begin{figure}[h]
\centering
\includegraphics[scale=0.58,clip]{figs/halo-comm.eps}
\caption{Communication time} \label{fig:halo-comm}
\end{figure}

\begin{figure}[h]
\centering
\includegraphics[scale=0.58,clip]{figs/halo-pack-unpack.eps}
\caption{Pack/unpack time} \label{fig:halo-pack-unpack}
\end{figure}

While Fig. \ref{fig:halo-comm} describes communication time of the halo updating time of Fig \ref{fig:performance},
Fig. \ref{fig:halo-pack-unpack} describes pack/unpack time of it.
Fig. \ref{fig:halo-comm} shows that
the communication performance levels of all implementations are almost the same.
However,
Fig. \ref{fig:halo-pack-unpack} shows that the pack/unpack performance levels of MPI+CUDA are better than those of XACC,
and that those of MPI+OpenACC are worse than those of XACC.
The reason for the pack/unpack operation performance level difference is that the XACC operation is implemented in CUDA at XACC runtime.
Thus,
some performance levels of XACC are better than those of MPI+OpenACC in Fig. \ref{fig:performance}.
However,
the performance levels of XACC in Fig. \ref{fig:halo-pack-unpack} is worse than those of MPI+CUDA because XACC requires the cost of XACC runtime calls.

\begin{figure}[h]
\centering
\includegraphics[scale=0.58,clip]{figs/exceptforhalo.eps}
\caption{Time excluding halo updating time} \label{fig:exceptforhalo}
\end{figure}

Fig. \ref{fig:exceptforhalo} shows the overall time excluding the halo updating time,
where performance levels of MPI+CUDA are the best, and those of XACC are almost the same as those of MPI+OpenACC.
The reason for the difference is due to how to use GPU threads.
In the CUDA implementation,
we assign loop iterations to GPU threads in a cyclic-manner manually.
In contrast, in the OpenACC and XACC implementations,
how to assign GPU threads is an implementation dependent of an OpenACC compiler.
In the Omni OpenACC compiler,
initially loop iterations are assigned to GPU threads by a gang (threadblock) in a block manner,
and then are also assigned to them by a vector (thread) in a cyclic manner.
With the {\bf gang} clause with the {\bf static} argument proposed in the OpenACC specification version 2.0,
programmers can determine how to use GPU threads to some extent,
but the Omni OpenACC compiler does not yet support it.

\begin{figure}[h]
\centering
\includegraphics[scale=0.58,clip]{figs/exceptforhalo-per.eps}
\caption{Updating halo ratio} \label{fig:exceptforhalo-per}
\end{figure}

Fig. \ref{fig:exceptforhalo-per} shows the ratio of the halo updating time to overall time.
As can be seen,
as the number of {\tt nodes} increases, the ratio increases as well.
Therefore,
when a large number of {\tt nodes} are used,
there is little difference in performance level of Fig. \ref{fig:performance} among the three implementations.
The reason why the ratio of MPI+CUDA is slightly larger than those of the others is that
the time excluding the halo communication of MPI+CUDA in Fig. \ref{fig:halo-comm} is relatively small.

\section{Productivity Improvement}
\subsection{Requirement for Productive Parallel Language}
\begin{figure}[h]
\centering
\includegraphics[scale=0.75,clip]{figs/howtocreate1.eps}
\caption{Application development order on accelerated cluster} \label{fig:howtocreate}
\end{figure}

In Section \ref{sec:performance},
we developed three Lattice QCD codes using MPI+CUDA, MPI+OpenACC, and XACC.
Fig. \ref{fig:howtocreate} shows our procedure for developing each code where
we first develop the code for an accelerator from the serial code,
and then extend it to handle an accelerated cluster.

To parallelize the serial code for an accelerator using CUDA requires large code changes (``a'' in Fig. \ref{fig:howtocreate}),
most of which are necessary to create new kernel functions and to make 1D arrays out of multi-dimensional arrays.
By contrast,
OpenACC accomplishes the same parallelization with just small code changes (``b''),
because OpenACC's directive-based approach encourages reuse of an existing code.
Besides,
to parallelize the code for a distributed memory system,
MPI also requires large changes (``c'' and ``d''),
primarily to convert global indices into local indices.
By contrast,
XACC requires smaller code changes (``e'') because XACC is also directive-based language as OpenACC.

In many cases,
a parallel code for an accelerated cluster is based on an existing serial code.
The code changes to the existing serial code are likely to trigger program bugs.
Therefore,
XACC is designed to reuse an existing code as possible.

\subsection{Quantitative Evaluation by Delta Source Lines of Codes}
\begin{table}[h]
\renewcommand{\arraystretch}{1.2}
\centering
\caption{DSLOC of Lattice QCD implementations} \label{tab:dsloc}
\begin{tabular}[h]{r|rrrrr|rrr} \hline
       & a   & b  & c   & d   & e   & a+c & b+d & {\bf b+e}  \\ \hline
DSLOC  & 552 & 22 & 280 & 201 & 138 & 832 & 223 & {\bf 160} \\  \hline
Add    & 137 & 20 & 185 & 140 & 134 & 322 & 160 & {\bf 154} \\
Delete & 73 & 0 & 0 & 0 & 0 & 73 & 0 & {\bf 0} \\
Modify & 342 & 2 & 95 & 61 & 4 & 437 & 63 & {\bf 6} \\ \hline
\end{tabular}
\end{table}

As one of metrics for productivity,
Delta Source Lines of Codes (DSLOC) is proposed\cite{CGPOP2011}.
The DSLOC indicates how the codes change from a corresponding implementation.
The DSLOC is the sum of three components: how many lines are added, deleted and modified.
When the DSLOC is small,
the programming costs and the possibility of program bugs will be small as well.
We use the DSLOC to count the amount of change required to implement an accelerated cluster code from a serial code.

Table \ref{tab:dsloc} shows the DSLOC where lowercase characters correspond to Fig. \ref{fig:howtocreate}.
The DSLOC of XACC (b+e) is smaller than MPI+CUDA (a+c) and MPI+OpenACC (b+d).
The difference between XACC and MPI+CUDA is 420.0\%, and that between XACC and MPI+OpenACC is 39.4\%.

\subsection{Discussion}\label{sec:pro-con}
\begin{figure}[h]
\centering
\begin{lstlisting}
#pragma acc parallel loop collapse(4) present(v_out, u, v)
 for(int t=0;t<NT;t++)
  for(int z=0;z<NZ;z++)
   for(int y=0;y<NY;y++)
    for(int x=0;x<NX;x++){
     int tt = (t + 1) % NT;
     v_out[tt][z][y][x].v[0][0][0] = ... ;
\end{lstlisting}\vspace{-1.5ex}
\caption{Code modification of {\it WD()} in OpenACC}\label{fig:modification-openacc}
\end{figure}

\begin{figure}[h]
\centering
\begin{lstlisting}
#pragma xmp loop (t,z) on t[t][z]
#pragma acc parallel loop collapse(4) present(v_out, u, v)
 for(int t=0;t<NT;t++)
  for(int z=0;z<NZ;z++)
   for(int y=0;y<NY;y++)
    for(int x=0;x<NX;x++){
     int tt = t + 1;
     v_out[tt][z][y][x].v[0][0][0] = ... ;
\end{lstlisting}\vspace{-1.5ex}
\caption{Code modification of {\it WD()} in XcalableACC}\label{fig:modification-xacc}
%\caption{Code modification in {\it WD()}} \label{fig:modification}
\end{figure}

In ``e'' of Table \ref{tab:dsloc},
four lines for modification are required to implement the XACC code from the OpenACC code.
Fig. \ref{fig:modification-openacc} and \ref{fig:modification-xacc} show the modification,
which is a part of {\it WD()} of Fig. \ref{fig:dirac}.
A variable {\it tt} is used to be an index for halo region.
The {\it tt} is modified from line 6 of Fig. \ref{fig:modification-openacc} to line 7 of Fig. \ref{fig:modification-xacc}.
In Fig. \ref{fig:modification-openacc},
when a value of a variable {\it t} is ``{\it NT}-1'',
that of the variable {\it tt} becomes ``0'' which is the lower bound index of the first dimension of the array {\it v\_out}.
On the other hand,
in Fig. \ref{fig:modification-xacc},
communication of the halo is performed before execution of {\it WD()} by the {\bf reflect} directive shown in Fig. \ref{fig:callingDirac}.
Thus,
the variable {\it tt} need only be incremented in Fig. \ref{fig:modification-xacc}.
There are four such modifications in {\it WD()}.
Note that
XACC does not keep the semantics of the base code perfectly in this case
in exchange for simplified parallelization.
In addition,
there are two lines for modification shown in ``b'' of Table \ref{tab:dsloc}.
It is a very fine modification for OpenACC constraints,
which keeps the semantics of the base code.

\begin{table}[h]
\renewcommand{\arraystretch}{1.2}
\centering
\caption{SLOC of Lattice QCD implementations} \label{tab:sloc}
\begin{tabular}[h]{r|rrr} \hline
               & MPI+CUDA & MPI+OpenACC & {\bf XcalableACC}  \\ \hline
SLOC           & 1091     & 1002        & {\bf 996} \\ \hline
\#XcalableMP   & -        & -           & {\bf 122} \\
\#OpenACC      & -        & 26          & {\bf 16}  \\
\#XcalableACC  & -        & -           & {\bf 3}   \\
\#MPI function & 39       & 39          & {\bf -}   \\ \hline
\end{tabular}
\end{table}

\begin{figure}[h]
\centering
\begin{lstlisting}
void WD(Quark_t v_out[NT][NZ][NY][NX], const Gluon_t u[4][NT][NZ][NY][NX], const Quark_t v[NT][NZ][NY][NX])
{
#pragma xmp align [i][j][*][*] with t[i][j] shadow [1:1][1:1][0][0] :: v_out, v
#pragma xmp align [*][i][j][*][*] with t[i][j] shadow [0][1:1][1:1][0][0] :: u
\end{lstlisting}
\caption{New directive combination syntax that applies to Fig. \ref{fig:dirac}}\label{fig:combine}
\end{figure}

As basic information,
we count the source lines of codes (SLOC) of each of the Lattice QCD implementations.
Table \ref{tab:sloc} shows the SLOC excluding comments and blank lines,
as well as the numbers of each directive and MPI functions included in their SLOC.
For reader information,
SLOC of the serial version Lattice QCD code is 842.
Table \ref{tab:sloc} shows that
the 122 XMP directives are used in the XACC implementation,
many of which are declarations for function arguments.
To reduce the XMP directives,
we are planning to develop a new syntax that combines declarations with the same attribute into one directive.
Fig. \ref{fig:combine} shows an example of the new syntax applied to the declarations in Fig. \ref{fig:dirac}.
Since the arrays {\it v\_out} and {\it v} have the same attribute,
they can be declared into a single XMP directive.
Moreover,
the {\bf shadow} directive attribute is added to the {\bf align} directive as its clause.
When applying the new directive to XACC implementation,
the number of XMP directives decreases from the 122 shown in Table \ref{tab:sloc} to 64,
and the XACC DSLOC decreases from the 160 shown in Table \ref{tab:dsloc} to 102.
 \cleardoublepage
%\input{glossary.tex}

% \cleardoublepage
% \chapter*{Acknowledgment}
% %\begin{itemize}
% %\setlength{\itemsep}{-1mm}
% %\item Taisuke Boku       \dotfill \ University of Tsukuba
% %\item Hidetoshi Iwashita \dotfill \ RIKEN/Fujitsu Inc.
% %\item Hitoshi Murai      \dotfill \ RIKEN
% %\item Masahiro Nakao     \dotfill \ RIKEN
% %\item Mitsuhisa Sato     \dotfill \ RIKEN
% %\item Akihiro Tabuchi     \dotfill \ University of Tsukuba
% %\end{itemize}

% The work was supported by the Japan Science and Technology Agency, 
% Core Research for Evolutional Science and Technology program entitled 
% ``Research and Development on Unified Environment of Accelerated Computing and Interconnection for Post-Petascale Era'' 
% in the research area of ``Development of System Software Technologies for Post-Peta Scale High Performance Computing.''

%The specification of {\XMP} is designed by the {\XMP} Specification
%Working Group, which consists of the following members from academia,
%research laboratories, and industries.

\begin{thebibliography}{99}
\addcontentsline{toc}{chapter}{\bibname}
\bibitem{aaa} Masahiro Nakao et al.
 ``Evaluation of XcalableACC with Tightly Coupled Accelerators/InfiniBand Hybrid Communication on Accelerated Cluster'', International Journal of High Performance Computing Applications (2019)
 \bibitem{xmp} XcalableMP Language Specification, \url{http://xcalablemp.org/specification.html} (2017).
 \bibitem{openacc} The OpenACC Application Programming Interface, \url{http://www.openacc.org} (2015).
 \bibitem{mpi} MPI: A Message-Passing Interface Standard, \url{http://mpi-forum.org} (2015).
 \bibitem{bridge++} Lattice QCD code Bridge++, \url{http://bridge.kek.jp/Lattice-code/index\_e.html}.
 \bibitem{PhysRevD.10.2445} Wilson, K. G., ``Confinement of quarks'' (1974).
 \bibitem{hapacs} HA-PACS, \url{https://www.ccs.tsukuba.ac.jp/supercomputer/}.
 \bibitem{2013tabuchi} Akihiro Tabuchi et al, ``A Source-to-Source OpenACC Compiler for CUDA'', Euro-Par Workhops (2013)
 \bibitem{nakao2014} Masahiro Nakao et al. ``XcalableACC: Extension of XcalableMP PGAS Language Using OpenACC for Accelerator Clusters'',
   Proceedings of the First Workshop on Accelerator Programming Using Directives (2014)
 \bibitem{CGPOP2011} Andrew I. Stone et al. ``Evaluating Coarray Fortran with the CGPOP Miniapp'',
   Proceedings of the Fifth Conference on Partitioned Global Address Space Programming Models (2011)
\end{thebibliography}

\end{document}
